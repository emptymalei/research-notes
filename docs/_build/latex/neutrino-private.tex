%% Generated by Sphinx.
\def\sphinxdocclass{puthesis}
\documentclass[letterpaper,12pt,english]{sphinxmanual}
\ifdefined\pdfpxdimen
   \let\sphinxpxdimen\pdfpxdimen\else\newdimen\sphinxpxdimen
\fi \sphinxpxdimen=49336sp\relax

\usepackage[margin=1in,marginparwidth=0.5in]{geometry}
\usepackage[utf8]{inputenc}
\ifdefined\DeclareUnicodeCharacter
  \DeclareUnicodeCharacter{00A0}{\nobreakspace}
\fi
\usepackage{cmap}
\usepackage[T1]{fontenc}
\usepackage{amsmath,amssymb,amstext}
\usepackage{babel}
\usepackage{times}
\usepackage[Bjarne]{fncychap}
\usepackage{longtable}
\usepackage{sphinx}

\usepackage{multirow}
\usepackage{eqparbox}

% Include hyperref last.
\usepackage{hyperref}
% Fix anchor placement for figures with captions.
\usepackage{hypcap}% it must be loaded after hyperref.
% Set up styles of URL: it should be placed after hyperref.
\urlstyle{same}
\addto\captionsenglish{\renewcommand{\contentsname}{Table of Contents}}

\addto\captionsenglish{\renewcommand{\figurename}{Fig.}}
\addto\captionsenglish{\renewcommand{\tablename}{Table}}
\addto\captionsenglish{\renewcommand{\literalblockname}{Listing}}

\addto\extrasenglish{\def\pageautorefname{page}}

\setcounter{tocdepth}{2}



\title{Neutrino Private Notebook}
\date{Mar 20, 2017}
\release{0.1.4}
\author{Lei Ma}
\newcommand{\sphinxlogo}{}
\renewcommand{\releasename}{Release}
\makeindex

\begin{document}

\maketitle
\sphinxtableofcontents
\phantomsection\label{\detokenize{index::doc}}


Neutrino is one of the most interesting particles in our world. The first proposal of such a new particle was given by Pauli. He managed to explain the spectrum of beta decay. In 1956, neutrinos were first detected in Cowan\textendash{}Reines neutrino experiment. %
\begin{footnote}[1]\sphinxAtStartFootnote
\href{https://en.wikipedia.org/wiki/Cowan\%E2\%80\%93Reines\_neutrino\_experiment}{Cowan\textendash{}Reines neutrino experiment}
%
\end{footnote} Later on a lot of neutrino experiments have been carried out.
\begin{figure}[htbp]
\centering
\capstart

\noindent\sphinxincludegraphics{{FirstNeutrinoEventAnnotated}.jpg}
\caption{First detection of neutrino}\label{\detokenize{index:id3}}\end{figure}

This project is part of \href{http://neutrino.xyz}{NeuPhysics} . Just click the link already and see what's going on.

Notes of some research is hosted on bitbucket as private repository \href{https://bitbucket.org/neuphysics/neutrino-reseaching}{here} . Login to view the most current research if you have access to it.

Support me:
\href{https://gratipay.com/~emptymalei/}{\sphinxincludegraphics{{tips-neutrino-4c924c}.svg}}
Cite this doc:
\href{https://zenodo.org/record/167903}{\sphinxincludegraphics{{zenodo.167903-blue}.svg}}
Here is also an index:
\sphinxincludegraphics{{Neutrino-Index-blue}.svg}
Sitemap can be downloaded: \sphinxcode{sitemap.xml} or \sphinxcode{sitemap.xml.gz} .


\chapter{Matter Stimulated Oscillation}
\label{\detokenize{matter-stimulated/index:neutrino-physics}}\label{\detokenize{matter-stimulated/index::doc}}\label{\detokenize{matter-stimulated/index:matter-stimulated-oscillation}}
We investigate a system with matter potential
\begin{equation*}
\begin{split}\lambda(x) = \lambda_0 + \delta \lambda (x),\end{split}
\end{equation*}
where \(\lambda_0 = \sqrt{2}G_F n_{e0}\) and \(\delta \lambda(x) = \sqrt{2}G_F \delta n_e(x)\).

The Hamiltonian in background matter basis \(\{\ket{\nu_{\mathrm{L}}},\ket{\nu_{\mathrm{H}}}\}\) is
\begin{equation*}
\begin{split}H = - \frac{\omega_m}{2} \sigma_3 + \frac{\delta \lambda}{2} \cos 2\theta_m \sigma_3 - \frac{\delta \lambda}{2} \sin 2 \theta_m \sigma_1.\end{split}
\end{equation*}
Here by background matter basis, we mean that the Hamiltonian is diagonalized if perturbation is zero in matter profile.

\begin{sphinxadmonition}{note}{Derive the Hamiltonian}

This Hamiltonian can be derived easily using
\begin{equation*}
\begin{split}H = -\frac{\omega_m}{3}\sigma_3 + \frac{\delta \lambda}{2} U^\dagger \sigma_3 U.\end{split}
\end{equation*}\end{sphinxadmonition}

What we are interested in is the transition between two background mass states. If we can have full converstion between the two mass states, we can have full conversion between flavor states.

\sphinxstylestrong{A Unitary Transformation}

Suppose the wave function in this basis is written as
\begin{equation*}
\begin{split}\begin{pmatrix} \psi_1 \\ \psi_2 \end{pmatrix}.\end{split}
\end{equation*}
To remove the position dependent \(\sigma_3\) term in the Hamiltonian which prevents us from solving the equation of motion easily, we define a new basis \(\{\ket{\tilde\nu_{\mathrm{L}}},\ket{\tilde\nu_{\mathrm{H}}}\}\) where the wave function is related to background matter basis through
\begin{equation*}
\begin{split}\begin{pmatrix} \psi_1 \\ \psi_2 \end{pmatrix} = \begin{pmatrix} e^{-i \eta (x)} & 0 \\  0 & e^{i \eta (x)}  \end{pmatrix} \begin{pmatrix} \psi_{b1} \\ \psi_{b2} \end{pmatrix}.\end{split}
\end{equation*}
\begin{sphinxadmonition}{note}{Transformation of Pauli Matrices}

This transformation, defined as \(\mathbf{T}\), is unitary,
\begin{equation*}
\begin{split}\mathbf{T}^\dagger \mathbf{T} = \mathbf{I}.\end{split}
\end{equation*}
It doesn't change \(\sigma_3\).
\begin{equation*}
\begin{split}\mathbf{T} \sigma_3 \mathbf{T}^\dagger &= \sigma_3\\
\mathbf{T}^\dagger \sigma_3 \mathbf{T} &= \sigma_3.\end{split}
\end{equation*}
It adds a phase to the off-diagonal elements of \(\sigma_1\),
\begin{equation*}
\begin{split}\mathbf{T} \sigma_1 \mathbf{T}^\dagger &= \begin{pmatrix} 0 & e^{-2i\eta} \\ e^{2 i\eta } & 0 \end{pmatrix} \\
\mathbf{T}^\dagger \sigma_1 \mathbf{T} &= \begin{pmatrix} 0 & e^{2i\eta} \\ e^{-2 i\eta } & 0 \end{pmatrix}.\end{split}
\end{equation*}
We can also look at the following very general transformation.
\begin{equation*}
\begin{split}& \begin{pmatrix} e^{i\eta_1} & 0 \\ 0 & e^{-i\eta_1}\end{pmatrix} \begin{pmatrix} a_{11} & a_{12} \\ a_{21} & a_{22}\end{pmatrix}  \begin{pmatrix} e^{i\eta_2} & 0 \\ 0 & e^{-i\eta_2}\end{pmatrix} \\
= & \begin{pmatrix} a_{11} e^{i(\eta_1+\eta_2)} & a_{12} e^{i(\eta_1 - \eta_2)} \\ a_{21} e^{-i(\eta_1-\eta_2)} & a_{22} e^{-i(\eta_1+\eta_2)}  \end{pmatrix}\end{split}
\end{equation*}
Another very useful relation is
\begin{equation*}
\begin{split}i\mathbf{T}^{\dagger} \partial_x \mathbf{T} = \partial_x \eta(x) \sigma_3.\end{split}
\end{equation*}\end{sphinxadmonition}

The Schrodinger equation in background matter basis is
\begin{equation*}
\begin{split}i\frac{d}{dx}\begin{pmatrix} \psi_{1} \\ \psi_2 \end{pmatrix} = \left(- \frac{\omega_m}{2} \sigma_3 + \frac{\delta \lambda}{2} \cos 2\theta_m \sigma_3 - \frac{\delta \lambda}{2} \sin 2 \theta_m \sigma_1 \right) \begin{pmatrix} \psi_{1} \\ \psi_2 \end{pmatrix}\end{split}
\end{equation*}
To write down the Schodinger equation in the new basis, we need the transformation of the Hamiltonian
\begin{equation*}
\begin{split}\mathbf{T}^\dagger \cdot \mathrm{LHS} &= \mathbf{T}^\dagger\left[ i \begin{pmatrix} - i \frac{d\eta}{dx} e^{-i\eta} & 0 \\ 0 & i \frac{d\eta}{dx} e^{i\eta} \end{pmatrix} \begin{pmatrix} \psi_{b1} \\ \psi_{b2} \end{pmatrix} + i \begin{pmatrix}  e^{-i\eta} & 0 \\ 0 & e^{i\eta} \end{pmatrix} \frac{d}{dx} \begin{pmatrix} \psi_1 \\ \psi_2 \end{pmatrix} \right] \\
& =   i \begin{pmatrix} - i \frac{d\eta}{dx}  & 0 \\ 0 & i \frac{d\eta}{dx}  \end{pmatrix} \begin{pmatrix} \psi_{b1} \\ \psi_{b2} \end{pmatrix} + i \frac{d}{dx} \begin{pmatrix} \psi_1 \\ \psi_2 \end{pmatrix}  .\end{split}
\end{equation*}\begin{equation*}
\begin{split}\mathbf{T}^\dagger \cdot \mathrm{RHS} &= \left[ -\frac{\omega_m}{2} \mathbf{T} ^\dagger \sigma_3 \mathbf{T} + \frac{\delta \lambda}{2} \cos 2\theta_m \mathbf{T}^\dagger \sigma_3 \mathbf{T} - \frac{\delta \lambda}{2} \sin 2\theta_m \mathbf{T}^\dagger \sigma_1 \mathbf{T}   \right] \begin{pmatrix} \psi_{b1} \\ \psi_{b2} \end{pmatrix} \\
& =  \left[ -\frac{\omega_m}{2} \sigma_3  + \frac{\delta \lambda}{2} \cos 2\theta_m  \sigma_3  - \frac{\delta \lambda}{2} \sin 2\theta_m \begin{pmatrix} 0 & e^{2i\eta} \\ e^{-2 i\eta } & 0 \end{pmatrix}   \right] \begin{pmatrix} \psi_{b1} \\ \psi_{b2} \end{pmatrix} .\end{split}
\end{equation*}
The equation of motion in this new basis becomes
\begin{equation*}
\begin{split}\begin{pmatrix}  \frac{d\eta}{dx}  & 0 \\ 0 & - \frac{d\eta}{dx}  \end{pmatrix} \begin{pmatrix} \psi_{b1} \\ \psi_{b2} \end{pmatrix} + i \frac{d}{dx} \begin{pmatrix} \psi_{b1} \\ \psi_{b2} \end{pmatrix} =
\left[ -\frac{\omega_m}{2} \sigma_3  + \frac{\delta \lambda}{2} \cos 2\theta_m  \sigma_3  - \frac{\delta \lambda}{2} \sin 2\theta_m \begin{pmatrix} 0 & e^{2i\eta} \\ e^{-2 i\eta } & 0 \end{pmatrix}   \right] \begin{pmatrix} \psi_{b1} \\ \psi_{b2} \end{pmatrix}\end{split}
\end{equation*}
The key is to remove the \(\sigma_3\) terms using this transformation, which requires
\begin{equation*}
\begin{split}\begin{pmatrix}  \frac{d\eta}{dx}  & 0 \\ 0 & - \frac{d\eta}{dx}  \end{pmatrix} \begin{pmatrix} \psi_{b1} \\ \psi_{b2} \end{pmatrix} = \left[ -\frac{\omega_m}{2} \sigma_3  + \frac{\delta \lambda}{2} \cos 2\theta_m  \sigma_3 \right] \begin{pmatrix} \psi_{b1} \\ \psi_{b2} \end{pmatrix}.\end{split}
\end{equation*}
It reduces to
\begin{equation*}
\begin{split}\frac{d\eta(x)}{dx} = - \frac{\omega_m}{2} + \frac{\delta \lambda(x)}{2} \cos 2\theta_m ,\end{split}
\end{equation*}
which has a general solution of the form
\phantomsection\label{\detokenize{matter-stimulated/index:matter-stimulated-equation-eta-x-general}}\phantomsection\label{\detokenize{matter-stimulated/index:equation-eta-x-general}}\begin{equation}\label{equation:matter-stimulated/index:eta-x-general}
\begin{split}\eta(x) - \eta(0) = - \frac{\omega_m}{2} x + \frac{\cos 2\theta_m}{2} \int_0^x \delta\lambda (\tau) d\tau.\end{split}
\end{equation}
We might choose \(\eta(0)=0\), which simplifies the relation
\begin{equation*}
\begin{split}\eta(x)= - \frac{\omega_m}{2} x + \frac{\cos 2\theta_m}{2} \int_0^x \delta\lambda (\tau) d\tau.\end{split}
\end{equation*}
What is left in the equation of motion is the part where off-diagonal Hamiltonian takes effect,
\begin{equation*}
\begin{split}i \frac{d}{dx} \begin{pmatrix} \psi_{b1} \\ \psi_{b2} \end{pmatrix} = - \frac{\delta \lambda}{2} \sin 2\theta_m \begin{pmatrix} 0 & e^{2i\eta} \\ e^{-2 i\eta } & 0 \end{pmatrix}  \begin{pmatrix} \psi_{b1} \\ \psi_{b2} \end{pmatrix}.\end{split}
\end{equation*}
\begin{sphinxadmonition}{note}{Other Initial Conditions}

The initial condition can be other convinient ones. For example we can remove the integration constant of the last term in the relation.
\end{sphinxadmonition}

At any position/time, the wave function in background matter basis is
\phantomsection\label{\detokenize{matter-stimulated/index:matter-stimulated-equation-wavefunction-diff-basis}}\phantomsection\label{\detokenize{matter-stimulated/index:equation-wavefunction-diff-basis}}\begin{equation}\label{equation:matter-stimulated/index:wavefunction-diff-basis}
\begin{split}\begin{pmatrix} \psi_1 (x) \\ \psi_2(x)  \end{pmatrix} = \begin{pmatrix} e^{- i \eta} \psi_{b1} (x) \\ e^{i\eta} \psi_{b2} (x)  \end{pmatrix}.\end{split}
\end{equation}
To calculated the transition from low energy state to high energy state in background matter basis, with initial condition
\begin{equation*}
\begin{split}\begin{pmatrix} \psi_1 (0) \\ \psi_2(0)  \end{pmatrix} = \begin{pmatrix} 1 \\ 0  \end{pmatrix},\end{split}
\end{equation*}
we simply calculate
\begin{equation*}
\begin{split}P_{1 \to 2} (x) = \lvert e^{i\eta} \psi_{b2} (x)  \rvert^2 = \lvert \psi_{b2} (x)  \rvert^2 .\end{split}
\end{equation*}

\section{Scales}
\label{\detokenize{matter-stimulated/scales::doc}}\label{\detokenize{matter-stimulated/scales:scales}}
We need to discuss the scales relavent to different environments.


\subsection{Supernova}
\label{\detokenize{matter-stimulated/scales:supernova}}

\subsubsection{Initial Matter Density Configuration}
\label{\detokenize{matter-stimulated/scales:initial-matter-density-configuration}}\begin{figure}[htbp]
\centering
\capstart

\noindent\sphinxincludegraphics{{supernova-matter-profile-initial-configuration}.png}
\caption{Supernova initial matter profile from \label{\detokenize{matter-stimulated/scales:id1}}{\hyperref[\detokenize{matter-stimulated/scales:shigeyama1990}]{\sphinxcrossref{{[}shigeyama1990{]}}}}.}\label{\detokenize{matter-stimulated/scales:id5}}\end{figure}


\subsubsection{Early Stage}
\label{\detokenize{matter-stimulated/scales:early-stage}}
One of the interesting stages is the bounce on the core.

To find out the relavent length and energy scales, we need to know
\begin{enumerate}
\item {} 
size of supernova,

\item {} 
matter profile.

\end{enumerate}

From the 2D simulation by E. Borriello et al, the size of supernova at 2sec after the bounce is about \(2.7\times 10^4\mathrm{km}\) \phantomsection\label{\detokenize{matter-stimulated/scales:id2}}{\hyperref[\detokenize{matter-stimulated/scales:eborriello2014}]{\sphinxcrossref{{[}EBorriello2014{]}}}}.
\begin{figure}[htbp]
\centering
\capstart

\noindent\sphinxincludegraphics{{supernova-shock-turbulence}.png}
\caption{2D supernova simulation for a 15 solar mass supernova \label{\detokenize{matter-stimulated/scales:id3}}{\hyperref[\detokenize{matter-stimulated/scales:eborriello2014}]{\sphinxcrossref{{[}EBorriello2014{]}}}}.}\label{\detokenize{matter-stimulated/scales:id6}}\end{figure}

The angle averaged density on radial direction is of the order \(10^3\mathrm{g/cm^3}\) to \(10^5\mathrm{g/cm^3}\).
\begin{figure}[htbp]
\centering
\capstart

\noindent\sphinxincludegraphics{{supernova-radial-density-profile}.png}
\caption{Supernova 2D simulation radial density (averaged over angles) \label{\detokenize{matter-stimulated/scales:id4}}{\hyperref[\detokenize{matter-stimulated/scales:eborriello2014}]{\sphinxcrossref{{[}EBorriello2014{]}}}}.}\label{\detokenize{matter-stimulated/scales:id7}}\end{figure}
\begin{figure}[htbp]
\centering
\capstart

\noindent\sphinxincludegraphics{{earth-density-profile-PREM-model}.png}
\caption{Earth density profile from PREM model.}\label{\detokenize{matter-stimulated/scales:id8}}\end{figure}

That was the densities. As a reference, we can calculate MSW resonance densities for some neutrino energies. The relation between neutrino energy and MSW resonance electron number density is calculated through
\begin{equation*}
\begin{split}\omega_v \cos 2\theta_v = \sqrt{2} G_F n_e .\end{split}
\end{equation*}
Move around the terms and we get
\begin{equation*}
\begin{split}n_e &= \frac{\delta m^2 \cos 2\theta_v}{2\sqrt{2} G_F} \frac{1}{E} \\
&= 7.49\times 10^{-15}\mathrm{GeV^3} \left(  \frac{\delta m^2}{2.6\times 10^{-3}\mathrm{eV^2}} \frac{\cos 2\theta_v}{0.95} \frac{1}{G_F/(1.17\times 10^{-5}\mathrm{GeV^{-2}})} \frac{1}{E/10\mathrm{MeV}} \right),\end{split}
\end{equation*}
where we have used \(\delta m^2 = \delta m^2_{\mathrm{atm}}\).

Using natural units, we know that
\begin{equation*}
\begin{split}1\mathrm{GeV^3} = 1.3\times 10^{41} \mathrm{cm^{-3}}.\end{split}
\end{equation*}
The number density of electrons of MSW resonance is
\begin{equation*}
\begin{split}n_e = 9.74\times 10^{26}\mathrm{cm^{-3}} \left(  \frac{\delta m^2}{2.6\times 10^{-3}\mathrm{eV^2}} \frac{\cos 2\theta_v}{0.95} \frac{1}{G_F/(1.17\times 10^{-5}\mathrm{GeV^{-2}})} \frac{1}{E/10\mathrm{MeV}} \right).\end{split}
\end{equation*}
Assuming an electron fraction \(Y_e\), which is defined as the ration of electrons to baryons, the mass density is
\begin{equation*}
\begin{split}\rho &= \frac{n_e}{Y_e} \times m_b \\
& = \frac{ 9.74\times 10^{26} \mathrm{cm^{-3}} \times 1.67\times 10^{-24} \mathrm{g} }{ 0.5 } \frac{1}{ Y_e/0.5 } \left(  \frac{\delta m^2}{2.6\times 10^{-3}\mathrm{eV^2}} \frac{\cos 2\theta_v}{0.95} \frac{1}{G_F/(1.17\times 10^{-5}\mathrm{GeV^{-2}})} \frac{1}{E/10\mathrm{MeV}} \right)\\
& =  3.25\times 10^3 \mathrm{g/cm^3} \frac{1}{ Y_e/0.5 } \left(  \frac{\delta m^2}{2.6\times 10^{-3}\mathrm{eV^2}} \frac{\cos 2\theta_v}{0.95} \frac{1}{G_F/(1.17\times 10^{-5}\mathrm{GeV^{-2}})} \frac{1}{E/10\mathrm{MeV}} \right),\end{split}
\end{equation*}
where I have used \(\sin^2(2\theta_v) = \sin^2(2\theta_{13})=0.093\) from PDG 2014.

In the case of \(\delta m^2 = \delta m^2_{\mathrm{sol}} \sim 7.59\times 10^{-5}\mathrm{eV^2}\), we would have a density that is two orders smaller.


\subsubsection{Lengths}
\label{\detokenize{matter-stimulated/scales:lengths}}
Assuming energy of neutrinos to be 10MeV, the general idea is that if the stimulated neutrino oscillations have a larger oscillation length than \(10^4\) km, we don't really need to consider the oscillations. Otherwise, oscillations could cause depolarization or little effect at all.

In this case what should be done is to check whether what are the rel.

Another view is that to see the stimulated neutrino oscillations, we need to find a relavent neutrino energy range.


\section{MSW Effect Revisted}
\label{\detokenize{matter-stimulated/msw-revisted::doc}}\label{\detokenize{matter-stimulated/msw-revisted:msw-effect-revisted}}
\begin{sphinxadmonition}{note}{Pauli Matrices and Rotations}

Given a rotation
\begin{equation*}
\begin{split}U = \begin{pmatrix} \cos \theta & \sin \theta \\ -\sin\theta & \cos \theta \end{pmatrix},\end{split}
\end{equation*}
its effect on Pauli matrices are
\begin{equation*}
\begin{split}U^\dagger \sigma_3 U  &=\cos 2\theta \sigma_3 + \sin 2\theta \sigma_1 \\
U^\dagger \sigma_1 U & = -\sin 2\theta \sigma_3 + \cos 2\theta \sigma_1.\end{split}
\end{equation*}\end{sphinxadmonition}


\subsection{Flavor Basis}
\label{\detokenize{matter-stimulated/msw-revisted:flavor-basis}}
\begin{sphinxadmonition}{note}{Vacuum Oscillations}

Vacuum oscillations is already a Rabi oscillation at resonance with oscillation width \(\omega_v \sin 2\theta_v\).
\end{sphinxadmonition}

Neutrino oscillation in matter has a Hamiltonian in flavor basis
\begin{equation*}
\begin{split}H^{(f)} = \left(- \frac{1}{2} \omega_v \cos 2\theta_v +\frac{1}{2}\lambda(x)  \right)\sigma_3 + \frac{1}{2} \omega_v \sin 2\theta_v \sigma_1.\end{split}
\end{equation*}
The Schroding equation is
\begin{equation*}
\begin{split}i \partial_x \Psi^{(f)} = H^{(f)} \Psi^{(f)}.\end{split}
\end{equation*}
To make connections to Rabi oscillations, we would like to remove the changing \(\sigma_3\) terms, using a transformation
\begin{equation*}
\begin{split}T = \begin{pmatrix} e^{-i \eta (x)} & 0 \\  0 & e^{i \eta (x)}  \end{pmatrix},\end{split}
\end{equation*}
which transform the flavor basis to another basis
\begin{equation*}
\begin{split}\begin{pmatrix} \psi_e \\ \psi_x \end{pmatrix} = \begin{pmatrix} e^{-i \eta (x)} & 0 \\  0 & e^{i \eta (x)}  \end{pmatrix} \begin{pmatrix} \psi_{a} \\ \psi_{b} \end{pmatrix}.\end{split}
\end{equation*}
The Schrodinger equation can be written into this new basis
\begin{equation*}
\begin{split}i \partial_x (T \Psi^{(r)}) = H^{(f)} T\Psi^{(r)},\end{split}
\end{equation*}
which is simplified to
\begin{equation*}
\begin{split}i \partial_x \Psi^{(r)} = H^{(r)} \Psi^{(r)},\end{split}
\end{equation*}
where
\begin{equation*}
\begin{split}H^{(r)} = - \frac{1}{2}\omega_v \cos 2\theta_v \sigma_3 + \frac{1}{2} \omega_v \sin 2\theta_v \begin{pmatrix}
0 & e^{2i\eta(x)} \\
e^{-2i\eta(x)} & 0 \\
\end{pmatrix},\end{split}
\end{equation*}
in which we remove the varying component of \(\sigma_3\) elements using
\begin{equation*}
\begin{split}\frac{d}{dx}\eta(x) = \frac{\lambda(x)}{2}.\end{split}
\end{equation*}
The final Hamiltonian would have some form
\begin{equation*}
\begin{split}H^{(r)} = - \frac{1}{2}\omega_v \cos 2\theta_v \sigma_3 + \frac{1}{2} \omega_v \sin 2\theta_v \begin{pmatrix}
0 & e^{i\int_0^x \lambda(\tau)d\tau + 2i\eta(0)} \\
e^{-i\int_0^x \lambda(\tau)d\tau - 2i\eta(0)} & 0 \\
\end{pmatrix},\end{split}
\end{equation*}
where \(\eta(0)\) is chosen to conter the constant terms from the integral.

For arbitary matter profile, we could first apply Fourier expand the profile into trig function then use Jacobi-Anger expansion so that the system becomes a lot of Rabi oscillations.

Any transformations or expansions that decompose \(\exp{\left(i\int_0^x \lambda(\tau)d\tau\right)}\) into many summations of \(\exp{\left( i a x + b \right)}\) would be enough for an Rabi oscillation interpretation.

Let's discuss the constant matter profile, \(\lambda(x) = \lambda_0\). Thus we have
\begin{equation*}
\begin{split}\eta(x) = \frac{1}{2} \lambda_0 x.\end{split}
\end{equation*}
The Hamiltonian becomes
\begin{equation*}
\begin{split}H^{(r)} = - \frac{1}{2}\omega_v \cos 2\theta_v \sigma_3 + \frac{1}{2} \omega_v \sin 2\theta_v \begin{pmatrix}
0 & e^{i\lambda_0 x} \\
e^{-i\lambda_0 x} & 0 \\
\end{pmatrix},\end{split}
\end{equation*}
which is exactly a Rabi oscillation. The resonance condition is
\begin{equation*}
\begin{split}\lambda_0 = \omega_v \cos 2\theta_v.\end{split}
\end{equation*}

\subsection{Instanteneous Matter Basis}
\label{\detokenize{matter-stimulated/msw-revisted:instanteneous-matter-basis}}
Neutrino oscillation in matter has a Hamiltonian in flavor basis
\begin{equation*}
\begin{split}H^{(f)} = \left(- \frac{1}{2} \omega_v \cos 2\theta_v +\frac{1}{2}\lambda(x)  \right)\sigma_3 + \frac{1}{2} \omega_v \sin 2\theta_v \sigma_1.\end{split}
\end{equation*}
The Schroding equation is
\begin{equation*}
\begin{split}i \partial_x \Psi^{(f)} = H^{(f)} \Psi^{(f)},\end{split}
\end{equation*}
which can be transformed to instantaneous matter basis by applying a rotation \(U\),
\begin{equation*}
\begin{split}i \partial_x \left(  U\Psi^{(m)} \right)= H^{(f)} U\Psi^{(m)},\end{split}
\end{equation*}
where
\begin{equation*}
\begin{split}U = \begin{pmatrix} \cos \theta_m & \sin \theta_m \\ -\sin\theta_m & \cos \theta_m \end{pmatrix}.\end{split}
\end{equation*}
With a little algebra, we can write the system into
\begin{equation*}
\begin{split}i \partial _x \Psi^{(m)} = H^{(m)}\Psi^{(m)}\end{split}
\end{equation*}\begin{equation*}
\begin{split}H^{(m)} = U^\dagger H^{(f)} U - i U^\dagger \partial_x U.\end{split}
\end{equation*}
By setting the off-diagonal elements of the first term \(U^\dagger H^{(f)} U\) to zero, we can derive the relation
\begin{equation*}
\begin{split}\tan 2\theta_m = \frac{\sin 2\theta_v}{\cos 2\theta_v - \lambda/\omega_v}.\end{split}
\end{equation*}
Furthermore, we derive the term
\begin{equation*}
\begin{split}i U^\dagger \partial_x U = - \dot\theta_m \sigma_2.\end{split}
\end{equation*}
We can calculate \(\dot\theta_m\) by taking the derivative of \(\tan 2\theta_m\),
\begin{equation*}
\begin{split}\frac{d}{dx} \tan 2\theta_m = \frac{2}{\cos^2 2\theta_m} \dot\theta_m,\end{split}
\end{equation*}
so that
\begin{equation*}
\begin{split}\dot\theta_m &= \frac{1}{2} \cos^2 (2\theta_m) \frac{d}{dx} \tan 2\theta_m \\
& = \frac{1}{2} \frac{(\cos 2\theta_v - \lambda/\omega_v)^2}{ (\lambda/\omega_v)^2 + 1 - 2\lambda \cos 2\theta_v /\omega_v } \frac{d}{dx} \frac{\sin 2\theta_v}{\cos 2\theta_v - \lambda/\omega_v} \\
& = \frac{1}{2} \frac{(\cos 2\theta_v - \lambda/\omega_v)^2}{ (\lambda/\omega_v)^2 + 1 - 2\lambda \cos 2\theta_v /\omega_v }  \frac{\sin 2\theta_v}{(\cos 2\theta_v - \lambda/\omega_v)^2} \frac{1}{\omega)v} \frac{d}{dx} \lambda(x) \\
& = \frac{1}{2} \sin 2\theta_m \frac{1}{\omega_m} \frac{d}{dx} \lambda(x).\end{split}
\end{equation*}

\section{Parametric Resonance Revisted}
\label{\detokenize{matter-stimulated/parametric-resonance-revisted:parametric-resonance-revisted}}\label{\detokenize{matter-stimulated/parametric-resonance-revisted::doc}}
For a matter profile
\phantomsection\label{\detokenize{matter-stimulated/parametric-resonance-revisted:equation-parametric-resonance-matter-profile-single-frequency}}\begin{equation}\label{equation:matter-stimulated/parametric-resonance-revisted:parametric-resonance-matter-profile-single-frequency}
\begin{split}\lambda(x) = \lambda_0 + A\sin( k x),\end{split}
\end{equation}
the parametric resonance condition is \phantomsection\label{\detokenize{matter-stimulated/parametric-resonance-revisted:id1}}{\hyperref[\detokenize{matter-stimulated/parametric-resonance-revisted:krastev1989}]{\sphinxcrossref{{[}Krastev1989{]}}}}
\begin{equation*}
\begin{split}\omega_m \sim n k,\end{split}
\end{equation*}
where \(k=1,2,3,\cdots\).

\begin{sphinxadmonition}{note}{A Minimal Derivation of Parametric Resonance Condition}

This derivation is from \phantomsection\label{\detokenize{matter-stimulated/parametric-resonance-revisted:id2}}{\hyperref[\detokenize{matter-stimulated/parametric-resonance-revisted:krastev1989}]{\sphinxcrossref{{[}Krastev1989{]}}}}.

On one matter profile wavelength, the accumulated phase of the system should be a multiple of \(2\pi\),
\begin{equation*}
\begin{split}\int_0^{2\pi/k} \omega_m = 2\pi n,\end{split}
\end{equation*}
where
\begin{equation*}
\begin{split}\omega_m = \omega_v \sqrt{ (\lambda(x)/\omega_v - \cos 2\theta_v)^2 + \sin^2 2\theta_v }.\end{split}
\end{equation*}
\(\omega_m\) doesn't change a lot during one matter profile oscillation wavelength \(2\pi/k\), thus we perform the integration by treating the integrand as a constant, which returns
\begin{equation*}
\begin{split}\omega_m \frac{2\pi}{k} =2\pi n,\end{split}
\end{equation*}
which can be simplified
\begin{equation*}
\begin{split}\omega_m = n k.\end{split}
\end{equation*}\end{sphinxadmonition}

For matter profile \eqref{equation:matter-stimulated/parametric-resonance-revisted:parametric-resonance-matter-profile-single-frequency}, we already know that this is the condition of resonance from Rabi oscillation.


\subsection{Castle Wall - Direct Rotation Using T Matrix}
\label{\detokenize{matter-stimulated/parametric-resonance-revisted:castle-wall-direct-rotation-using-t-matrix}}
Matter profile
\phantomsection\label{\detokenize{matter-stimulated/parametric-resonance-revisted:equation-parametric-resonance-castle-wall-profile}}\begin{equation}\label{equation:matter-stimulated/parametric-resonance-revisted:parametric-resonance-castle-wall-profile}
\begin{split}\lambda(x) = \begin{cases} \lambda_1, \quad 0 \leq \mathrm{Mod}[x] \leq X_1  \\  \lambda_2, \quad  X_1 < \mathrm{Mod}[x] < X_1+X_2  \end{cases},\end{split}
\end{equation}
which is periodic with period \(X \equiv X_1 + X_2\).

We can, on first thought, rotate the system first, which gives us the Hamiltonian
\begin{equation*}
\begin{split}H^{(f)} = -\frac{\omega_v}{2} \cos 2\theta_v \sigma_3 + \frac{1}{2} \omega_v \sin 2\theta_v \begin{pmatrix}
0 & \exp\left( i\int_0^x \lambda(\tau) d\tau + 2i \eta(0) \right) \\
\exp\left( -i\int_0^x \lambda(\tau) d\tau - 2i \eta(0) \right) & 0
\end{pmatrix},\end{split}
\end{equation*}
where \(\eta(0)\) is the initial condition we chose for
\begin{equation*}
\begin{split}\frac{d}{dx}\eta(x) = \frac{\lambda(x)}{2}.\end{split}
\end{equation*}
For \(x<X_1\), we have
\begin{equation*}
\begin{split}i\int_0^x \lambda(\tau) d\tau + 2i \eta(0) = i \lambda_1 x,\end{split}
\end{equation*}
while for \(X_1<x<X_1+X_2\), we have
\begin{equation*}
\begin{split}i\int_0^x \lambda(\tau) d\tau + 2i \eta(0) = i \lambda_1 X_1 + i \lambda_2 (x-X_1).\end{split}
\end{equation*}
One of the important conclusion from the research in Rabi oscilations is that the constant phase term has no effect on the system at all. Thus even for a much larger x, we can always drop the constant phase \(i(\lambda_1X_1+\lambda_2X_2)N\) which is obtained through integration over N periods. What's more, the term \(i \lambda_1 X_1\) and \(-i\lambda_2X_1\) can also be dropped.

At this point, we would conclude that the resonance condition should be obtained by setting either mode \(e^{i\lambda_1 x}\) or \(e^{i\lambda_2 x}\).

\begin{sphinxadmonition}{note}{TO DO}
\begin{enumerate}
\item {} 
\sphinxstylestrong{NUMERICAL CALCULATIONS?}

\item {} 
\DUrole{strike}{I can not relate this result with the Akhmedov condition.} (See update {\hyperref[\detokenize{matter-stimulated/parametric-resonance-revisted:clarification-of-the-resonance-condition}]{\emph{Clarification of Resonance Condition}}}.)

\end{enumerate}
\end{sphinxadmonition}
\phantomsection\label{\detokenize{matter-stimulated/parametric-resonance-revisted:clarification-of-the-resonance-condition}}
\begin{sphinxadmonition}{note}{Clarification of Resonance Condition}

The idea of increasing in transition is initial condition related. If we set the resonance for matter density \(\lambda_1\), after a evolution of distance \(X_1\), the state of the system has changed and the corresponding flavor isospin vector is not on the direction of z axis anymore. Thus we do not have a simple prediction of evolution across \(X_2\) unless we know the state right before this evolution.

From this simple argument, we have a hint that the resonance could well depend on period of the matter profile.
\end{sphinxadmonition}


\subsection{Castle Wall Profile - Fourier Series}
\label{\detokenize{matter-stimulated/parametric-resonance-revisted:castle-wall-profile-fourier-series}}

\subsubsection{First Attemptation: Full Fourier Series}
\label{\detokenize{matter-stimulated/parametric-resonance-revisted:first-attemptation-full-fourier-series}}
Another approach is to decompose the system into a lot of sin or cos modes so that we can use the result we had before.

Such a periodic matter profile \eqref{equation:matter-stimulated/parametric-resonance-revisted:parametric-resonance-castle-wall-profile} can be decomposed using Fourier series,
\begin{equation*}
\begin{split}\lambda(x) = \sum_{n=-\infty}^{\infty} \Lambda_n \exp\left( \frac{i2\pi n x}{X} \right) = \sum_{n=-\infty}^{\infty} \Lambda_n \exp\left( i \omega_0 n x \right),\end{split}
\end{equation*}
where \(\omega_0 = \frac{2\pi}{X}\). The coefficients are evaluated using the orthogonal relation of exponentials for \(n\neq 0\),
\phantomsection\label{\detokenize{matter-stimulated/parametric-resonance-revisted:equation-parametric-resonance-castle-wall-fourier-coeff}}\begin{equation}\label{equation:matter-stimulated/parametric-resonance-revisted:parametric-resonance-castle-wall-fourier-coeff}
\begin{split}\Lambda_n &= \frac{1}{X} \int_0^X \lambda(x) e^{ - i \omega_0 n x} dx \\
& = \frac{1}{X} \left( \int_{0}^{X_1} \lambda_1 e^{ - i \omega_0 n x} dx + \int_{X_1}^{X_1+X_2} \lambda_2 e^{ - i \omega_0 n x} dx  \right) \\
& = \frac{1}{X} \frac{X}{-i2\pi n} \left( \lambda_1 e^{-i\omega_0 n X_1} + \lambda_2 \left( e^{-i\omega_0 n X} - e^{-i\omega_0 n X_1}  \right) \right) \\
& = \frac{i}{2\pi n} \left( -\lambda_1 + (\lambda_1 - \lambda_2) e^{-i2\pi n X_1/X} + \lambda_2 e^{-i 2\pi n} \right).\end{split}
\end{equation}
For n=0, we have
\begin{equation*}
\begin{split}\Lambda_0 = \frac{X_1 \lambda_1 + X_2 \lambda_2}{X}.\end{split}
\end{equation*}
\begin{sphinxadmonition}{note}{Verification of This Result}

We can verify this result by setting \(\lambda_1 =\lambda_2 = \lambda\), which should give use the result \(\Lambda_n = \frac{i}{2\pi n}\lambda (e^{- i 2\pi n} - 1)\). By setting \(\lambda_1 =\lambda_2 = \lambda\) in the last step of \eqref{equation:matter-stimulated/parametric-resonance-revisted:parametric-resonance-castle-wall-fourier-coeff}, we have the result that matches our expectation.

Another more complete way to verify this result is to compare the numerical results using this Fourier series and the original profile, which is shown in \hyperref[\detokenize{matter-stimulated/parametric-resonance-revisted:parametric-resonance-reconstruction-of-castle-wall-0point01-0point02-1-1point8}]{Fig.\@ \ref{\detokenize{matter-stimulated/parametric-resonance-revisted:parametric-resonance-reconstruction-of-castle-wall-0point01-0point02-1-1point8}}}.
\begin{figure}[htbp]
\centering
\capstart

\noindent\sphinxincludegraphics{{reconstruction-of-castle-wall-0point01-0point02-1-1point8}.png}
\caption{Reconstruction of castle wall profile using Fourier series.}\label{\detokenize{matter-stimulated/parametric-resonance-revisted:parametric-resonance-reconstruction-of-castle-wall-0point01-0point02-1-1point8}}\label{\detokenize{matter-stimulated/parametric-resonance-revisted:id3}}\end{figure}
\end{sphinxadmonition}

\begin{sphinxadmonition}{note}{Akhmedov's Castle Wall Parametric Resonance Condition}

The resonance condition is given by
\begin{equation*}
\begin{split}\frac{\tan (\omega_{m1}X_1/2)}{\tan (\omega_{m2}X_2/2)} = - \frac{\cos 2\theta_{m2}}{\cos 2\theta_{m1}} = - \frac{ \cos 2\theta_v - \lambda_1/\omega_v }{  \cos 2\theta_v - \lambda_2/\omega_v } \frac{ \sqrt{ (\lambda_2/\omega_v)^2  + 1 - 2\cos 2\theta_v \lambda_2/\omega_v } }{ \sqrt{ (\lambda_1/\omega_v)^2  + 1 - 2\cos 2\theta_v \lambda_1/\omega_v } }.\end{split}
\end{equation*}
This is rather opaque.
\end{sphinxadmonition}

To find out the Rabi modes, we first rotate to the rotation frame using T matrix. Since the Hamiltonian in flavor basis is
\begin{equation*}
\begin{split}H^{(f)} = -\frac{\omega_v}{2} \cos 2\theta_v \sigma_3 + \frac{\omega_v}{2} \sin 2\theta_v \sigma_1  + \frac{1}{2}  \sum_{n=-\infty}^{\infty} \Lambda_n \exp\left( i \omega_0 n x \right) \sigma_3.\end{split}
\end{equation*}
We notice that the 0-mode is a constant which plays a role as the background matter profile. We should rotate to the background matter basis first because the zero mode is always there and will generate some flavor transitions. However, this is not anything special but the constant matter flavor transitions. \DUrole{highlight-text}{Rotating to the background matter profile allows us to concentrate on the actual parametric effect.} To do so we rewrite the Hamiltonian
\begin{equation*}
\begin{split}H^{(f)} = -\frac{\omega_v}{2} \cos 2\theta_v \sigma_3 + \frac{\omega_v}{2} \sin 2\theta_v \sigma_1  + \left( \frac{1}{2}\Lambda_0  + \frac{1}{2}  \sum_{q=-\infty, q\neq 0}^{\infty} \Lambda_q \exp\left( i \omega_0 q x \right) \right) \sigma_3,\end{split}
\end{equation*}
where we treat \(\Lambda_0\) as the background matter profile. For simplifity we define
\begin{equation*}
\begin{split}\delta \lambda =  \sum_{q=-\infty, q\neq 0}^{\infty} \Lambda_q \exp\left( i \omega_0 q x \right),\end{split}
\end{equation*}
so that
\begin{equation*}
\begin{split}H^{(f)} = -\frac{\omega_v}{2} \cos 2\theta_v \sigma_3 + \frac{\omega_v}{2} \sin 2\theta_v \sigma_1  + \left( \frac{1}{2}\Lambda_0  + \frac{1}{2}  \delta \lambda \right) \sigma_3.\end{split}
\end{equation*}
The Hamiltonian in background matter basis becomes
\begin{equation*}
\begin{split}H &= - \frac{1}{2}\omega_m \sigma_3 + \frac{1}{2} \delta\lambda - \frac{1}{2} \delta \lambda \sigma_1 \\
&= - \frac{1}{2}\omega_m \sigma_3  + \frac{1}{2} \sum_{q=-\infty, q\neq 0}^{\infty} \Lambda_q e^{i q \omega_0 x} \cos 2\theta_m \sigma_3 - \frac{1}{2} \sum_{q=-\infty, q\neq 0}^{\infty} \Lambda_q e^{i q \omega_0 x} \sin 2\theta_m \sigma_1.\end{split}
\end{equation*}
\begin{sphinxadmonition}{note}{Rotating to Background Matter Basis}

The Hamiltonian in background basis is
\begin{equation*}
\begin{split}H = U^\dagger H^{(f)} U,\end{split}
\end{equation*}
where
\begin{equation*}
\begin{split}U = \begin{pmatrix}
\cos \theta_{\mathrm{m}} & \sin \theta_{\mathrm{m}} \\
-\sin \theta_{\mathrm{m}} & \cos \theta_{\mathrm{m}}
\end{pmatrix} = \cos \theta_m I + i\sin \theta_m \sigma_2.\end{split}
\end{equation*}
where \(\sin \theta_m\) and \(\cos \theta_m\) are the corresponding values for background matter density equals \(\Lambda_0\). We also know that
\begin{equation*}
\begin{split}&\frac{1}{2}\omega_v\left( \Lambda_0/\omega_v \cos 2\theta_m - \cos 2\theta_v\cos 2\theta_m - \sin 2\theta_v\sin 2\theta_m \right) \\
=&-\frac{1}{2}\omega_m,\end{split}
\end{equation*}
and
\begin{equation*}
\begin{split}&\frac{1}{2}\Lambda_0 \sin 2\theta_m - \frac{1}{2} \omega_v \cos 2\theta_v \sin 2\theta_m + \frac{1}{2} \omega_v \sin 2\theta_v \cos 2\theta_m \\
=& 0.\end{split}
\end{equation*}\end{sphinxadmonition}

Then we go to the frame that the z component of Hamiltonina vector is stationary. The caveat here is that the rotation we would use
\begin{equation*}
\begin{split}\begin{pmatrix} \psi_1 \\ \psi_2 \end{pmatrix} = \begin{pmatrix} e^{-i \eta (x)} & 0 \\  0 & e^{i \eta (x)}  \end{pmatrix} \begin{pmatrix} \psi_{s1} \\ \psi_{s2} \end{pmatrix}\end{split}
\end{equation*}
is only valid for \sphinxstylestrong{real} \(\eta(x)\) because complex \(\eta(x)\) will make this transformation \sphinxstylestrong{non-unitary}.

\DUrole{highlight-text}{So this derivation has to stop here.}


\subsubsection{The Trick: Even Fourier Series}
\label{\detokenize{matter-stimulated/parametric-resonance-revisted:the-trick-even-fourier-series}}
\begin{sphinxadmonition}{note}{Fourier Series of Even and Odd Functions}

In general the Fourier series of a periodic function defined on \(\left[ -\frac{X}{2}, \frac{X}{2} \right]\) is
\begin{equation*}
\begin{split}\lambda(x) = \frac{a_0}{2} + \sum_{n=1}^\infty a_n \cos(n 2\pi x/X) + \sum_{n=1}^\infty b_n \sin(n 2\pi x/X),\end{split}
\end{equation*}
where
\begin{equation*}
\begin{split}a_0 & = \frac{2}{X} \int^{X/2}_{-X/2} \lambda(x) d x \\
a_n & = \frac{2}{X} \int_{-X/2}^{X/2} \lambda(x) \cos ( n2\pi x/X ) dx\\
b_n & = \frac{2}{X} \int_{-X/2}^{X/2} \lambda(x) \sin( n 2\pi x/X ) dx.\end{split}
\end{equation*}
For EVEN function \(\lambda(x)\), we have
\begin{equation*}
\begin{split}\lambda(x) = \frac{1}{2}a_0 + \sum_{n=1}^\infty a_n \cos (n 2\pi x/X).\end{split}
\end{equation*}\end{sphinxadmonition}
\begin{figure}[htbp]
\centering
\capstart

\noindent\sphinxincludegraphics{{piecewise-profile}.png}
\caption{Shifted castle wall profile.}\label{\detokenize{matter-stimulated/parametric-resonance-revisted:id4}}\end{figure}

We shift the castle wall profile and make it always even, so that
\begin{equation*}
\begin{split}\lambda(x) = \begin{cases} \lambda_2 , &\quad -\frac{X_2}{2}-\frac{X_1}{2}\le x\le -\frac{X_1}{2} \\
\lambda_1, &\quad -\frac{X_1}{2}\le x\le \frac{X_1}{2} \\
\lambda_2, &\quad \frac{X_1}{2}\le x\le \frac{X_1}{2}+\frac{X_2}{2}
\end{cases}\end{split}
\end{equation*}
Fourier series of the profile is
\begin{equation*}
\begin{split}\lambda(x) = \frac{1}{2}\Lambda_0 + \sum_{q=1}^{\infty} \Lambda_q \cos\left( \frac{ 2\pi q x}{X} \right) = \frac{1}{2} \Lambda_0 + \sum_{q=1}^{\infty} \Lambda_q \cos\left( \omega_0 q x \right),\end{split}
\end{equation*}
where
\begin{equation*}
\begin{split}\Lambda_0 &= \frac{2}{X} \int^{X/2}_{-X/2} \lambda(x) d x \\
& = \frac{2}{X} \left(  \lambda_2 X_2 + \lambda_1 X_1   \right) \\
\Lambda_q &= \frac{2}{X} \int_{-X/2}^{X/2} \lambda(x) \cos(n 2\pi x/X)dx \\
& = \frac{2}{X} \left( \lambda_2 \int_{-X/2}^{-X_1/2} \cos(n 2\pi x/X)dx + \lambda_1 \int_{-X_1/2}^{X_1/2} \cos(n 2\pi x/X)dx + \lambda_2 \int_{X_1/2}^{X/2} \cos(n 2\pi x/X)dx \right) \\
& = \frac{2}{q\pi} \left( \lambda_2\left( \sin(q\omega_0 X/2) - \sin(q\omega_0 X_1/2) \right) + \lambda_1 \sin( q\omega_0 X_1/2)  \right)  \\
& = \frac{2}{q\pi} \left( \lambda_2\left( \sin(q \pi ) - \sin(q \pi X_1/X) \right) + \lambda_1 \sin( q\pi X_1/X)  \right)\end{split}
\end{equation*}
\begin{sphinxadmonition}{note}{Numerical Verification}
\begin{figure}[htbp]
\centering
\capstart

\noindent\sphinxincludegraphics{{reconstruction-of-even-castle-wall-0point01-0point02-1-1point8}.png}
\caption{Reconstruction of castle wall profile using Fourier series.}\label{\detokenize{matter-stimulated/parametric-resonance-revisted:parametric-resonance-reconstruction-of-even-castle-wall-0point01-0point02-1-1point8}}\label{\detokenize{matter-stimulated/parametric-resonance-revisted:id5}}\end{figure}
\end{sphinxadmonition}

Transform the system into background matter basis, so that we can inspect the stimulated transitions more clearly. As a first step, we rewrite the Hamiltonian
\begin{equation*}
\begin{split}H^{(f)} = -\frac{\omega_v}{2} \cos 2\theta_v \sigma_3 + \frac{\omega_v}{2} \sin 2\theta_v \sigma_1  + \left( \frac{1}{2}\Lambda_0  + \frac{1}{2}  \delta \lambda \right) \sigma_3,\end{split}
\end{equation*}
where we define
\begin{equation*}
\begin{split}\delta \lambda = \sum_{q=1}^{\infty} \Lambda_q \cos\left( \omega_0 q x \right).\end{split}
\end{equation*}
This profile is exactly a multi-frequency matter profile.

Rotate to the background matter basis
\begin{equation*}
\begin{split}H &= - \frac{1}{2}\omega_m \sigma_3 + \frac{1}{2} \delta\lambda \sigma_3 - \frac{1}{2} \delta \lambda \sigma_1 \\
&= - \frac{1}{2}\omega_m \sigma_3  + \frac{1}{2} \sum_{q=1}^{\infty} \Lambda_q \cos\left( \omega_0 q x \right) \cos 2\theta_m \sigma_3 - \frac{1}{2} \sum_{q=1}^{\infty} \Lambda_q \cos\left( \omega_0 q x \right) \sin 2\theta_m \sigma_1.\end{split}
\end{equation*}

\subsection{Refs \& Notes}
\label{\detokenize{matter-stimulated/parametric-resonance-revisted:refs-notes}}

\section{Single Frequency Matter Perturbation}
\label{\detokenize{matter-stimulated/single-frequency::doc}}\label{\detokenize{matter-stimulated/single-frequency:single-frequency-matter-perturbation}}
As a first step, we solve single frequency matter perturbation
\begin{equation*}
\begin{split}\delta \lambda(x)  = A \sin (k x + \phi).\end{split}
\end{equation*}
Using the relation between \(\eta\) and \(\delta\lambda\), we solve out \(\eta\).
\begin{equation*}
\begin{split}\eta(x) = - \frac{\omega_m}{2}x - \frac{\cos 2\theta_m}{2} \frac{A}{k} \cos (k x + \phi),\end{split}
\end{equation*}
where we have chosen \(\eta(0)=-\frac{\cos 2\theta_m}{2}\frac{A}{k}\cos\phi\).

The problem is to solve the equation of motion
\begin{equation*}
\begin{split}i \frac{d}{dx} \begin{pmatrix} \psi_{b1} \\ \psi_{b2} \end{pmatrix} = \frac{\sin 2\theta_m}{2}\delta\lambda(x) \begin{pmatrix} 0 &  e^{2i\eta(x)} \\   e^{-2i\eta(x)} &  0 \end{pmatrix}  \begin{pmatrix} \psi_{b1} \\ \psi_{b2} \end{pmatrix} .\end{split}
\end{equation*}
We also define
\phantomsection\label{\detokenize{matter-stimulated/single-frequency:equation-single-frequency-hamiltonian-element}}\begin{equation}\label{equation:matter-stimulated/single-frequency:single-frequency-hamiltonian-element}
\begin{split}h &= \frac{\sin 2\theta_m}{2}\delta\lambda(x)  e^{2i\eta(x)} \\
& = \frac{\sin 2\theta_m}{2} A \sin (kx+\phi) e^{i\left( -\omega_m x - \frac{A \cos 2\theta_m}{k} \cos (kx+\phi) \right)},\end{split}
\end{equation}
so that the equation of motion becomes
\begin{equation*}
\begin{split}i \frac{d}{dx} \begin{pmatrix} \psi_{b1} \\ \psi_{b2} \end{pmatrix} =  \begin{pmatrix} 0 &  h \\   h^* &  0 \end{pmatrix}  \begin{pmatrix} \psi_{b1} \\ \psi_{b2} \end{pmatrix} .\end{split}
\end{equation*}
Obviously, the exponential terms is too complicate. On the other hand, this equation of motion reminds us of the Rabi oscillation. So we decide to rewrite the exponential into some plane wave terms using Jacobi-Anger expansion. (Refs \& Notes: Patton et al)

\begin{sphinxadmonition}{note}{Jacobi-Anger Expansion}

One of the forms of Jacobi-Anger expansion is
\phantomsection\label{\detokenize{matter-stimulated/single-frequency:equation-jacobi-anger-expansion}}\begin{equation}\label{equation:matter-stimulated/single-frequency:jacobi-anger-expansion}
\begin{split}e^{i z \cos (\Phi)} = \sum_{n=-\infty}^\infty i^n J_n(z) e^{i n\Phi}.\end{split}
\end{equation}\end{sphinxadmonition}

We define \(z_k = \frac{A}{k} \cos 2\theta_m\), with which we expand the term
\begin{equation*}
\begin{split}e^{-i\frac{\cos 2\theta_m A}{k} \cos (kx +\phi)} = \sum_{n=-\infty}^\infty i^n J_n (-z_k) e^{in (kx +\phi)} =  \sum_{n=-\infty}^\infty (-i)^n J_n (z_k) e^{in (kx +\phi)},\end{split}
\end{equation*}
where I used \(J_n(-z_k) = (-1)^n J_n(z_k)\) for integer \(n\).

The expansion is plugged into the Hamiltonian elements,
\begin{equation*}
\begin{split}h &= \frac{A \sin 2\theta_m \sin (kx + \phi)}{2} e^{-i\omega_m x } \sum_{n = - \infty}^\infty (-i)^n J_n(z_k) e^{i n ( kx + \phi)} \\
& = \frac{A\sin 2\theta_m}{4i} \left( e^{i(kx + \phi)} - e^{-i(kx+\phi)} \right) e^{-i\omega_m x } \sum_{n = - \infty}^\infty (-i)^n J_n(z_k) e^{i n ( kx + \phi)} \\
& = \frac{A\sin 2\theta_m}{4i} \left( \sum_{n=-\infty}^\infty e^{i(n+1)} i^n J_n (z_k) e^{i((n+1) k - \omega_m)x}  - \sum_{n'=-\infty}^\infty e^{i(n'-1)} (-i)^{n'}J_{n'}(z_k) e^{i( (n'-1)k - \omega_m)x}  \right)\\
& = \frac{A\sin 2\theta_m}{4} \sum_{n=-\infty}^{\infty} e^{in\phi} \left( - (-i)^n \right) \frac{2n}{z_k} J_n (z_k) e^{i(nk-\omega_m)x},\end{split}
\end{equation*}
where I have used
\begin{equation*}
\begin{split}J_{n-1}(z_k) + J_{n+1}(z_k) = \frac{2n}{z_k} J_n(z_k).\end{split}
\end{equation*}
Here comes the approximation. The most important oscillation we need is the one with largest period, which indicates the phase to be almost zero,
\phantomsection\label{\detokenize{matter-stimulated/single-frequency:equation-single-frequency-rwa-requirement}}\begin{equation}\label{equation:matter-stimulated/single-frequency:single-frequency-rwa-requirement}
\begin{split}(n+1) k -\omega_m &\sim 0 \\
(n'-1) k -\omega_m &\sim 0.\end{split}
\end{equation}
The two such conditions for the two summations are
\begin{equation*}
\begin{split}n \equiv n_- &= \mathrm{Int}\left( \frac{\omega_m}{k} \right) - 1 \\
n' \equiv n_+ &= \mathrm{Int}\left( \frac{\omega_m}{k} \right) + 1 .\end{split}
\end{equation*}
We define \(\mathrm{Int}\left( \frac{\omega_m}{k} \right) = n_0\),
\begin{equation*}
\begin{split}n_- &= n_0 - 1 \\
n_+ &= n_0 + 1 .\end{split}
\end{equation*}
The element of Hamiltonian is written as
\begin{equation*}
\begin{split}h = - \frac{A\sin 2\theta_m}{2} e^{in_0\phi} (-i)^{n_0} \frac{n_0}{z_k} J_{n_0 }(z_k) e^{i(n_0 k -\omega_m)x}.\end{split}
\end{equation*}
To save keystrokes, we define
\phantomsection\label{\detokenize{matter-stimulated/single-frequency:equation-definition-F}}\begin{equation}\label{equation:matter-stimulated/single-frequency:definition-F}
\begin{split}F = - A\sin 2\theta_m e^{i n_0 \phi} (-i)^{n_0} \frac{n_0}{z_k} J_{n_0} (z_k) ,\end{split}
\end{equation}
which depends on \(n_0\) and \(z_k = \frac{A}{k} \cos 2\theta_m\). Notice that
\begin{equation*}
\begin{split}\lvert F \rvert^2 = \left\lvert  k \tan 2\theta_m  n_0 J_{n_0} (z_k) \right\rvert^2 .\end{split}
\end{equation*}
Thus the 12 element of the Hamiltonian is rewritten as
\phantomsection\label{\detokenize{matter-stimulated/single-frequency:equation-eqn-12-element-and-F}}\begin{equation}\label{equation:matter-stimulated/single-frequency:eqn-12-element-and-F}
\begin{split}h = \frac{1}{2}F e^{i(n_0 k -\omega_m)x}.\end{split}
\end{equation}
\begin{sphinxadmonition}{note}{Solving Using Mathematica}

The Mathematica code:

\begin{sphinxVerbatim}[commandchars=\\\{\}]
\PYG{n}{In}\PYG{p}{[}\PYG{l+m+mi}{1}\PYG{p}{]}\PYG{p}{:}\PYG{o}{=} \PYG{n}{sys} \PYG{o}{=} \PYG{n}{I} \PYG{n}{D}\PYG{p}{[}\PYG{p}{\PYGZob{}}\PYG{n}{phi1}\PYG{p}{[}\PYG{n}{x}\PYG{p}{]}\PYG{p}{,} \PYG{n}{phi2}\PYG{p}{[}\PYG{n}{x}\PYG{p}{]}\PYG{p}{\PYGZcb{}}\PYG{p}{,} \PYG{n}{x}\PYG{p}{]} \PYG{o}{==} \PYG{p}{\PYGZob{}}\PYG{p}{\PYGZob{}}\PYG{l+m+mi}{0}\PYG{p}{,} \PYG{p}{(}\PYG{n}{g0R} \PYG{o}{+} \PYG{n}{I} \PYG{n}{g0I}\PYG{p}{)} \PYG{n}{Exp}\PYG{p}{[} \PYG{n}{I} \PYG{p}{(}\PYG{o}{\PYGZhy{}}\PYG{n}{omegam} \PYG{o}{+} \PYG{n}{n0} \PYG{n}{k}\PYG{p}{)} \PYG{n}{x}\PYG{p}{]}\PYG{p}{\PYGZcb{}}\PYG{p}{,} \PYG{p}{\PYGZob{}}\PYG{p}{(}\PYG{n}{g0R} \PYG{o}{\PYGZhy{}} \PYG{n}{I} \PYG{n}{g0I}\PYG{p}{)} \PYG{n}{Exp}\PYG{p}{[}\PYG{o}{\PYGZhy{}}\PYG{n}{I} \PYG{p}{(}\PYG{o}{\PYGZhy{}}\PYG{n}{omegam} \PYG{o}{+} \PYG{n}{n0} \PYG{n}{k}\PYG{p}{)} \PYG{n}{x}\PYG{p}{]}\PYG{p}{,} \PYG{l+m+mi}{0}\PYG{p}{\PYGZcb{}}\PYG{p}{\PYGZcb{}}\PYG{o}{.}\PYG{p}{\PYGZob{}}\PYG{n}{phi1}\PYG{p}{[}\PYG{n}{x}\PYG{p}{]}\PYG{p}{,} \PYG{n}{phi2}\PYG{p}{[}\PYG{n}{x}\PYG{p}{]}\PYG{p}{\PYGZcb{}}
\PYG{n}{In}\PYG{p}{[}\PYG{l+m+mi}{2}\PYG{p}{]}\PYG{p}{:}\PYG{o}{=} \PYG{n}{DSolve}\PYG{p}{[}\PYG{n}{sys}\PYG{p}{,} \PYG{p}{\PYGZob{}}\PYG{n}{phi1}\PYG{p}{,} \PYG{n}{phi2}\PYG{p}{\PYGZcb{}}\PYG{p}{,} \PYG{n}{x}\PYG{p}{]}\PYG{o}{/}\PYG{o}{/} \PYG{n}{FullSimplify}
\PYG{n}{Out}\PYG{p}{[}\PYG{l+m+mi}{3}\PYG{p}{]}\PYG{p}{:}\PYG{o}{=} \PYG{p}{\PYGZob{}}\PYG{p}{\PYGZob{}}\PYG{n}{phi1} \PYG{o}{\PYGZhy{}}\PYG{o}{\PYGZgt{}} \PYG{n}{Function}\PYG{p}{[}\PYG{p}{\PYGZob{}}\PYG{n}{x}\PYG{p}{\PYGZcb{}}\PYG{p}{,}
\PYG{n}{E}\PYG{o}{\PYGZca{}}\PYG{p}{(}\PYG{l+m+mi}{1}\PYG{o}{/}\PYG{l+m+mi}{2} \PYG{n}{I} \PYG{p}{(}\PYG{n}{k} \PYG{n}{n0} \PYG{o}{+} \PYG{n}{I} \PYG{n}{Sqrt}\PYG{p}{[}\PYG{o}{\PYGZhy{}}\PYG{l+m+mi}{4} \PYG{p}{(}\PYG{n}{g0I}\PYG{o}{\PYGZca{}}\PYG{l+m+mi}{2} \PYG{o}{+} \PYG{n}{g0R}\PYG{o}{\PYGZca{}}\PYG{l+m+mi}{2}\PYG{p}{)} \PYG{o}{\PYGZhy{}} \PYG{p}{(}\PYG{n}{k} \PYG{n}{n0} \PYG{o}{\PYGZhy{}} \PYG{n}{omegam}\PYG{p}{)}\PYG{o}{\PYGZca{}}\PYG{l+m+mi}{2}\PYG{p}{]} \PYG{o}{\PYGZhy{}} \PYG{n}{omegam}\PYG{p}{)} \PYG{n}{x}\PYG{p}{)} \PYG{n}{C}\PYG{p}{[}\PYG{l+m+mi}{1}\PYG{p}{]}
\PYG{o}{+} \PYG{n}{E}\PYG{o}{\PYGZca{}}\PYG{p}{(}\PYG{l+m+mi}{1}\PYG{o}{/}\PYG{l+m+mi}{2} \PYG{p}{(}\PYG{n}{Sqrt}\PYG{p}{[}\PYG{o}{\PYGZhy{}}\PYG{l+m+mi}{4} \PYG{p}{(}\PYG{n}{g0I}\PYG{o}{\PYGZca{}}\PYG{l+m+mi}{2} \PYG{o}{+} \PYG{n}{g0R}\PYG{o}{\PYGZca{}}\PYG{l+m+mi}{2}\PYG{p}{)} \PYG{o}{\PYGZhy{}} \PYG{p}{(}\PYG{n}{k} \PYG{n}{n0} \PYG{o}{\PYGZhy{}} \PYG{n}{omegam}\PYG{p}{)}\PYG{o}{\PYGZca{}}\PYG{l+m+mi}{2}\PYG{p}{]} \PYG{o}{+} \PYG{n}{I} \PYG{p}{(}\PYG{n}{k} \PYG{n}{n0} \PYG{o}{\PYGZhy{}} \PYG{n}{omegam}\PYG{p}{)}\PYG{p}{)} \PYG{n}{x}\PYG{p}{)} \PYG{n}{C}\PYG{p}{[}\PYG{l+m+mi}{2}\PYG{p}{]}\PYG{p}{]}\PYG{p}{,}
\PYG{n}{phi2} \PYG{o}{\PYGZhy{}}\PYG{o}{\PYGZgt{}} \PYG{n}{Function}\PYG{p}{[}\PYG{p}{\PYGZob{}}\PYG{n}{x}\PYG{p}{\PYGZcb{}}\PYG{p}{,} \PYG{p}{(}\PYG{l+m+mi}{1}\PYG{o}{/}\PYG{p}{(}\PYG{l+m+mi}{2} \PYG{p}{(}\PYG{n}{g0I} \PYG{o}{\PYGZhy{}} \PYG{n}{I} \PYG{n}{g0R}\PYG{p}{)}\PYG{p}{)}\PYG{p}{)}
\PYG{n}{I} \PYG{n}{E}\PYG{o}{\PYGZca{}}\PYG{p}{(}\PYG{o}{\PYGZhy{}}\PYG{n}{I} \PYG{p}{(}\PYG{n}{k} \PYG{n}{n0} \PYG{o}{\PYGZhy{}} \PYG{n}{omegam}\PYG{p}{)} \PYG{n}{x} \PYG{o}{+}
 \PYG{l+m+mi}{1}\PYG{o}{/}\PYG{l+m+mi}{2} \PYG{n}{I} \PYG{p}{(}\PYG{n}{k} \PYG{n}{n0} \PYG{o}{+} \PYG{n}{I} \PYG{n}{Sqrt}\PYG{p}{[}\PYG{o}{\PYGZhy{}}\PYG{l+m+mi}{4} \PYG{p}{(}\PYG{n}{g0I}\PYG{o}{\PYGZca{}}\PYG{l+m+mi}{2} \PYG{o}{+} \PYG{n}{g0R}\PYG{o}{\PYGZca{}}\PYG{l+m+mi}{2}\PYG{p}{)} \PYG{o}{\PYGZhy{}} \PYG{p}{(}\PYG{n}{k} \PYG{n}{n0} \PYG{o}{\PYGZhy{}} \PYG{n}{omegam}\PYG{p}{)}\PYG{o}{\PYGZca{}}\PYG{l+m+mi}{2}\PYG{p}{]} \PYG{o}{\PYGZhy{}} \PYG{n}{omegam}\PYG{p}{)} \PYG{n}{x}\PYG{p}{)}
 \PYG{p}{(}\PYG{n}{k} \PYG{n}{n0} \PYG{o}{+} \PYG{n}{I} \PYG{n}{Sqrt}\PYG{p}{[}\PYG{o}{\PYGZhy{}}\PYG{l+m+mi}{4} \PYG{p}{(}\PYG{n}{g0I}\PYG{o}{\PYGZca{}}\PYG{l+m+mi}{2} \PYG{o}{+} \PYG{n}{g0R}\PYG{o}{\PYGZca{}}\PYG{l+m+mi}{2}\PYG{p}{)} \PYG{o}{\PYGZhy{}} \PYG{p}{(}\PYG{n}{k} \PYG{n}{n0} \PYG{o}{\PYGZhy{}} \PYG{n}{omegam}\PYG{p}{)}\PYG{o}{\PYGZca{}}\PYG{l+m+mi}{2}\PYG{p}{]} \PYG{o}{\PYGZhy{}} \PYG{n}{omegam}\PYG{p}{)} \PYG{n}{C}\PYG{p}{[}\PYG{l+m+mi}{1}\PYG{p}{]}
 \PYG{o}{+} \PYG{p}{(}\PYG{l+m+mi}{1}\PYG{o}{/}\PYG{p}{(}\PYG{l+m+mi}{2} \PYG{p}{(}\PYG{n}{g0I} \PYG{o}{\PYGZhy{}} \PYG{n}{I} \PYG{n}{g0R}\PYG{p}{)}\PYG{p}{)}\PYG{p}{)} \PYG{n}{E}\PYG{o}{\PYGZca{}}\PYG{p}{(}\PYG{l+m+mi}{1}\PYG{o}{/}\PYG{l+m+mi}{2} \PYG{p}{(}\PYG{n}{Sqrt}\PYG{p}{[}\PYG{o}{\PYGZhy{}}\PYG{l+m+mi}{4} \PYG{p}{(}\PYG{n}{g0I}\PYG{o}{\PYGZca{}}\PYG{l+m+mi}{2} \PYG{o}{+} \PYG{n}{g0R}\PYG{o}{\PYGZca{}}\PYG{l+m+mi}{2}\PYG{p}{)} \PYG{o}{\PYGZhy{}} \PYG{p}{(}\PYG{n}{k} \PYG{n}{n0} \PYG{o}{\PYGZhy{}} \PYG{n}{omegam}\PYG{p}{)}\PYG{o}{\PYGZca{}}\PYG{l+m+mi}{2}\PYG{p}{]}
 \PYG{o}{+} \PYG{n}{I} \PYG{p}{(}\PYG{n}{k} \PYG{n}{n0} \PYG{o}{\PYGZhy{}} \PYG{n}{omegam}\PYG{p}{)}\PYG{p}{)} \PYG{n}{x} \PYG{o}{\PYGZhy{}} \PYG{n}{I} \PYG{p}{(}\PYG{n}{k} \PYG{n}{n0} \PYG{o}{\PYGZhy{}} \PYG{n}{omegam}\PYG{p}{)} \PYG{n}{x}\PYG{p}{)} \PYG{p}{(}\PYG{n}{Sqrt}\PYG{p}{[}\PYG{o}{\PYGZhy{}}\PYG{l+m+mi}{4} \PYG{p}{(}\PYG{n}{g0I}\PYG{o}{\PYGZca{}}\PYG{l+m+mi}{2} \PYG{o}{+} \PYG{n}{g0R}\PYG{o}{\PYGZca{}}\PYG{l+m+mi}{2}\PYG{p}{)} \PYG{o}{\PYGZhy{}} \PYG{p}{(}\PYG{n}{k} \PYG{n}{n0} \PYG{o}{\PYGZhy{}} \PYG{n}{omegam}\PYG{p}{)}\PYG{o}{\PYGZca{}}\PYG{l+m+mi}{2}\PYG{p}{]}
 \PYG{o}{+} \PYG{n}{I} \PYG{p}{(}\PYG{n}{k} \PYG{n}{n0} \PYG{o}{\PYGZhy{}} \PYG{n}{omegam}\PYG{p}{)}\PYG{p}{)} \PYG{n}{C}\PYG{p}{[}\PYG{l+m+mi}{2}\PYG{p}{]}\PYG{p}{]}\PYG{p}{\PYGZcb{}}\PYG{p}{\PYGZcb{}}
\end{sphinxVerbatim}
\end{sphinxadmonition}

The general solution to the equation of motion is
\begin{equation*}
\begin{split}\psi_{b1} = & C_1 e^{\frac{1}{2} i \left( n_0 k -\omega_m - \sqrt{  \lvert F \rvert^2 +  (n_0 k -\omega_m)^2 } \right)x} + C_2 e^{\frac{1}{2} i \left( n_0 k -\omega_m + \sqrt{  \lvert F \rvert^2 +  (n_0 k -\omega_m)^2 } \right)x} \\
\psi_{b2} = & \frac{C_1}{F^*} i \left( n_0 k - \omega_m - \sqrt{ \lvert F\rvert^2 + ( n_0 k - \omega_m )^2 } \right) e^{ -\frac{1}{2}i (n_0 k - \omega_m ) x - \frac{1}{2} i \sqrt{ \lvert F \rvert^2 + (n_0 k - \omega_m )^2 }  } \\
& + \frac{C_2}{F^*} i \left( n_0 k - \omega_m + \sqrt{ \lvert F\rvert^2 + ( n_0 k - \omega_m )^2 }  \right)   e^{ -\frac{1}{2}i (n_0 k - \omega_m ) x + \frac{1}{2} i \sqrt{ \lvert F \rvert^2 + (n_0 k - \omega_m )^2 }  } .\end{split}
\end{equation*}
For simplicity, we define
\phantomsection\label{\detokenize{matter-stimulated/single-frequency:equation-definition-g-q}}\begin{equation}\label{equation:matter-stimulated/single-frequency:definition-g-q}
\begin{split}g &= n_0 k  - \omega_m, \\
q^2 &= \lvert F \rvert^2 + g^2.\end{split}
\end{equation}
To determine the constants, we need intial condition,
\begin{equation*}
\begin{split}\begin{pmatrix} \psi_1 (0) \\ \psi_2(0)  \end{pmatrix} = \begin{pmatrix} 1 \\ 0  \end{pmatrix} ,\end{split}
\end{equation*}
which leads to
\begin{equation*}
\begin{split}\begin{pmatrix} \psi_{b1} (0) \\ \psi_{b2}(0)  \end{pmatrix} = \begin{pmatrix} e^{i\eta(0)} \\ 0  \end{pmatrix},\end{split}
\end{equation*}
using equation {\hyperref[\detokenize{matter-stimulated/index:matter-stimulated-equation-wavefunction-diff-basis}]{\sphinxcrossref{\DUrole{std,std-ref}{wavefunction in background matter basis}}}}.

Plug in the initial condition for the wave function,
\begin{equation*}
\begin{split}C_1 + C_2 &= e^{i \frac{z_k}{2}\cos \phi} \\
\frac{C_1}{2F^ * } i \left( g - q \right) + \frac{C _ 2}{ F ^ *} i \left( q + g  \right) & = 0.\end{split}
\end{equation*}
The constants are solved out
\begin{equation*}
\begin{split}C_1 &= e^{i \frac{z_k}{2}\cos \phi} \frac{q + g }{2 q} , \\
C_2 &= e^{i \frac{z_k}{2}\cos \phi} \frac{ q - g }{2 q}.\end{split}
\end{equation*}
where \(F\) is defined in \eqref{equation:matter-stimulated/single-frequency:definition-F} and \(l\) and \(g\) are defined in \eqref{equation:matter-stimulated/single-frequency:definition-g-q}.

The second element of wave function becomes
\begin{equation*}
\begin{split}\psi_{b2}(x) = \frac{- F}{ q } e^{i\frac{z_k}{2} \cos\phi} e^{- \frac{i}{2}gx} \sin \left( \frac{1}{2} q x \right).\end{split}
\end{equation*}
The transition probability becomes
\begin{equation*}
\begin{split}P_{1\to 2} = \lvert \psi_{b2} \rvert^2 = \frac{\lvert F \rvert^2}{q^2} \sin^2\left( \frac{ q }{2} x \right),\end{split}
\end{equation*}
where \(q\) is the oscillation wavenumber. Period of this oscillation is given by \(T = \frac{2\pi}{q}\).

\begin{sphinxadmonition}{note}{Compare The Result with Kneller et al}

Kneller et al have a transition probability
\begin{equation*}
\begin{split}\color{red}P_{12} = \frac{\kappa_n^2}{q_n^2} \sin^2 (q_n x),\end{split}
\end{equation*}
where \(\color{red}q_n^2 = k_n^2 + \kappa_n^2\) and \(\color{red}2k_n = \tilde{\delta k}_{12} + n k_\star\).

In my notation, \(k\) is the same as their \(\color{red}k_\star\). After the first step of translation, we have \(g = \color{red} 2 k_n\).

The definition of \(\color{red}\kappa_n\) is given by

in Kneller's notation and
\begin{equation*}
\begin{split}\delta V_{ee}(x) = C_\star V_\star \sin (k_\star x + \eta).\end{split}
\end{equation*}
So we conclude that my \(\lvert F \rvert ^2\) is related to Kneller's \(\lvert \kappa_n \rvert^2\) through
\begin{equation*}
\begin{split}\lvert F \rvert^2 = 4 \color{red} \lvert \kappa_n \rvert^2.\end{split}
\end{equation*}
We also have
\begin{equation*}
\begin{split}q^2 = \lvert F \rvert ^2 + g^2  = 4 \color{red} q_n^2,\end{split}
\end{equation*}
i.e., \({\color{red}q_n} = \frac{ q }{2}\).

Now we see the method we have used gives exactly the same transition probability as Kneller's.
\end{sphinxadmonition}

To make the numerical calculations easier, we rewrite the result by defining the scaled variables
\begin{equation*}
\begin{split}\hat x & = \omega_m x,\\
\hat k &= \frac{k}{\omega_m}, \\
\hat A & = \frac{A}{\omega_m}, \\
\hat g & = \frac{g}{\omega_m} = n_0 \hat k - 1,\\
\hat q &= \sqrt{ \lvert \hat F \rvert^2 + \hat g^2 } = \sqrt{ \lvert \hat k \tan 2\theta_m n_0 J_{n_0} (z_k) \rvert + \hat g^2 },\end{split}
\end{equation*}
so that \(n_0 = \mathrm{Round}\left( 1/\hat k\right)\), \(z_k=\frac{\hat A}{\hat k} \cos 2 \theta_m\) and
\phantomsection\label{\detokenize{matter-stimulated/single-frequency:single-frequency-equation-stimulated-single-freq-trans-probability}}\phantomsection\label{\detokenize{matter-stimulated/single-frequency:equation-stimulated-single-freq-trans-probability}}\begin{equation}\label{equation:matter-stimulated/single-frequency:stimulated-single-freq-trans-probability}
\begin{split}P_{1\to 2} = \frac{\left\lvert \hat k \tan 2\theta_m n_0 J_{n_0} (z_k) \right\rvert^2}{\left\lvert  \hat k \tan 2\theta_m n_0 J_{n_0} (z_k) \right\rvert^2 + \hat g ^2}\sin^2\left( \frac{ \hat q }{2} \hat x \right) .\end{split}
\end{equation}

\subsection{Mathematical Analysis of The Result}
\label{\detokenize{matter-stimulated/single-frequency:mathematical-analysis-of-the-result}}
There are several question to answer before we can understand the picture of the math.
\begin{enumerate}
\item {} 
What does each term mean in the Hamiltonian?

\item {} 
What exactly is the unitary transformation we used to rotate the wave function?

\item {} 
What is the physical meaning of Jacobi-Anger expansion in our calculation?

\end{enumerate}

To answer the first question, we need to write down the solution to Schrodinger equation assuming the Hamiltonian has only one term. The results are listed below.

\noindent\begin{tabulary}{\linewidth}{|L|L|}
\hline
\sphinxstylethead{\relax 
Hamiltonian
\unskip}\relax &\sphinxstylethead{\relax 
Solution to The First Element of Wave Function
\unskip}\relax \\
\hline\begin{equation*}
\begin{split}-\frac{\omega_m}{2}\sigma_3\end{split}
\end{equation*}&\begin{equation*}
\begin{split}\psi_1 \sim e^{i\omega_m x/2}\end{split}
\end{equation*}\\
\hline\begin{equation*}
\begin{split}\frac{\delta\lambda}{2}\cos 2\theta_m \sigma_3\end{split}
\end{equation*}&\begin{equation*}
\begin{split}\psi_1 \sim e^{i\frac{A\cos 2\theta_m}{2k}\cos(kx+\phi)}\end{split}
\end{equation*}\\
\hline\begin{equation*}
\begin{split}\frac{\delta\lambda}{2}\cos 2\theta_m \sigma_3\end{split}
\end{equation*}&\begin{equation*}
\begin{split}\psi_1 = C_1 e^{i\frac{A\sin 2\theta_m}{2k}\cos(kx+\phi)} + C_2 e^{-i\frac{A\sin 2\theta_m}{2k}\cos(kx+\phi)}\end{split}
\end{equation*}\\
\hline\end{tabulary}


The unitary transformation used is to move our reference frame to a co-rotating one. \(-\frac{\omega_m}{2}\sigma_3\) is indeed causing the wave function to rotate and removing this term using a transformation means we are co-rotating with it. \(\frac{\delta\lambda}{2}\cos 2\theta_m \sigma_3\) causes a more complicated rotation however it is still a rotation.

As for Jacobi-Anger expansion, it expands an oscillating matter profile to infinite constant matter potentials. To see it more clearly, we assume that \(\delta\lambda= \lambda_c\) is constant. After the unitary transformation, the effective Hamiltonian is
\begin{equation*}
\begin{split}H' = \frac{\sin 2\theta_m}{2} \lambda_c \begin{pmatrix} 0 & e^{2i\eta(x)} \\ e^{-2i\eta(x)} & 0 \end{pmatrix},\end{split}
\end{equation*}
where \(\eta(x) = -\frac{\omega_m}{2}x + \frac{\cos 2\theta_m}{2}\lambda_c x\) and we have chosen \(\eta(0)=0\).

The 12 element of the Hamiltonian becomes
\begin{equation*}
\begin{split}\frac{\sin 2\theta_m}{2} \lambda_c e^{2i\eta(x)} = \frac{\sin 2\theta_m}{2} \lambda_c e^{2i\left( \frac{\omega_m}{2} + \frac{\cos 2\theta_m}{2} \lambda_c \right)x} .\end{split}
\end{equation*}
The significance of it is to show that a constant matter profile will result in a simple exponential term. However, as we move on to periodic matter profile, we have a Hamiltonian element of the form
\begin{equation*}
\begin{split}h = \frac{\sin 2\theta_m}{2} A \sin (kx+\phi) e^{2i\left( -\frac{\omega_m}{2} x + \frac{A \cos 2\theta_m}{2k} \cos (kx+\phi) \right)},\end{split}
\end{equation*}
as derived in equation \eqref{equation:matter-stimulated/single-frequency:single-frequency-hamiltonian-element}. To compare with the constant matter case, we make a table of relevant terms in Hamiltonian.

\noindent\begin{tabulary}{\linewidth}{|L|L|}
\hline
\sphinxstylethead{\relax 
Constant Matter Profile \(\delta\lambda = \lambda_c\)
\unskip}\relax &\sphinxstylethead{\relax 
Period Matter Profile \(\delta\lambda=A\sin (kx+\phi)\)
\unskip}\relax \\
\hline\begin{equation*}
\begin{split}\frac{\sin 2\theta_m}{2}\lambda_c e^{i \cos 2\theta_m \lambda_c x}\end{split}
\end{equation*}&\begin{equation*}
\begin{split}\frac{A\sin 2\theta_m}{4} \sum_{n=-\infty}^{\infty} e^{in\phi} \left( - i^n \right) \frac{2n+1}{z_k} J_n (z_k) e^{i(nk-\omega_m)x}\end{split}
\end{equation*}\\
\hline\end{tabulary}


The periodic profile comes into the exponential. Jacobi-Anger expansion (equation \eqref{equation:matter-stimulated/single-frequency:jacobi-anger-expansion}) expands the periodic matter profile into infinite constant matter profiles. By comparing the two cases, we conclude that \(\cos 2\theta_m\lambda_c\) corresponds to \(nk\).

The RWA approximation we used to drop fast oscillatory terms in the summation is to find the most relevant constant matter profile per se.

The big question is which constant matter profiles are the most important ones? Mathematically, we require the phase to be almost zero, i.e. equation \eqref{equation:matter-stimulated/single-frequency:single-frequency-rwa-requirement} or
\begin{equation*}
\begin{split}n_0 k - \omega_m \sim 0 ,\end{split}
\end{equation*}
where \(n_0=\mathrm{Round}\left( \frac{\omega_m}{k} \right)\).

\sphinxstylestrong{What is the meaning of this condition in this new basis?} If we define a effective matter density out of the Jacobi-Anger expanded series, we should define it to be
\begin{equation*}
\begin{split}\lambda_c' = \frac{n_0 k}{\cos 2\theta_m}.\end{split}
\end{equation*}
Then we can rewrite the RWA requirement as
\begin{equation*}
\begin{split}\lambda_c' - \cos 2\theta_m \omega_m = 0.\end{split}
\end{equation*}
\begin{sphinxadmonition}{note}{A Reminder of MSW Resonance}

The MSW Hamiltonian in flavor basis is
\begin{equation*}
\begin{split}\mathbf H = \frac{\omega_v}{2}( -\cos2\theta_v \sigma_3 + \sin 2\theta_v \sigma_1 )   {\color{red} + \frac{\lambda}{2} \mathbf {\sigma_3}}  {\color{green}+ \Delta \mathbf I},\end{split}
\end{equation*}
where the MSW resonance happens when all the \(\sigma_3\) terms cancel eath other, i.e.,
\begin{equation*}
\begin{split}- \omega_v \cos 2\theta_v  + \lambda = 0.\end{split}
\end{equation*}\end{sphinxadmonition}


\subsection{The Resonances}
\label{\detokenize{matter-stimulated/single-frequency:the-resonances}}
\begin{sphinxadmonition}{note}{Questions}

There are several questions to be answered.
\begin{enumerate}
\item {} 
How good is the RWA approximation? What are the conditions?

\item {} 
What can we use for other calculations?

\item {} 
Multiple matter frequency?

\end{enumerate}
\end{sphinxadmonition}

Now we check the Hamiltonian again to see if we could locate some physics. In the newly defined basis and using scaled quantities
\begin{equation*}
\begin{split}\hat{\mathbf{H}} = \begin{pmatrix}
0 & \hat h \\
\hat h^* & 0
\end{pmatrix},\end{split}
\end{equation*}
where
\begin{equation*}
\begin{split}\hat h = \frac{1}{2} \hat B_n e^{i(n \hat k - 1)\hat x},\end{split}
\end{equation*}
and
\begin{equation*}
\begin{split}\hat B_n = (-i)^n \hat k \tan 2\theta_{\mathrm{m}} n J_n (\frac{\hat A}{\hat k} \cos 2\theta_{\mathrm{m}}).\end{split}
\end{equation*}
It becomes much more clearer if we plug \(\hat h\) back into Hamiltonian. What we find is that
\begin{equation*}
\begin{split}\hat{\mathbf{H}} = \sum_{n=-\infty}^{\infty} \begin{pmatrix}
0 & -\frac{1}{2} \hat B_n e^{i(n \hat k - 1)\hat x} \\
-\frac{1}{2} \hat B_n^* e^{-i(n \hat k - 1)\hat x} & 0
\end{pmatrix}.\end{split}
\end{equation*}
With some effort, we find that the solution to the second amplitude of the wave function is
\begin{equation*}
\begin{split}\psi_2 = \frac{i}{ \hat B_n \hat W} e^{-i(n \hat k -1)\hat x}  \left\vert \hat B_n \right\vert^2 \sin\left( \frac{1}{2}(n \hat k -1 -\hat W) \hat x  \right),\end{split}
\end{equation*}
where
\begin{equation*}
\begin{split}\hat W = \sqrt{ (n \hat k - 1)^2 + \left\vert \hat B_n \right\vert^2 }.\end{split}
\end{equation*}
At this stage, it is quite obvious that our system is a composite Rabi oscilation system. For each specific \(n\) term we write down the hopping probability from light state to the heavy state,
\begin{equation*}
\begin{split}P_{\mathrm{L}\to\mathrm{H}}^{(n)} = \frac{ \left\lvert \hat B_{n}  \right\rvert^2 }{ \left\lvert   \hat B_{n}  \right\rvert^2 + ( n \hat k - 1 )^2  } \sin^2 \left( \frac{ \hat q^{(n)} }{2} \hat x \right),\end{split}
\end{equation*}
where
\begin{equation*}
\begin{split}\Gamma^{(n)} &= \left\lvert \hat B_{n} \right\rvert, \quad \text{width of resonance ($n\hat k$ as parameter)} \\
\hat q^{(n)} &= \sqrt{\left\lvert  \Gamma^{(n)} \right\rvert^2 + ( n \hat k - 1 )^2},\quad \text{frequency of oscillations}\end{split}
\end{equation*}
\begin{sphinxadmonition}{note}{Scaled Quantities}

As a reminder, the scaled quantities are defined as
\begin{equation*}
\begin{split}\hat x &= \omega_{\mathrm{m}} x, \\
\hat h &= h/\omega_{\mathrm{m}}, \\
\hat B_n &= B_n/\omega_{\mathrm{m}}, \\
\hat k &= k/\omega_{\mathrm{m}}, \\
\hat A &= A/\omega_{\mathrm{m}}, \\
\hat q &= q/\omega_{\mathrm{m}} .\end{split}
\end{equation*}
Just a comment. \(B_n\) is used here because I actually want to use it for multi-frequency case and I just think \(B_n\) is better than \(F\).
\end{sphinxadmonition}

\begin{sphinxadmonition}{note}{Rabi Oscillations}

The form of Hamiltonian reminds of us Rabi oscillations, whose Hamiltonian is
\begin{equation*}
\begin{split}\begin{pmatrix}
-\omega_0/2 & \alpha \omega_0 e^{i \omega x}\\
\alpha \omega_0  e^{-i k x} & \omega_0/2
\end{pmatrix},\end{split}
\end{equation*}
where \(\omega_0\) is the energy gap between the two energy levels and \(\omega\) is the frequency of the incoming light. To be more specific, we explain this phenomenon using two level systems.
\begin{figure}[htbp]
\centering
\capstart

\noindent\sphinxincludegraphics{{rabi-diagram}.png}
\caption{Rabi oscillation system}\label{\detokenize{matter-stimulated/single-frequency:id2}}\end{figure}

The system is prepared in low energy state. When the incoming light frequency matches the energy gap between two states, we have a resonance. Otherwise, we still have the transition from low energy state to high energy state but with a smaller transition amplitude.
\begin{figure}[htbp]
\centering
\capstart

\noindent\sphinxincludegraphics{{rabi-oscillations}.png}
\caption{Rabi oscillations for two differen incoming light frequencies. \(\omega/\omega_0 =1\) is the resonance condition.}\label{\detokenize{matter-stimulated/single-frequency:id3}}\end{figure}

What we really mean by resonance is that the transition amplitude is maximized. Or the phase inside the off-diagonal element of Hamiltonian is minimized.

What's more, we explore the transition amplitude as a function of differen incoming light frequencies.
\begin{figure}[htbp]
\centering
\capstart

\noindent\sphinxincludegraphics{{rabi-resonance}.png}
\caption{Resonance of Rabi oscillations.}\label{\detokenize{matter-stimulated/single-frequency:id4}}\end{figure}
\end{sphinxadmonition}
\begin{figure}[htbp]
\centering
\capstart

\noindent\sphinxincludegraphics{{stimulated-probability-apmlitude-vs-k}.svg}
\caption{Probabiity Amplitude as a function of \(k/\omega_m\) within RWA, with parameters \(A=0.1, \theta_m=\pi/5, \phi=0\).}\label{\detokenize{matter-stimulated/single-frequency:id5}}\end{figure}
\begin{figure}[htbp]
\centering
\capstart

\noindent\sphinxincludegraphics{{stimulated-probability-apmlitude-vs-k-non-RWA}.svg}
\caption{Probabiity Amplitude as a function of \(k/\omega_m\) for each term in Jacobi-Anger expansion, with parameters \(A=0.1, \theta_m=\pi/5, \phi=0\).}\label{\detokenize{matter-stimulated/single-frequency:id6}}\end{figure}

To look at the resonances I define a Mathematica function to calculate the FWHM.

\begin{sphinxadmonition}{note}{Find FWHM Using Mathematica}

The Mathematica code:

\begin{sphinxVerbatim}[commandchars=\\\{\}]
\PYG{n}{fwhm}\PYG{p}{[}\PYG{n}{n\PYGZus{}}\PYG{p}{]} \PYG{p}{:}\PYG{o}{=} \PYG{n}{First}\PYG{n+nd}{@Differences}\PYG{p}{[}\PYG{n}{k} \PYG{o}{/}\PYG{o}{.} \PYG{p}{\PYGZob{}}\PYG{n}{ToRules}\PYG{n+nd}{@Reduce}\PYG{p}{[}\PYG{n}{amplitude}\PYG{p}{[}\PYG{n}{k}\PYG{p}{,} \PYG{l+m+mf}{0.1}\PYG{p}{,} \PYG{n}{Pi}\PYG{o}{/}\PYG{l+m+mi}{5}\PYG{p}{,} \PYG{n}{n}\PYG{p}{]} \PYG{o}{==} \PYG{l+m+mf}{0.5} \PYG{o}{\PYGZam{}}\PYG{o}{\PYGZam{}}\PYG{n}{k} \PYG{o}{\PYGZgt{}} \PYG{p}{(}\PYG{l+m+mi}{1} \PYG{o}{\PYGZhy{}} \PYG{l+m+mf}{0.5}\PYG{o}{/}\PYG{n}{Exp}\PYG{p}{[}\PYG{n}{n}\PYG{p}{]}\PYG{o}{/}\PYG{n}{n}\PYG{o}{\PYGZca{}}\PYG{l+m+mi}{2}\PYG{p}{)}\PYG{o}{/}\PYG{n}{n} \PYG{o}{\PYGZam{}}\PYG{o}{\PYGZam{}} \PYG{n}{k} \PYG{o}{\PYGZlt{}} \PYG{p}{(}\PYG{l+m+mi}{1} \PYG{o}{+} \PYG{l+m+mf}{0.5}\PYG{o}{/}\PYG{n}{Exp}\PYG{p}{[}\PYG{n}{n}\PYG{p}{]}\PYG{o}{/}\PYG{n}{n}\PYG{o}{\PYGZca{}}\PYG{l+m+mi}{2}\PYG{p}{)}\PYG{o}{/}\PYG{n}{n}\PYG{p}{,} \PYG{n}{k}\PYG{p}{]}\PYG{p}{\PYGZcb{}}\PYG{p}{]}
\end{sphinxVerbatim}
\end{sphinxadmonition}
\begin{figure}[htbp]
\centering
\capstart

\noindent\sphinxincludegraphics{{stimulated-probability-apmlitude-vs-k-resonance-width}.svg}
\caption{Width of the resonances for \(A=0.1, \theta_m=\pi/5, \phi=0\).}\label{\detokenize{matter-stimulated/single-frequency:id7}}\end{figure}

How do we understand the resonance? Resonance width of each order of resonance (each n) should be calculated analytically.

\begin{sphinxadmonition}{note}{Lorentzian Distribution}

Three-parameter Lorentzian function is
\begin{equation*}
\begin{split}f_{x_0,\sigma,A}(x)= \frac{1}{\pi} \frac{\sigma}{\sigma^2 + (x-x_0)^2},\end{split}
\end{equation*}
which has a width \(2\gamma\).
\end{sphinxadmonition}

To find the exact width is hopeless since we need to inverse Bessel functions. Nonethless, we can assume that the resonance is very narrow so that \(\left\lvert F \right\rvert^2\) doesn't change a lot. With the assumption, the FWHM is found be setting the amplitude to half, which is
\begin{equation*}
\begin{split}\Gamma = \left\lvert \frac{\hat F}{n_0} \right\rvert = \left\lvert \hat k \tan 2\theta_m \frac{ J_{n_0}( n_0 \hat A \cos 2\theta_m/\hat k )}{n_0} \right\rvert .\end{split}
\end{equation*}
To verify this result, we compare it with the width found numerically from the exact amplitude.

Given this result, and equation \eqref{equation:matter-stimulated/single-frequency:eqn-12-element-and-F}, we infer that the coefficient in front of the phase term of 12 element in Hamiltonian is related to the width, while the the deviation from the exact resonance is given by \(\hat g=n_0 \hat k - 1\).

\begin{sphinxadmonition}{note}{Guessing The Width}

Given a Hamiltonian 12 element here
\begin{equation*}
\begin{split}h = \frac{F}{2} e^{i(n_0 \hat k - 1) \hat x} = \frac{F}{2} e^{i \hat g \hat x},\end{split}
\end{equation*}
the width of the resonance is
\phantomsection\label{\detokenize{matter-stimulated/single-frequency:single-frequency-equation-eqn-single-frequency-width-guessing}}\phantomsection\label{\detokenize{matter-stimulated/single-frequency:equation-eqn-single-frequency-width-guessing}}\begin{equation}\label{equation:matter-stimulated/single-frequency:eqn-single-frequency-width-guessing}
\begin{split}\Gamma = \left\lvert \frac{F}{n_0} \right\rvert.\end{split}
\end{equation}\end{sphinxadmonition}
\begin{figure}[htbp]
\centering
\capstart

\noindent\sphinxincludegraphics{{stimulated-single-frequency-width-approximation-amp-point1}.png}
\caption{Comparison of approximated width and numerical results for perturbation amplitude \(\hat A = \frac{A}{\omega_m} = 0.1\).}\label{\detokenize{matter-stimulated/single-frequency:id8}}\end{figure}
\begin{figure}[htbp]
\centering
\capstart

\noindent\sphinxincludegraphics{{stimulated-single-frequency-width-approximation-amp-1}.png}
\caption{Comparison of approximated width and numerical results for perturbation amplitude \(\hat A = \frac{A}{\omega_m} = 1\).}\label{\detokenize{matter-stimulated/single-frequency:id9}}\end{figure}

\begin{sphinxadmonition}{note}{A Special Property of Bessel Function}

A special relation of Bessel function is that \phantomsection\label{\detokenize{matter-stimulated/single-frequency:id1}}{\hyperref[\detokenize{matter-stimulated/single-frequency:ploumistakis2009}]{\sphinxcrossref{{[}Ploumistakis2009{]}}}}
\begin{equation*}
\begin{split}J_n(n \sech \alpha) \sim \frac{ e^{n(\tanh\alpha - \alpha)} }{\sqrt{ 2\pi n \tanh \alpha } }\end{split}
\end{equation*}
for large \(n\). As a matter of fact, for all positive \(\alpha\), we always have \(\tanh \alpha - \alpha < 0\).

Using this relation and defining \(\sech \alpha = A \cos 2\theta_m\), which renders
\phantomsection\label{\detokenize{matter-stimulated/single-frequency:equation-eqn-width-alpha-solved}}\begin{equation}\label{equation:matter-stimulated/single-frequency:eqn-width-alpha-solved}
\begin{split}\alpha = 2 n \pi i + \ln \left(  \frac{ 1 \pm \sqrt{ -A^2 \cos^2 2\theta_m + 1 } }{ A\cos 2\theta_m } \right),\qquad n\in \mathrm{Integers},\end{split}
\end{equation}
where the Mathematica code to solve it is shown below,
\begin{quote}

In{[}1{]}:= Solve{[}Exp{[}z{]} + Exp{[}-z{]} == 2/(A Cos{[}2 Subscript{[}{[}Theta{]}, m{]}{]}), z{]} // FullSimplify
\end{quote}

we find out an more human readabale analytical expression for the width
\begin{equation*}
\begin{split}\Gamma = \left\lvert 2 \hat k \tan 2\theta_m \frac{ e^{n ( \tanh \alpha - \alpha )} }{n_0 \sqrt{2\pi n_0 \tanh \alpha} } \right\rvert\end{split}
\end{equation*}
where \(\alpha\) is solved out in \eqref{equation:matter-stimulated/single-frequency:eqn-width-alpha-solved}.

For small \(\alpha\), we have expansions for exponentials and hyperbolic functions \(\tanh \alpha \sim \alpha - \frac{\alpha^3}{3}\),
\begin{equation*}
\begin{split}\Gamma \asymp 2\tan 2\theta_m \frac{ e^{n \alpha^3/3} }{\sqrt{2\pi \alpha} n_0^{3/2}  }.\end{split}
\end{equation*}
However, it doesn't really help that much since \(n\) is large and no expansion could be done except for significantly small \(\alpha\).
\end{sphinxadmonition}

(\url{http://www.sciencedirect.com/science/article/pii/S0375960109007166})
Keywords: Pair production; Multiphoton processes; High intensity lasers


\subsection{Perturbation Amplitude and Transition Probability}
\label{\detokenize{matter-stimulated/single-frequency:perturbation-amplitude-and-transition-probability}}\begin{figure}[htbp]
\centering
\capstart

\noindent\sphinxincludegraphics{{pltPertAmpPertWaveNumTransitionAmp}.svg}
\caption{Transition probability amplitude at different perturbation amplitude and perturbation wavenumber.}\label{\detokenize{matter-stimulated/single-frequency:id10}}\end{figure}


\section{Two-frequency Matter Perturbation}
\label{\detokenize{matter-stimulated/two-frequency:two-frequency-matter-perturbation}}\label{\detokenize{matter-stimulated/two-frequency::doc}}
However, as we proceed to the more realistic matter profile, multi-frequency matter profiles are necessary. Generally, we choose the perturbation matter profile upon a constant background to be
\begin{equation*}
\begin{split}\delta \lambda(x) = \sum_n A_n \sin (k_n x + \phi_n).\end{split}
\end{equation*}
Using {\hyperref[\detokenize{matter-stimulated/index:matter-stimulated-equation-eta-x-general}]{\sphinxcrossref{\DUrole{std,std-ref}{definition}}}} of \(\eta\) we conclude that
\begin{equation*}
\begin{split}\eta(x) = - \frac{\omega_m}{2}x - \frac{\cos 2\theta_m}{2} \sum_n \frac{A_n}{k_n} \cos ( k_n x+ \phi_n ).\end{split}
\end{equation*}
Hence we write down
\begin{equation*}
\begin{split}h = \frac{\sin 2\theta_m}{2} \sum_a A_a \sin (k_a x + \phi_a) e^{-i\omega_m x}\prod_{a} \sum_{n=-\infty}^{\infty} (-i)^n J_n (z_{k_a}) e^{i n(k_a x + \phi_a) }\end{split}
\end{equation*}

\subsection{Two Frequencies}
\label{\detokenize{matter-stimulated/two-frequency:two-frequencies}}
To work out the equation of motion that we could solve, we deal with two frequencies first,
\begin{equation*}
\begin{split}\delta \lambda ( x ) = A_1\sin (k_1 x + \phi_1) + A_2 \sin (k_2 x + \phi_2),\end{split}
\end{equation*}
while
\begin{equation*}
\begin{split}h = \frac{\sin 2\theta_m}{2} \sum_{a = 1}^2 A_a \sin (k_a x + \phi_a) e^{-i\omega_m x}\prod_{a=1}^2 \sum_{n=-\infty}^{\infty} (-i)^n J_n (z_{k_a}) e^{i n(k_a x + \phi_a) }.\end{split}
\end{equation*}

\subsubsection{Rewrite Multiplication into Summation}
\label{\detokenize{matter-stimulated/two-frequency:rewrite-multiplication-into-summation}}
\begin{sphinxadmonition}{note}{Summation Algebra}

A multiplication of two summations
\phantomsection\label{\detokenize{matter-stimulated/two-frequency:equation-multiplication-summation-rule}}\begin{equation}\label{equation:matter-stimulated/two-frequency:multiplication-summation-rule}
\begin{split}\sum_n a_n \sum_m b_m  = \sum_{N = -\infty}^{\infty} \sum_{m+n=N} a_n b_m = \sum_{N=-\infty}^\infty \sum_{n=-\infty}^{N} a_n b_{N-n}.\end{split}
\end{equation}
The rule is to sum over a line \(m+n=N\) then sum over \(N\).
\begin{figure}[htbp]
\centering
\capstart

\noindent\sphinxincludegraphics{{summation-algebra}.svg}
\caption{Rewrite multiplication of summations into summations only.}\label{\detokenize{matter-stimulated/two-frequency:id1}}\end{figure}
\end{sphinxadmonition}

The multiplication becomes
\phantomsection\label{\detokenize{matter-stimulated/two-frequency:two-frequency-equation-stimulated-multi-freq-hamiltonian-12-element}}\phantomsection\label{\detokenize{matter-stimulated/two-frequency:equation-stimulated-multi-freq-hamiltonian-12-element}}\begin{equation}\label{equation:matter-stimulated/two-frequency:stimulated-multi-freq-hamiltonian-12-element}
\begin{split}h &= \frac{\sin 2\theta_m}{2} \sum_{a = 1}^2 A_a \sin (k_a x + \phi_a) e^{-i\omega_m x} \sum_{N=-\infty}^{\infty} \sum_{n=-\infty}^{N} (-i)^n J_n (z_{k_1}) e^{i n(k_1 x + \phi_1) } (-i)^{N-n} J_{N-n}(z_{k_2}) e^{i (N-n)(k_2 x + \phi_2)} \\
&=\frac{\sin 2\theta_m}{2} \sum_{a = 1}^2 A_a \sin (k_a x + \phi_a) e^{-i\omega_m x} \sum_{N=-\infty}^{\infty} \sum_{n=-\infty}^{N} (-i)^N J_{n}(z_{k_1}) J_{N-n}(z_{k_2}) e^{i n ((k_1-k_2)x + \phi_1 - \phi_2) + i N (k_2 x + \phi_2)}\end{split}
\end{equation}
To proceed on, we rewrite \(\sum_{a = 1}^2 A_a \sin (k_a x + \phi_a)\),
\begin{equation*}
\begin{split}&A_1 \sin(k_1 x +\phi_1) + A_2 \sin(k_2 x +\phi_2) \\
= & \frac{A_1}{2i}\left( e^{i(k_1 x + \phi_1)} +  e^{-i(k_1 x + \phi_1)} \right) + \frac{A_2}{2i} \left( e^{i(k_2 x + \phi_2)} +  e^{-i(k_2 x + \phi_2)} \right).\end{split}
\end{equation*}
We define
\begin{equation*}
\begin{split}h = h_1 + h_2,\end{split}
\end{equation*}
where
\begin{equation*}
\begin{split}h_1 =& \frac{A_1\sin 2\theta_m}{4i}\bigg( \sum_{N=-\infty}^\infty \sum_{n=-\infty}^N (-i)^N J_n(z_{k_1}) J_{N-n}(z_{k_2}) e^{ i  \left(  (n+1) (k_1 x + \phi_1) +  (N-n)(k_2 x + \phi_2) - \omega_m x \right) } \\
& \sum_{N=-\infty}^\infty \sum_{n=-\infty}^N (-i)^N J_n(z_{k_1}) J_{N-n}(z_{k_2}) e^{ i \left(  (n-1) (k_1 x + \phi_1) + (N-n)(k_2 x + \phi_2) -  \omega_m x \right) }  \bigg),\end{split}
\end{equation*}
and
\begin{equation*}
\begin{split}h_2=& \frac{A_2\sin 2\theta_m}{4i}\bigg( \sum_{N=-\infty}^\infty \sum_{n=-\infty}^N (-i)^N J_n(z_{k_1}) J_{N-n}(z_{k_2}) e^{ i  \left(  n (k_1 x + \phi_1) + (N-n+1)(k_2 x + \phi_2) -  \omega_m x \right) } \\
& \sum_{N=-\infty}^\infty \sum_{n=-\infty}^N (-i)^N J_n(z_{k_1}) J_{N-n}(z_{k_2}) e^{ i  \left(  n (k_1 x + \phi_1) + (N-n-1)(k_2 x + \phi_2) -  \omega_m x \right) }  \bigg).\end{split}
\end{equation*}
To adopt the RWA approximation, we require some integers for each summation to satisfy the relations
\begin{equation*}
\begin{split}(n_{11,N} + 1)k_1 + (N-n_{11,N}) k_2 -\omega_m &\sim 0 \\
(n_{12,N} - 1)k_1 + (N-n_{12,N}) k_2 -\omega_m &\sim 0 \\
n_{21,N}k_1 + (N-n_{21,N}+1) k_2 -\omega_m &\sim 0 \\
n_{22,N} k_1 + (N-n_{22,N}-1) k_2 -\omega_m &\sim 0.\end{split}
\end{equation*}
so that the \(x\) dependent exponential almost vanishes (obtain the largest wavelength in fact). Notice that each \(n_{ij,N}\) depends on the summation index \(N\).

We solve each \(n_{ij,N}\),
\begin{equation*}
\begin{split}n_{11,N} &\sim \mathrm{Round}\left[\frac{\omega_m - N k_2 -k_1}{k_1 - k_2} \right] \\
n_{12,N} &\sim \mathrm{Round}\left[\frac{\omega_m - N k_2 + k_1}{k_1 - k_2}\right] \\
n_{21,N} &\sim \mathrm{Round}\left[\frac{\omega_m - (N + 1) k_2 }{k_1 - k_2} \right]\\
n_{22,N} &\sim \mathrm{Round}\left[\frac{\omega_m - (N - 1) k_2 }{k_1 - k_2} \right].\end{split}
\end{equation*}
Another important constrain is that \(n\leq N\), thus we have
\begin{equation*}
\begin{split}N_{11} &\sim \mathrm{Round}\left[\frac{\omega_m - k_1}{k_1}\right] \\
N_{12} &\sim \mathrm{Round}\left[\frac{\omega_m + k_1}{k_1}\right] \\
N_{21} &\sim \mathrm{Round}\left[\frac{\omega_m - k_2}{k_1}\right] \\
N_{22} &\sim \mathrm{Round}\left[\frac{\omega_m + k_2}{k_1}\right],\end{split}
\end{equation*}
for each summation over \(N\) and we require \(N\geq N_{ij}\) for each summation. We also assumed \(k_1 > k_2\). In other words, \(N_{ij}\) are the lower limits of the summations over \(N\)`s.

Using RWA, we keep only the resonance terms for the summation over \(n\)`s,
\begin{equation*}
\begin{split}h_1 \approx & \frac{A_1\sin 2\theta_m}{4i} \bigg( \sum_{N=N_{11}}^\infty (-i)^N J_{n_{11}} (z_{k_1}) J_{N-n_{11}}(z_{k_2}) e^{ i \left(  (n_{11}+1) (k_1 x + \phi_1) + (N-n_{11})(k_2 x + \phi_2) - \omega_m x \right) }   \\
& \sum_{N=N_{12}}^\infty (-i)^N J_{n_{12}} (z_{k_1}) J_{N-n_{12}}(z_{k_2}) e^{ i \left(  (n_{12}-1) (k_1 x + \phi_1) + (N-n_{12})(k_2 x + \phi_2) - \omega_m x \right) }\bigg),\end{split}
\end{equation*}
and
\begin{equation*}
\begin{split}h_2 \approx & \frac{A_2\sin 2\theta_m}{4i} \bigg( \sum_{N=N_{21}}^\infty (-i)^N J_{n_{21}} (z_{k_1}) J_{N-n_{21}}(z_{k_2}) e^{ i \left(  n_{21} (k_1 x + \phi_1) + (N-n_{21} + 1)(k_2 x + \phi_2) - \omega_m x \right) }   \\
& \sum_{N=N_{22}}^\infty (-i)^N J_{n_{22}} (z_{k_1}) J_{N-n_{22}}(z_{k_2}) e^{ i \left(  n_{22} (k_1 x + \phi_1) + (N-n_{22}-1)(k_2 x + \phi_2) - \omega_m x \right) }\bigg) .\end{split}
\end{equation*}
\begin{sphinxadmonition}{note}{Comment on This Result}

I can imagine how hard it is to solve the equation of motion with this \(h\). Well, is it?
\end{sphinxadmonition}


\subsubsection{One by One Approximation}
\label{\detokenize{matter-stimulated/two-frequency:one-by-one-approximation}}
After reading Kelly Patton et al, we decided to try the approximation they are using.

By looking at the Hamiltonian, we can identify terms like this
\begin{equation*}
\begin{split}h_a = \left( \frac{\sin 2\theta_m}{2}  A_a \sin (k_a x + \phi_a) e^{-i\omega_m x}  \sum_{n=-\infty}^{\infty} (-i)^n J_n (z_{k_a}) e^{i n(k_a x + \phi_a) }  \right) \prod_{b\neq a} \sum_{n=-\infty}^{\infty} (-i)^n J_n (z_{k_b}) e^{i n(k_b x + \phi_b) },\end{split}
\end{equation*}
where the parenthensis part is what we would have if only one frequency is used and we also have
\begin{equation*}
\begin{split}h = \sum_{a} h_a,\end{split}
\end{equation*}
for all frequencies.

This reminds us that each of these terms means the interference due to other frequencies. As a simple example, we demonstrate two-frequency case.

The two-frequency matter perturbation system has a Hamiltonian element \(H_{12}\)
\begin{equation*}
\begin{split}h = h_1 + h_2,\end{split}
\end{equation*}
where
\begin{equation*}
\begin{split}h_1 & = {\color{blue}-\frac{k_1 \tan 2\theta_m}{2} \sum_{n_1=-\infty}^{\infty} (-i)^{n_1} n_1 J_{n_1} (z_{k_1}) e^{i (n_1 k_1-\omega_m)x} e^{i n_1\phi_1} } {\color{red} \sum_{n_2=-\infty}^{\infty} (-i)^{n_2} J_{n_2}(z_{k_2}) e^{i(n_2k_2)x} e^{i n_2 \phi_2}  }, \\
h_2 & = {\color{red} - \frac{k_2\tan 2\theta_m}{2} \sum_{n_2=-\infty}^{\infty} (-i)^{n_2} n_2 J_{n_2}(z_{k_2}) e^{i(n_2k_2-\omega_m)x} e^{i n_2 \phi_2}  }{\color{blue} \sum_{n_1=-\infty}^{\infty} (-i)^{n_1} J_{n_1}(z_{k_1}) e^{in_1 k_1 x} e^{i n_1 \phi_1} }\end{split}
\end{equation*}
with red color coding the second frequency and blue coding the first frequency. \(h\) is symmetric under exchange of index 1, 2 since the exchange simply switches \(h_1\) and \(h_2\).


\subsection{Which one Dominates?}
\label{\detokenize{matter-stimulated/two-frequency:which-one-dominates}}
To grasp a clue, we need to identify which term in the summation dominates. Without a good analytical analysis, the only way to do is to numerically calculate the effect of each order.

By order, we are already thinking of a dominating term which is not true. Nonethless, we assume RWA can be applied to the part that looks like one frequency only. In our two-frequency example, first RWA leads to
\phantomsection\label{\detokenize{matter-stimulated/two-frequency:equation-eqn-after-first-rwa}}\begin{equation}\label{equation:matter-stimulated/two-frequency:eqn-after-first-rwa}
\begin{split}h_1 & \approx {\color{blue}-\frac{k_1 \tan 2\theta_m}{2} (-i)^{n_{1,0}} n_{1,0} J_{n_{1,0}} (z_{k_1}) e^{i (n_{1,0} k_1-\omega_m)x} e^{i n_{1,0}\phi_1} } {\color{red} \sum_{n_2=-\infty}^{\infty} (-i)^{n_2} J_{n_2}(z_{k_2}) e^{i(n_2k_2)x} e^{i n_2 \phi_2}  }, \\
h_2 & \approx {\color{red} - \frac{k_2\tan 2\theta_m}{2} (-i)^{n_{2,0}} n_{2,0} J_{n_{2,0}}(z_{k_2}) e^{i(n_{2,0}k_2-\omega_m)x} e^{i n_{2,0} \phi_2}  }{\color{blue} \sum_{n_1=-\infty}^{\infty} (-i)^{n_1} J_{n_1}(z_{k_1}) e^{in_1 k_1 x} e^{i n_1 \phi_1} },\end{split}
\end{equation}
where
\begin{equation*}
\begin{split}n_{1,0} &= \mathrm{Round}\left[  \frac{\omega_m}{k_1}  \right], \\
n_{2,0} &= \mathrm{Round}\left[  \frac{\omega_m}{k_2}  \right] .\end{split}
\end{equation*}
With this approximation, we can use RWA again by requiring
\begin{equation*}
\begin{split}(n_{1,0} k_1-\omega_m + n_2 k_2)x &\sim 0, \\
(n_1 k_1-\omega_m + n_{2,0} k_2)x &\sim 0,\end{split}
\end{equation*}
where the integer solutions are
\begin{equation*}
\begin{split}n'_{2,0} &= \mathrm{Round}\left[ \frac{ n_{1,0} k_1-\omega_m }{k_2} \right], \\
n'_{1,0} &= \mathrm{Round}\left[ \frac{ n_{2,0} k_1-\omega_m }{k_1} \right].\end{split}
\end{equation*}
Now we can remove all the summations using another RWA approximation. However, whether it holds is up to investigation.

The final result is
\begin{equation*}
\begin{split}h_1 & \approx {\color{blue}-\frac{k_1 \tan 2\theta_m}{2} (-i)^{n_{1,0}} n_{1,0} J_{n_{1,0}} (z_{k_1}) e^{i (n_{1,0} k_1-\omega_m)x} e^{i n_{1,0}\phi_1} } {\color{red} (-i)^{n'_{2,0}} J_{n'_{2,0}}(z_{k_2}) e^{i(n'_{2,0}k_2)x} e^{i n'_{2,0} \phi_2}  } \\
& = -\frac{k_1 \tan 2\theta_m}{2} (-i)^{ n_{1,0}+n'_{2,0} }   e^{i (n_{1,0}\phi_1 + n'_{2,0} \phi_2)} n_{1,0} J_{n_{1,0}} (z_{k_1})  J_{n'_{2,0}}(z_{k_2}) e^{i(n_{1,0} k_1-\omega_m + n'_{2,0}k_2)x}  \\
& \equiv \frac{F_1}{2} e^{i(n_{1,0} k_1-\omega_m + n'_{2,0}k_2)x}  , \\
h_2 & \approx {\color{red} - \frac{k_2\tan 2\theta_m}{2} (-i)^{n_{2,0}} n_{2,0} J_{n_{2,0}}(z_{k_2}) e^{i(n_{2,0}k_2-\omega_m)x} e^{i n_{2,0} \phi_2}  }{\color{blue} \sum_{n_1=-\infty}^{\infty} (-i)^{n_1} J_{n_1}(z_{k_1}) e^{in_1 k_1 x} e^{i n_1 \phi_1} } \\
& = - \frac{k_2\tan 2\theta_m}{2} (-i)^{n_{2,0}+ n'_{1,0} }   e^{i (n_{2,0} \phi_2 + n'_{1,0} \phi_1)}  n_{2,0} J_{n_{2,0}}(z_{k_2}) J_{n'_{1,0}}(z_{k_1})  e^{i(n_{2,0}k_2-\omega_m+ n'_{1,0} k_1)x} \\
& \equiv \frac{F_2}{2} e^{i(n_{2,0}k_2-\omega_m+ n'_{1,0} k_1)x} .\end{split}
\end{equation*}
Lowest order only works for very special cases where on of the wave vectors is very close to resonance. To fix this problem, we could add more higher orders, however, what does it mean to have higher orders needs a discussion.

\begin{sphinxadmonition}{note}{How to Include Higher Orders}

The first thought of higher orders is to add more from the summation before the last RWA. However, it is highly suspicious that this is just like the one frequency case which has a very fast drop in the resonance width as we go to higher orders.

This guess needs proof, numerically and analytically.

But notice, when we said higher orders, we actually mean higher orders in both \(n_1\) and \(n_2\). Notice that we can alway write the 12 element of Hamiltonian as \eqref{equation:matter-stimulated/two-frequency:2-freq-hamiltonian-12-element}, i.e.,
\begin{equation*}
\begin{split}h &= h_1 + h_2 \\
& = \sum_{n_1=-\infty}^\infty \sum_{n_2=-\infty}^{\infty} B_{n_1,n_2}(k_1,k_2) \Phi e^{i(n_1 k_1 + n_2 k_2 - \omega_m)x} +  \sum_{n_1=-\infty}^\infty \sum_{n_2=-\infty}^{\infty} B_{n_2,n_1}(k_2,k_1) \Phi e^{i(n_1 k_1 + n_2 k_2 - \omega_m)x} \\
& = \sum_{n_1=-\infty}^\infty \sum_{n_2=-\infty}^{\infty} \left( B_{n_1,n_2}(k_1,k_2) + B_{n_2,n_1}(k_2,k_1) \right) \Phi e^{i(n_1 k_1 + n_2 k_2 - \omega_m)x},\end{split}
\end{equation*}
without any approximations.
\end{sphinxadmonition}
\begin{enumerate}
\item {} 
One of the choices of adding higher orders is to use \(n_1=\mathrm{Round}\left[ \frac{\omega_m}{k_1} \right]\) and \(n_2=\mathrm{Round}\left[ \frac{ n_1  k_1 - \omega_m }{k_2} \right]\) as the lowest order in \(h_1\) and \(n_2=\mathrm{Round}\left[ \frac{\omega_m}{k_2} \right]\) and \(n_1=\mathrm{Round}\left[ \frac{ n_2  k_2 - \omega_m }{k_1} \right]\) as the lowest order in \(h_2\). Adding higher orders means we add or remove one from \(n_1\) in \(h_1\) and recalculate \(n_2\), while add or remove one from \(n_2\) in \(h_2\) and recalculate \(n_1\).

That is to say, we always keep the RWA condition for the last RWA process. What can be changed is the first assumption that the most important term is when only one frequency is relavent which is not always true.

As an example, we now consider \(n_{i,\pm 1}=n_{i,0}\pm 1\) and \(n'_{i,\pm 1} =  \mathrm{Round}\left[ \frac{ n_{j,\pm 1} k_j - \omega_m }{k_i} \right]\) with \(j\neq i\), thus we replace \(n_{i,0}\) with \(n_{i,\pm 1}\) to get higher order corrections.
\begin{figure}[htbp]
\centering
\capstart

\noindent\sphinxincludegraphics{{compApproxNum}.png}
\caption{Top Left: Smaller wavenumber \(k_1=0.95\) is at resonance and it has smaller perturbation amplitude (\(k_2=1.55\));
Top Right: Smaller wavenumber \(k_1=0.95\) is at resonance and it has larger perturbation amplitude (\(k_2=1.55\));
Bottom Left: Larger wavenumber \(k_2=0.95\) is at resonance and it has smaller perturbation amplitude (\(k_1=0.35\));
Bottom Right: Larger wavenumber \(k_2=0.95\) is at resonance and it has larger perturbation amplitude (\(k_1=0.35\)).
Red dotted line is numerical solution, black line is lowest approximation of \(k_2\), magenta is higher order approximation of \(k_2\).}\label{\detokenize{matter-stimulated/two-frequency:id2}}\end{figure}

In real physical systems, it is more likely to have a matter profile so that we have the bottom left situation. In other words, RWA method breaks down in the most interesting case.

\item {} 
Another choice is to add or remove one for both \(n_1\) and \(n_2\) for both terms in the Hamiltonian. The approach will define the order \(n_{order}\) first, as will be applied to the n's. As an example, adding first order to \(n_1\) will include all the possible combinations of \(n_1,n_1\pm 1\) for both terms without changing \(n_2\). As an example, we compare the different orders of \(n_1\) only with the numerical calculation without approximations.
\begin{figure}[htbp]
\centering
\capstart

\noindent\sphinxincludegraphics{{stimulated-2-freq-higher-orders-approach-2}.png}
\caption{Compare the different orders with the numerical calculation without approximations, where red dotted line is the numerical calculation without approximation. As we could see from the figure, including up to third order in \(n_1\) fixes the deviation from numerical calculation (red dotted line). The wave vectors are \(k_1=0.5\), \(k_2=0.8\), amplitudes are \(A_1=0.1 k_1^{-5/3}\), \(A_2=0.1 k_2^{-5/3}\), mixing angle in background matter is \(\theta_m=\pi/5\).}\label{\detokenize{matter-stimulated/two-frequency:id3}}\end{figure}

\item {} 
Now according to the complate expression of the 12 element of Hamiltonian \eqref{equation:matter-stimulated/two-frequency:2-freq-hamiltonian-12-element}, there is no difference between \(n_1\) and \(n_2\). Thus whenever we talk about different orders, we should not distinguish between the two integers. However, how to define zero order is not clear to me at this point. To find out, we need to know the resonance width of each pair of integers. The insight comes from the single frequency result. We notice in equation {\hyperref[\detokenize{matter-stimulated/single-frequency:single-frequency-equation-eqn-single-frequency-width-guessing}]{\sphinxcrossref{\DUrole{std,std-ref}{single frequency width}}}}, single frequency width depends on the coefficient in front of the phase in the Hamiltonian and the integer. The task is to derive or guess the resonance width for each pair of integers \(n_1, n_2\).

\end{enumerate}

\begin{sphinxadmonition}{note}{Which Approximation Breaks Down}

We ask the question, which approximation is breaking down exactly during our RWA? To find out, we first include all the orders after the first assumption, i.e., we do not use RWA for the second time, which means \eqref{equation:matter-stimulated/two-frequency:eqn-after-first-rwa} holds but no RWA will be applied to this.

Not notice that the summation in \eqref{equation:matter-stimulated/two-frequency:eqn-after-first-rwa} is due to the Jacobi-Anger expansion, which is not even helpful in our next calculation. Therefore, we trace back to their original expressions, which leads to
\begin{equation*}
\begin{split}h_1 & \approx {\color{blue}-\frac{k_1 \tan 2\theta_m}{2} (-i)^{n_{1,0}} n_{1,0} J_{n_{1,0}} (z_{k_1}) e^{i (n_{1,0} k_1-\omega_m)x} e^{i n_{1,0}\phi_1} } {\color{red} e^{-i z_{k_2} \cos(k_2 x+\phi_2)  } }, \\
h_2 & \approx {\color{red} - \frac{k_2\tan 2\theta_m}{2} (-i)^{n_{2,0}} n_{2,0} J_{n_{2,0}}(z_{k_2}) e^{i(n_{2,0}k_2-\omega_m)x} e^{i n_{2,0} \phi_2}  }{\color{blue}  e^{-i z_{k_1} \cos(k_1 x+\phi_1)  } }.\end{split}
\end{equation*}
We then perform a numerical calculation using this Hamiltonian element and compare it with the full numerical results.
\end{sphinxadmonition}


\subsection{A More Systematic Thinking of 2-Frequency}
\label{\detokenize{matter-stimulated/two-frequency:a-more-systematic-thinking-of-2-frequency}}
The 12 element can be written as
\begin{equation*}
\begin{split}h = h_1 + h_2,\end{split}
\end{equation*}
where
\begin{equation*}
\begin{split}h_1 &=\sum_{n_1=-\infty}^\infty \sum_{n_2=-\infty}^{\infty}\left( -(-i)^{n_1+n_2}\frac{\tan 2\theta_m}{2} n_1 k_1 J_{n_1}(z_{k_1}) J_{n_2}(z_{k_2})  \right) e^{i(n_1\phi_1+n_2\phi_2)} e^{i(n_1 k_1 + n_2 k_2 - \omega_m)x}, \\
h_2 &=\sum_{n_1=-\infty}^\infty \sum_{n_2=-\infty}^{\infty}\left( -(-i)^{n_1+n_2}\frac{\tan 2\theta_m}{2} n_2 k_2 J_{n_1}(z_{k_1}) J_{n_2}(z_{k_2})  \right) e^{i(n_1\phi_1+n_2\phi_2)} e^{i(n_1 k_1 + n_2 k_2 - \omega_m)x}.\end{split}
\end{equation*}
For simplicity, we define
\begin{equation*}
\begin{split}B_{n_1,n_2}(k_1,k_2,A_1,A_2) &= -(-i)^{n_1+n_2} \tan 2\theta_m n_1 k_1 J_{n_1}(z_{k_1}) J_{n_2}(z_{k_2}) = -(-i)^{n_1+n_2} \tan 2\theta_m n_1 k_1 J_{n_1}(\frac{A_1}{k_1}\cos 2\theta_m) J_{n_2}(\frac{A_2}{k_2}\cos 2\theta_m)  ,\\
\Phi & = e^{i(n_1\phi_1+n_2\phi_2)}.\end{split}
\end{equation*}
Notice that
\begin{equation*}
\begin{split}B_{n_2,n_1}(k_2,k_1, A_2, A_1) &= -(-i)^{n_1+n_2} \tan 2\theta_m n_2 k_2 J_{n_1}( \frac{A_1}{k_1}\cos 2\theta_m ) J_{n_2}( \frac{A_2}{k_2}\cos 2\theta_m ).\end{split}
\end{equation*}
Using these definitions, we rewrite the Hamiltonian 12 element
\phantomsection\label{\detokenize{matter-stimulated/two-frequency:equation-2-freq-hamiltonian-12-element}}\begin{equation}\label{equation:matter-stimulated/two-frequency:2-freq-hamiltonian-12-element}
\begin{split}h &= h_1 + h_2 \\
& = \frac{1}{2}\sum_{n_1=-\infty}^\infty \sum_{n_2=-\infty}^{\infty} B_{n_1,n_2}(k_1,k_2,A_1,A_2) \Phi e^{i(n_1 k_1 + n_2 k_2 - \omega_m)x} +  \frac{1}{2}\sum_{n_1=-\infty}^\infty \sum_{n_2=-\infty}^{\infty} B_{n_2,n_1}(k_2,k_1,A_2,A_1) \Phi e^{i(n_1 k_1 + n_2 k_2 - \omega_m)x} \\
& = \frac{1}{2}\sum_{n_1=-\infty}^\infty \sum_{n_2=-\infty}^{\infty} \left( B_{n_1,n_2}(k_1,k_2,A_1,A_2) + B_{n_2,n_1}(k_2,k_1,A_2,A_1) \right) \Phi e^{i(n_1 k_1 + n_2 k_2 - \omega_m)x}\\
& = \frac{1}{2}\sum_{n_1=-\infty}^\infty \sum_{n_2=-\infty}^{\infty} B_2{n_1,n_2}(k_1,k_2,A_1,A_2)\Phi e^{i(n_1 k_1 + n_2 k_2 - \omega_m)x},\end{split}
\end{equation}
where \(B_2{n_1,n_2}(k_1,k_2,A_1,A_2)\equiv B_{n_1,n_2}(k_1,k_2,A_1,A_2) + B_{n_2,n_1}(k_2,k_1,A_2,A_1)\) is what we are interested in.

Comparing this expression with the single frequency one which is almost the same structure if we remove the two sums, and using the result {\hyperref[\detokenize{matter-stimulated/single-frequency:single-frequency-equation-stimulated-single-freq-trans-probability}]{\sphinxcrossref{\DUrole{std,std-ref}{transition probability for single frequency}}}}, we can infer that the transition probability,
\begin{equation*}
\begin{split}P_{1\to 2}(x) = \frac{\lvert \hat B_2 \rvert^2}{ \lvert \hat B_2 \rvert^2 + \hat g_2^2} \sin^2\left( \frac{q_2}{2}x \right),\end{split}
\end{equation*}
where \(\hat B_2=\frac{ B_2 }{\omega_m}=\frac{B_{n_1,n_2}(k_1,k_2,A_1,A_2) + B_{n_2,n_1}(k_2,k_1,A_2,A_1)}{\omega_m}\) and \(\hat g_2 = \frac{g}{\omega_m} = n_1 \hat k_1 + n_2 \hat k_2 - 1\) which tells us how far from resonance and \(q_2=\sqrt{ \lvert \hat B_2 \rvert^2 + \hat g^2 }\).

The width then is similar to {\hyperref[\detokenize{matter-stimulated/single-frequency:single-frequency-equation-eqn-single-frequency-width-guessing}]{\sphinxcrossref{\DUrole{std,std-ref}{single frequency width}}}}, except that we could not define the width as a function of single variables since two wave vector are used. However, it is still reasonable to give the FWHM condition,
\phantomsection\label{\detokenize{matter-stimulated/two-frequency:equation-stimulated-2-freq-width-requirement-raw}}\begin{equation}\label{equation:matter-stimulated/two-frequency:stimulated-2-freq-width-requirement-raw}
\begin{split}n_1 \hat k_1 + n_2 \hat k_2 - 1 = \pm \lvert \hat B_2 \rvert = \left\lvert \frac{B_{n_1,n_2}(k_1,k_2,A_1,A_2) + B_{n_2,n_1}(k_2,k_1,A_2,A_1)}{\omega_m} \right\rvert.\end{split}
\end{equation}
For a given pair of integers \(n_1,n_2\), we could find the amplitude as a function of \(k_1, k_2\).

\begin{sphinxadmonition}{note}{Solve The Problem}

A solution shows that this is correct. The solution to the second element of wave function is
\begin{equation*}
\begin{split}\psi_{b2} = i \frac{ \lvert \hat B_2\rvert^2 e^{-\frac{i}{2} \hat g_2 \hat x} }{ \hat B_2 \sqrt{\lvert \hat B_2\rvert^2 + \hat g^2} }\sin\left( \frac{\sqrt{ \lvert \hat B_2 \rvert^2 + \hat g^2 }}{2}x \right)  .\end{split}
\end{equation*}\end{sphinxadmonition}

It is very confusing when we write down the requirement for width \eqref{equation:matter-stimulated/two-frequency:stimulated-2-freq-width-requirement-raw}, since we need to assume \(\lvert \hat F_2 \rvert\) to be almost constant to arrive this result. What values of \(\hat k_1,\hat k_2\) do we need to calculate \(\lvert \hat F_2 \rvert\)?

The idea is to find the FWHM when a point is deviating from the line. To be specific, we find the line that is the resonance using \(n_1 k_1 + n_2 k_2 = 1\), which is plotted as dashed red line in \hyperref[\detokenize{matter-stimulated/two-frequency:diagram-of-width-2-freq}]{Fig.\@ \ref{\detokenize{matter-stimulated/two-frequency:diagram-of-width-2-freq}}}. To characterise the distance, we need a line that is perpendicular to this red dashed resonance line, which also is passing through the values of \((k_10,k_2)=(k_{10},k_{20})\) which is given in the system. Under this scheme, the resonance width is define as the distance from the resonance line when the amplitude reduces to half on this blue dotted perpendicular line.
\begin{figure}[htbp]
\centering
\capstart

\noindent\sphinxincludegraphics{{stimulated-2-freq-width-diagram}.png}
\caption{Diagram of Width.}\label{\detokenize{matter-stimulated/two-frequency:diagram-of-width-2-freq}}\label{\detokenize{matter-stimulated/two-frequency:id4}}\end{figure}

In the language of algebra, we could derive the interception point of the two lines, which is
\begin{equation*}
\begin{split}k_{1,\mathrm{intercept}} &= \frac{n_2^2 k_{10} + n_2 k_{20} + n_1 }{n_1^2 + n_2^2}, \\
k_{2,\mathrm{intercept}} &= \frac{n_1}{n_2}k_{1,\mathrm{intercept}} - \frac{1}{n_2},\end{split}
\end{equation*}
where \(k_{10}\) and \(k_{20}\) are the values given in the matter perturbation of the system.

Using this method, we can define a reasonable width for two frequency matter perturbation case,
\begin{equation*}
\begin{split}\Gamma_2 = \frac{B_2(k_{1,\mathrm{intercept}},k_{2,\mathrm{intercept}})}{\sqrt{n_1^2 + n_2^2}}.\end{split}
\end{equation*}
\begin{sphinxadmonition}{note}{Derivation of Width for 2 Frequency Matter Perturbation}

First of all, we assume that a point \((\hat k_{10},\hat k_{20})\) is a displace from the line by the FWHM \(\hat L\) in \(\hat k_2\), which means that, the line that is paralell to the resonance line and passing through the point \((\hat k_{10},\hat k_{20})\) is displaced by \(\hat L\) in \(\hat k_2\),
\begin{equation*}
\begin{split}n_1 \hat k_1 + n_2 \hat k_2 - n_2 \hat L = 1.\end{split}
\end{equation*}
We assume the width of resonance is not large so that we could use resonance values for \(\hat k_1, \hat k_2\). For FWHM, we require
\begin{equation*}
\begin{split}n_1 k_{1,\mathrm{intercept}} + n_2 k_{2,\mathrm{intercept}} -1  - n_2 \hat L = \lvert \hat B_2(k_{1,\mathrm{intercept}},k_{2,\mathrm{intercept}}) \rvert,\end{split}
\end{equation*}
where we could apply \(n_1 k_{1,\mathrm{intercept}} + n_2 k_{2,\mathrm{intercept}} -1 = 0\) because we assumed the width is narrow, thus
\begin{equation*}
\begin{split}- n_2 \hat L = \lvert \hat B_2(k_{1,\mathrm{intercept}},k_{2,\mathrm{intercept}}) \rvert.\end{split}
\end{equation*}
However, \(L\) is not the actually deviation from the interception point. We could calculate the actual deviation \(\Gamma_2\) on the blue line in figure \hyperref[\detokenize{matter-stimulated/two-frequency:diagram-of-width-2-freq}]{Fig.\@ \ref{\detokenize{matter-stimulated/two-frequency:diagram-of-width-2-freq}}}, which is given by
\begin{equation*}
\begin{split}\sqrt{n_1^2 + n_2^2} \Gamma_2 = n_2 L,\end{split}
\end{equation*}
i.e., we find the resonance width
\begin{equation*}
\begin{split}\Gamma_2 =  \frac{B_2(k_{1,\mathrm{intercept}},k_{2,\mathrm{intercept}})}{\sqrt{n_1^2 + n_2^2}}.\end{split}
\end{equation*}\end{sphinxadmonition}

To apply the width in a problem, we need to calculate the distance between the given point \((k_{10},k_{20})\) of the system to a certain resonance line which depends on \(n_1,n_2,A_1,A_2,\theta_m\). This is as simple as point to line distance, which is calculated using
\phantomsection\label{\detokenize{matter-stimulated/two-frequency:equation-stimulated-2-freq-distance-0}}\begin{equation}\label{equation:matter-stimulated/two-frequency:stimulated-2-freq-distance-0}
\begin{split}d = \frac{\lvert n_1 k_{10} + n_2 k_{20} - 1 \rvert}{\sqrt{n_1^2 + n_2^2} }.\end{split}
\end{equation}
Here comes the question: \sphinxstylestrong{what is the requirement for a pair of} \((n_1,n_2)\) \sphinxstylestrong{to be important?}

We answer this by defining a quantity that compares the distance from a certain resonance line with the width of this resonance line,
\begin{equation*}
\begin{split}Q_2 = \frac{d}{\Gamma_2}.\end{split}
\end{equation*}
\begin{sphinxadmonition}{note}{Caveats}

There are caveats when calculating the distance \(d\) or the width \(\Gamma_2\).

The first problem is the zeros. In special cases, \(n_1=0\) as an example, the distance \(d\) using the equation \eqref{equation:matter-stimulated/two-frequency:stimulated-2-freq-distance-0} will lead to infinities. Same thing happens to the width.

The solution is to treat the special cases seperately. As an result, we conclude that
\phantomsection\label{\detokenize{matter-stimulated/two-frequency:equation-stimulated-2-freq-distance-d}}\begin{equation}\label{equation:matter-stimulated/two-frequency:stimulated-2-freq-distance-d}
\begin{split}d=\begin{cases}
\frac{\lvert n_1 k_{10} + n_2 k_{20} -1 \rvert}{\sqrt{ n_1^2 + n_2 ^2 }}, & n_1\neq 0 \&\& n_2 \neq 0 \\
\infty , & & n_1= 0 \&\& n_2 = 0.
\end{cases}\end{split}
\end{equation}
The \(\infty\) is simply a defined value which is to ensure the final values of \(Q_2\) to be reasonable.

Meanwhile, the width can always be written as \(\Gamma_2 = \frac{B_2(k_{1,\mathrm{intercept}},k_{2,\mathrm{intercept}})}{\sqrt{n_1^2 + n_2^2}}.\) as long as \(n_1\neq 0\&\& n_2\neq 0\). However, what we mean by \(k_{1,\mathrm{intercept}},k_{2,\mathrm{intercept}}\) has special situations.

For \(n_1\neq 0\&\& n_2\neq 0\), we have the general solution
\begin{equation*}
\begin{split}k_{1,\mathrm{intercept}} &= \frac{n_2^2 k_{10} + n_2 k_{20} + n_1 }{n_1^2 + n_2^2}, \\
k_{2,\mathrm{intercept}} &= \frac{n_1}{n_2}k_{1,\mathrm{intercept}} - \frac{1}{n_2}.\end{split}
\end{equation*}
For \(n_1=0\&\& n_2\neq 0\), we have
\begin{equation*}
\begin{split}k_{1,\mathrm{intercept}} &= k_{10}, \\
k_{2,\mathrm{intercept}} &= \frac{1}{n_2}.\end{split}
\end{equation*}
Finally, for \(n_1\neq 0\&\& n_2 =0\), we need
\begin{equation*}
\begin{split}k_{1,\mathrm{intercept}} &=\frac{1}{n_1}, \\
k_{2,\mathrm{intercept}} &= k_{20}.\end{split}
\end{equation*}
As for \(n_1=0\&\& n_2=0\), we define the width to be zero.

One last thing,
\begin{equation*}
\begin{split}Q_2 = \begin{cases}
\frac{d}{\Gamma_2}, & \Gamma_2\neq 0 \\
\infty, & \Gamma_2 = 0\&\& d\neq 0\\
0, & \Gamma_2=0\&\& d = 0
\end{cases}\end{split}
\end{equation*}\end{sphinxadmonition}


\section{Multi-frequency}
\label{\detokenize{matter-stimulated/multi-frequency::doc}}\label{\detokenize{matter-stimulated/multi-frequency:multi-frequency}}

\subsection{3 Frequencies}
\label{\detokenize{matter-stimulated/multi-frequency:frequencies}}
To extend what we have done to more frequencies is easy. For 3 frequencies, the 12 element of Hamiltonian has such a pattern that
\begin{equation*}
\begin{split}h &= {\color{blue} \delta \lambda_1 \sum_{n_1} \cdots } {\color{red} \sum_{n_2} \cdots } {\color{cyan}\sum_{n_3} \cdots } + {\color{blue}\sum_{n_1} \cdots } {\color{red} \delta \lambda_2 \sum_{n_2} \cdots } {\color{cyan}\sum_{n_3} \cdots } + {\color{blue}\sum_{n_1} \cdots } {\color{red}  \sum_{n_2} \cdots } {\color{cyan} \delta \lambda_3 \sum_{n_3} \cdots } \\
& \equiv h_1 + h_2 + h_3.\end{split}
\end{equation*}
\begin{sphinxadmonition}{note}{Split Hamiltonian Strategy}

As what we have tried in 2-frequency case, these terms can be viewed as three terms each is the interference between one frequency and the other two.

First of all, we assume the most important terms for \(h_1,h_2,h_3\) are determined by the \(k_1,k_2,k_3\) alone resonance respectively. Then we need to explore which terms to keep in the other two summations. Since the other two summations in each term are symmetric, we would like to include the resonance condition for one of them.

As an example, the first assumption leads to
\begin{equation*}
\begin{split}h_1 \approx {\color{blue} \delta \lambda_1 \cdots(with n_{1,0} in it) } {\color{red} \sum_{n_2} \cdots } {\color{cyan}\sum_{n_3} \cdots }.\end{split}
\end{equation*}
Then we include the other resonances.

\sphinxstylestrong{A simple test using Mathematica shows that it doesn't work.}
\end{sphinxadmonition}

We continue to rewrite the Hamiltonian by combining all terms. First of all, we write down the general form of Hamiltonian 12 element for \(N\) frequencies which is similar to {\hyperref[\detokenize{matter-stimulated/two-frequency:two-frequency-equation-stimulated-multi-freq-hamiltonian-12-element}]{\sphinxcrossref{\DUrole{std,std-ref}{Hamiltonian of two frequency case}}}},
\begin{equation*}
\begin{split}h = \frac{\sin 2\theta_m}{2}\sum_{a=1}^{N}A_a\sin(k_a x + \phi_a) e^{-i\omega_m  x} \Pi_{a_1}^N \sum_{n_a=-\infty}^{\infty} (-i)^{n_a} J_{n_a}(\frac{A_a}{k_a}\cos 2\theta_m) e^{i n_a ( k_a x + \phi_a)} .\end{split}
\end{equation*}
For three frequency perturbation,
\begin{equation*}
\begin{split}h &= \frac{\sin 2\theta_m}{4i}\sum_{a=1}^{3}A_a( e^{i(k_a x + \phi_a - \omega_m x)} - e^{-i(k_a x + \phi_a + \omega_m x)} )  \Pi_{a_1}^N \sum_{n_a=-\infty}^{\infty} (-i)^{n_a} J_{n_a}(\frac{A_a}{k_a}\cos 2\theta_m) e^{i n_a ( k_a x + \phi_a)} .\end{split}
\end{equation*}
Elaborate more on the math, we have (\sphinxstylestrong{need a verification for this result})
\begin{equation*}
\begin{split}h = \frac{1}{2}\sum_{n_1=-\infty}^\infty \sum_{n_2=-\infty}^\infty \sum_{n_2=-\infty}^\infty \left( B_{3,1} + B_{3,2} + B_{3,3} \right) \Phi_3 e^{i(n_1 k_1 +n_2 k_2 + n_3 k_3 -\omega_m)x},\end{split}
\end{equation*}
where
\begin{equation*}
\begin{split}B_{3,1} &= - (-i)^{n_1+n_2+n_3} \tan 2\theta_m  n_1 k_1 J_{n_1}(\frac{A_1}{k_1}\cos 2\theta_m)J_{n_2}(\frac{A_2}{k_2}\cos 2\theta_m)J_{n_3}(\frac{A_3}{k_3}\cos 2\theta_m) \\
B_{3,2} & =  - (-i)^{n_1+n_2+n_3}  \tan 2\theta_m  n_2 k_2 J_{n_1}(\frac{A_1}{k_1}\cos 2\theta_m)J_{n_2}(\frac{A_2}{k_2}\cos 2\theta_m)J_{n_3}(\frac{A_3}{k_3}\cos 2\theta_m) \\
B_{3,3} & =  - (-i)^{n_1+n_2+n_3} \tan 2\theta_m  n_3 k_3 J_{n_1}(\frac{A_1}{k_1}\cos 2\theta_m)J_{n_2}(\frac{A_2}{k_2}\cos 2\theta_m)J_{n_3}(\frac{A_3}{k_3}\cos 2\theta_m)\\
\Phi &= e^{i(n_1\phi_1 + n_2\phi_2+n_3\phi_3)}.\end{split}
\end{equation*}
In other words,
\begin{equation*}
\begin{split}h = - \frac{1}{2}\sum_{n_1=-\infty}^\infty \sum_{n_2=-\infty}^\infty \sum_{n_2=-\infty}^\infty (-i)^{n_1+n_2+n_3} \left(  n_1 k_1 + n_2 k_2 + n_3 k_3 \right) J_{n_1}(\frac{A_1}{k_1}\cos 2\theta_m)J_{n_2}(\frac{A_2}{k_2}\cos 2\theta_m)J_{n_3}(\frac{A_3}{k_3}\cos 2\theta_m) \Phi_3 e^{i(n_1 k_1 +n_2 k_2 + n_3 k_3 -\omega_m)x}.\end{split}
\end{equation*}
\begin{sphinxadmonition}{note}{Are 2 Frequencies Enough to Solve 3 Frequency Problems}

We could try to use two frequencies to approach the actual 3 frequency problem first. To see the result, we first need to solve 3 frequency problem numerically without any approximation.

Then we choose pairs of integers to approach the 3 frequency system.

An calculations shows that this is not generally true, even though sometimes it is.
\end{sphinxadmonition}


\subsection{N Frequencies}
\label{\detokenize{matter-stimulated/multi-frequency:n-frequencies}}
Given a system with N perturbations
\begin{equation*}
\begin{split}\delta\lambda_N = \sum_{a=1}^N A_a \sin(k_a x + \phi_a),\end{split}
\end{equation*}
the Hamiltonian can be written as
\begin{equation*}
\begin{split}H = \begin{pmatrix}
0 & h_N \\
h_N^* & 0
\end{pmatrix},\end{split}
\end{equation*}
where the Hamiltonian 12 element is
\begin{equation*}
\begin{split}h_N = \frac{1}{2}\sum_{n_1=-\infty}^\infty \cdots \sum_{n_N=-\infty}^\infty B_N\Phi_N e^{i(\sum_a n_a k_a - \omega_m)x},\end{split}
\end{equation*}
where
\begin{equation*}
\begin{split}B_N &= -(-i)^{\sum_a n_a} \tan 2\theta_m \left( \sum_a n_a k_a \right) \left( \prod_a J_{n_a}\left( \frac{A_a}{k_a}\cos 2\theta_m \right) \right),\\
\Phi_N &= e^{i\left( \sum_a n_a \phi_a \right)}.\end{split}
\end{equation*}
It might be useful to notice that \((-i)^{\sum_a n_a} = (e^{-i \pi/2})^{\sum_a n_a} = e^{-i \pi\sum_a n_a/2}\). Such an phase won't change the final transition probability given the initial condition that the system is completely on the low energy state.

\begin{sphinxadmonition}{note}{To do}
\begin{enumerate}
\item {} 
To actually make some sense here. I need to find out which approximation breaks down in Kelly's paper; Ref to admonition Which Approximation Breaks Down. Check!

\item {} 
Use physical length scales to simplify/obscure the problem. Check!

\item {} 
How can small \(k\) destroy the resonance of large \(k\)? c.f., Kelly's PRD paper.

\end{enumerate}
\end{sphinxadmonition}


\subsection{Numerical Results}
\label{\detokenize{matter-stimulated/multi-frequency:numerical-results}}
For an example system with 6 frequencies, with background matter profile \(\lambda=10\lambda_{\mathrm{MSW}}=10\cos(2\theta_v)\omega_v\), \(\sin 2\theta_v = 0.093\), \(\delta m^2 = \delta m_{13}^2=2.6\times 10^{-15}\), energy \(10\mathrm{MeV}\)
\begin{equation*}
\begin{split}\hat k = \{2.6138748783741892734981263895723814892738971298790875814789472389461\\end{split}
\end{equation*}
29836741984650871234, 1.8, 1.4, 0.9, 0.7, 0.3\}

and perturbation amplitudes
\begin{equation*}
\begin{split}\hat a = 0.1\hat k^{-5/3} = \{0.0201616, 0.0375445, 0.057076, 0.119196, 0.181205, 0.743814\}.\end{split}
\end{equation*}\begin{figure}[htbp]
\centering
\capstart

\noindent\sphinxincludegraphics{{6-frequency-matter-profile-200-combinations}.png}
\caption{Matter profile with 6 frequencies}\label{\detokenize{matter-stimulated/multi-frequency:id1}}\end{figure}


\section{Interference}
\label{\detokenize{matter-stimulated/interference:interference}}\label{\detokenize{matter-stimulated/interference::doc}}
\begin{sphinxadmonition}{note}{Summary}

Overall the system has at least two limits,
\begin{enumerate}
\item {} 
strong interference regime (the wavelength of the perturbations are of the same orders);

\item {} 
low-interference regime (one of the perturbations has a wavelength much larger than the others').

\end{enumerate}

For either case, the system is explained within the ralm of superpositions of multiple Rabi oscillations.

As for the first case, th criteria is related to both the resonance itself and the size of physical system we are interested in.
\begin{enumerate}
\item {} 
Each mode has an associated resonance, and explained by a single Rabi oscillation. Higher orders have much smaller resonance width than low orders.

\item {} 
Whether a mode is important to the system is determined by Q value, which is defined as the ratio of distance to resonance and resonance width.

\item {} 
However a mode with a wavelength that is much larger than the system is not counted since the resonance has not accumulated any significant transition probabilities within the region of the physical system.

\end{enumerate}

\sphinxstylestrong{However, we do not have a cystal clear understanding of such a strong interference.}

The second case is essentially some limit of the strong interference. With Komogorov spectra of matter, the resonance is simplified in some sense. One of the limits is that one of the matter perturbations has much larger wavelength than the others, which will behave as a shift of the background matter density, within one wavelength of the small wavelength perturbations. Here we use two frequencies as examples, c.f. \hyperref[\detokenize{matter-stimulated/interference:interference-adding-new-destroy-resonance-q-value}]{Fig.\@ \ref{\detokenize{matter-stimulated/interference:interference-adding-new-destroy-resonance-q-value}}}.
\begin{enumerate}
\item {} 
Even if the short wavelength one is exactly on resonance, adding the second perturbation as a background shift could move the system out of the resonance. The math behind it is related to the amplitude,
\begin{equation*}
\begin{split}\frac{ \left\lvert  B_{n}  \right\rvert^2 }{ \left\lvert    B_{n}  \right\rvert^2 + ( n  k - \omega_m )^2  },\end{split}
\end{equation*}
where \(n  k - \omega_m = 0\) for the resonance situation. As we add in the second long wavelength perturbation, \(B_{n}\), \(\omega_m\) will change. Nonetheless the relative change in \(B_n\) is small while the change in \(nk-\omega_m\) is huge compared to \(B_n\) value, if the amplitude of the new perturbation is large enough, i.e., \(A_2 \gg \lvert B_n\rvert\).

\item {} 
The caveat is that the wavelength of the second perturbation \(k_2\) must be within a range much smaller than \(k_1\), and much larger than the resonance transition probability wavelength, roughly speaking, \(k_1\gg k_2 \gg \lvert B_n\rvert\).

\item {} 
The newly added long wavelength merely has any effect when the system is in a region \(\mathrm{mod}(k_2 x,\pi)\sim 0\), since \(A_2\sin(k_2x)\sim 0\) for these values. Such regions probabily will appear multiple times within the size of physical system. Then we see accumulation effect, so that the transition amplitude will rise as we go furthure out.

\end{enumerate}

\sphinxstylestrong{From this point of view, the resonance is never destroyed since the accumulation will always be there.}
\end{sphinxadmonition}

In the calculation of multi-frequency matter background,
\begin{equation*}
\begin{split}\delta\lambda_N = \sum_{a=1}^N A_a \sin(k_a x + \phi_a),\end{split}
\end{equation*}
we derived the equation of motion
\begin{equation*}
\begin{split}i \partial_x \Psi = H \Psi,\end{split}
\end{equation*}
where
\begin{equation*}
\begin{split}\Psi = \begin{pmatrix}
\psi_{b1} \\
\psi_{b2}
\end{pmatrix},\end{split}
\end{equation*}
is the wave function in the T basis (new basis unitary transform from background matter basis using T matrix) and
\begin{equation*}
\begin{split}H = \begin{pmatrix}
0 & h_N \\
h_N^* & 0
\end{pmatrix},\end{split}
\end{equation*}
where
\begin{equation*}
\begin{split}h_N = -\frac{1}{2}\sum_{n_1=-\infty}^\infty \cdots \sum_{n_N=-\infty}^\infty B_N\Phi_N e^{i(\sum_a^{N} n_a k_a - \omega_m)x},\end{split}
\end{equation*}
and
\begin{equation*}
\begin{split}B_N &= (-i)^{\sum_a n_a} \left( \sum_a n_a k_a \right) \left( \prod_a J_{n_a}\left( \frac{A_a}{k_a}\cos 2\theta_m \right) \right),\\
\Phi_N &= e^{i\left( \sum_a n_a \phi_a \right)}.\end{split}
\end{equation*}

\subsection{Apply Approximations to Two Frequencies}
\label{\detokenize{matter-stimulated/interference:apply-approximations-to-two-frequencies}}
We could identify the important combinations of n's \(\{n_1, \cdots, n_N \}\) so that we can approximate the actual result.

The important question is combining multiple lists of n's is not intuitively easy to understand. Here we consider the effect of adding new list of n's to the system.

For simplicity a simplified system will be used,
\begin{equation*}
\begin{split}i\partial_x  \begin{pmatrix}
\psi_{b1} \\
\psi_{b2}
\end{pmatrix} = \begin{pmatrix}
0 & A_1 e^{ i \phi_1 x } +A_2 e^{ i \phi_2 x } \\
A_1^* e^{ - i \phi_1 x } + A_2^* e^{ -i \phi_2 x } & 0
\end{pmatrix}
\begin{pmatrix}
\psi_{b1} \\
\psi_{b2}
\end{pmatrix}.\end{split}
\end{equation*}
Notice that \(\phi_n\) is always real but \(A_n\) are either real or pure imaginary.

The matrix equation can be rewritten as systems of differential equations,
\begin{equation*}
\begin{split}i \partial_x \psi_{b1} &= (A_1 e^{ i \phi_1 x } +A_2 e^{ i \phi_2 x } ) \psi_{b2}, \\
i \partial_x \psi_{b2} & = (A_1^* e^{ - i \phi_1 x } + A_2^* e^{ -i \phi_2 x }) \psi_{b1}.\end{split}
\end{equation*}
This set of equations is solved by rewrite them to a second order differential equation
\phantomsection\label{\detokenize{matter-stimulated/interference:equation-two-n-list-second-order-eq-o-m}}\begin{equation}\label{equation:matter-stimulated/interference:two-n-list-second-order-eq-o-m}
\begin{split}\partial_x^2 \psi_{b2} + \frac{ A_1^* e^{ - i \phi_1 x } + A_2^* e^{ -i \phi_2 x }) }{\partial_x (A_1^* e^{ - i \phi_1 x } + A_2^* e^{ -i \phi_2 x }) } \partial_x \psi_{b2} + (A_1^* e^{ - i \phi_1 x } + A_2^* e^{ -i \phi_2 x }) (A_1 e^{ i \phi_1 x } +A_2 e^{ i \phi_2 x } ) \psi_{b2}.\end{split}
\end{equation}
As a comparison, we also write down the equation for single n list,
\begin{equation*}
\begin{split}\partial_x^2 \psi_{b2} + \left( \frac{i}{\phi_1}  \right) \partial_x \psi_{b2} + \lvert A_1 \rvert^2 \psi_{b2} = 0.\end{split}
\end{equation*}
\begin{sphinxadmonition}{note}{Approximations?}

For single n list equation, the second term dominates if \(\phi_1\) is much smaller than 1. In the language of physics, the second term dominates if the system is very close to resonance.

\begin{sphinxadmonition}{note}{\(\phi_n\) and resonance}

\(\phi_n\) is in fact the deviation from exact resonance.
\begin{equation*}
\begin{split}\phi_n = \sum_{a}^N n_a k_a - \omega_m.\end{split}
\end{equation*}\end{sphinxadmonition}

In the multi-n-list case, this domination is easily destroyed. As an example, suppose we have \(A_1= A_2 = A\) and \(\phi_2 = 10^{10}\phi_1\), the second term in equation \eqref{equation:matter-stimulated/interference:two-n-list-second-order-eq-o-m} becomes
\begin{equation*}
\begin{split}\frac{ e^{ - i \phi_1 x } + e^{ -i 10^{10}\phi_1 x }) }{\partial_x (e^{ - i \phi_1 x } + e^{ -i 10^{10}\phi_1 x }) } \partial_x \psi_{b2} = \frac{ e^{ - i \phi_1 x } + e^{ -i 10^{10}\phi_1 x }) }{ (-i\phi_1 e^{ - i \phi_1 x } - i 10^{10}\phi_1 e^{ -i 10^{10}\phi_1 x }) } \partial_x \psi_{b2},\end{split}
\end{equation*}
which can be dropped on as long as \(\phi_1\) is not as small as \(10^{-10}\).

We exaggerated the situation.
\end{sphinxadmonition}


\subsection{Resonance Destroyed}
\label{\detokenize{matter-stimulated/interference:resonance-destroyed}}

\subsubsection{General Ideas}
\label{\detokenize{matter-stimulated/interference:general-ideas}}
We first investigate two frequencies. The Hamiltonian for a single frequency matter perturbation is
\begin{equation*}
\begin{split}\hat{\mathbf{H}} = \sum_{n=-\infty}^{\infty} \begin{pmatrix}
0 & \frac{1}{2} \hat B_n e^{i(n \hat k - 1)\hat x} \\
\frac{1}{2} \hat B_n^* e^{-i(n \hat k - 1)\hat x} & 0
\end{pmatrix},\end{split}
\end{equation*}
where
\begin{equation*}
\begin{split}\hat B_n = (-i)^n \tan 2\theta_{\mathrm{m}} n \hat k  J_n (\frac{\hat A}{\hat k} \cos 2\theta_{\mathrm{m}}).\end{split}
\end{equation*}
In some conditions, even we have on of the matter frequancy at resonance, this resonance could be destroyed when a new matter frequency is destroyed. Numerical calculations show that this could happen if the new perturbation is off resonance and with larger \(B_{n_2}\).

Let's first set this first perturbation at resonance. Suppose we add in another matter perturbation with a frequency which is higher, i.e., \(k_2\ll k\). Since the wavelength of this new perturbation is much larger, we will assume it doesn't change within one wavelength of the first perturbation. Thus it behaves as an additional background. Will this new background destroy the resonance? Illustration of this idea is shown in \hyperref[\detokenize{matter-stimulated/interference:interference-adding-new-destroy-resonance-q-value}]{Fig.\@ \ref{\detokenize{matter-stimulated/interference:interference-adding-new-destroy-resonance-q-value}}}.
\begin{figure}[htbp]
\centering
\capstart

\noindent\sphinxincludegraphics{{second-freq-as-bg-illustration}.png}
\caption{The two matter perturbations}\label{\detokenize{matter-stimulated/interference:interference-adding-new-destroy-resonance-q-value}}\label{\detokenize{matter-stimulated/interference:id1}}\end{figure}

To quantify this idea, we calculate the Q values for different modes with and without the new frequency. The procedure should be
\begin{enumerate}
\item {} 
Prepare the parameters: \(\omega_v\), \(\lambda_0\) (background matter profile), \(\lambda_1\) (perturbation amplitude for first perturbation), \(k_1\) (perturbation wavenumber for first perturbation)

\item {} 
Calculate the Q values.

\item {} 
Add in \(\lambda_2\) (perturbation amplitude for the second perturbation) and treat this new perturbation as a constant within one wavelength of the first perturbation. Just use some random phase for the new matter profile, i.e., set \(\sin(k_2 x)\) to some resonanable value.

\item {} 
Recalculate the Q values.

\end{enumerate}

Without the new perturbation (Mathematica Code):

\begin{sphinxVerbatim}[commandchars=\\\{\}]
\PYG{n}{deltamsquare13} \PYG{o}{=} \PYG{l+m+mf}{2.6}\PYG{o}{*}\PYG{l+m+mi}{10}\PYG{o}{\PYGZca{}}\PYG{p}{(}\PYG{o}{\PYGZhy{}}\PYG{l+m+mi}{15}\PYG{p}{)}\PYG{p}{;}\PYG{p}{(}\PYG{o}{*}\PYG{n}{MeV}\PYG{o}{\PYGZca{}}\PYG{l+m+mi}{2}\PYG{o}{*}\PYG{p}{)}
\PYG{n}{energy20} \PYG{o}{=} \PYG{l+m+mi}{20}\PYG{p}{;}\PYG{p}{(}\PYG{o}{*}\PYG{n}{Energy} \PYG{o+ow}{in} \PYG{n}{units} \PYG{n}{of} \PYG{n}{MeV}\PYG{o}{*}\PYG{p}{)}
\PYG{n}{thetaV} \PYG{o}{=} \PYG{n}{ArcSin}\PYG{p}{[}\PYG{n}{Sqrt}\PYG{p}{[}\PYG{l+m+mf}{0.093}\PYG{p}{]}\PYG{p}{]}\PYG{o}{/}\PYG{l+m+mi}{2}
\PYG{n}{omegaVKK} \PYG{o}{=} \PYG{n}{OmegaVacuum}\PYG{p}{[}\PYG{n}{energy20}\PYG{p}{,} \PYG{n}{deltamsquare13}\PYG{p}{]}\PYG{p}{(}\PYG{o}{*}\PYG{n}{MeV}\PYG{o}{*}\PYG{p}{)}\PYG{p}{(}\PYG{o}{*}\PYG{n}{deltamsquare13}\PYG{o}{/}\PYG{p}{(}\PYG{l+m+mi}{2} \PYG{n}{energy20}\PYG{p}{)}\PYG{p}{(}\PYG{o}{*}\PYG{n}{MeV}\PYG{o}{*}\PYG{p}{)}\PYG{o}{*}\PYG{p}{)}

\PYG{n}{lambdaNKK} \PYG{o}{=} \PYG{l+m+mf}{0.5}\PYG{o}{*}\PYG{n}{Cos}\PYG{p}{[}\PYG{l+m+mi}{2} \PYG{n}{thetaV}\PYG{p}{]} \PYG{n}{omegaVKK} \PYG{p}{(}\PYG{o}{*}\PYG{n}{MeV}\PYG{o}{*}\PYG{p}{)}

\PYG{n}{onekListKK} \PYG{o}{=} \PYG{p}{\PYGZob{}}\PYG{l+m+mi}{1}\PYG{p}{\PYGZcb{}}\PYG{p}{;}
\PYG{n}{oneaListKK} \PYG{o}{=}\PYG{p}{(}\PYG{o}{*}\PYG{p}{\PYGZob{}}\PYG{l+m+mf}{0.1}\PYG{p}{\PYGZcb{}}\PYG{p}{;}\PYG{o}{*}\PYG{p}{)}\PYG{p}{\PYGZob{}}\PYG{l+m+mf}{0.02} \PYG{p}{(}\PYG{n}{MeVInverse2km}\PYG{p}{[} \PYG{l+m+mi}{2} \PYG{n}{Pi}\PYG{o}{/}\PYG{p}{(}\PYG{n}{omegaMKK} \PYG{n}{onekListKK}\PYG{p}{[}\PYG{p}{[}\PYG{l+m+mi}{1}\PYG{p}{]}\PYG{p}{]}\PYG{p}{)}\PYG{p}{]}\PYG{o}{/}\PYG{l+m+mi}{1500}\PYG{p}{)}\PYG{o}{\PYGZca{}}\PYG{p}{(}\PYG{l+m+mi}{5}\PYG{o}{/}\PYG{l+m+mi}{3}\PYG{p}{)}\PYG{p}{\PYGZcb{}}\PYG{p}{;}
\PYG{n}{onephiListKK} \PYG{o}{=} \PYG{p}{\PYGZob{}}\PYG{l+m+mi}{0}\PYG{p}{\PYGZcb{}}\PYG{p}{;}

\PYG{n}{Part}\PYG{p}{[}\PYG{n}{qValueOrderdList}\PYG{p}{[}\PYG{n}{listNGenerator}\PYG{p}{[}\PYG{l+m+mi}{1}\PYG{p}{,} \PYG{l+m+mi}{10}\PYG{p}{]}\PYG{p}{,} \PYG{n}{onekListKK}\PYG{p}{,} \PYG{n}{oneaListKK}\PYG{p}{,} \PYG{n}{onephiListKK}\PYG{p}{,} \PYG{n}{thetamV}\PYG{p}{]}\PYG{p}{,} \PYG{l+m+mi}{1} \PYG{p}{;}\PYG{p}{;} \PYG{l+m+mi}{10}\PYG{p}{]}\PYG{p}{;}
\PYG{n}{Grid}\PYG{o}{@}\PYG{o}{\PYGZpc{}}
\end{sphinxVerbatim}

which returns the Q value results for each modes:

\begin{sphinxVerbatim}[commandchars=\\\{\}]
\PYG{p}{\PYGZob{}}\PYG{l+m+mi}{1}\PYG{p}{\PYGZcb{}}  \PYG{l+m+mf}{0.}
\PYG{p}{\PYGZob{}}\PYG{o}{\PYGZhy{}}\PYG{l+m+mi}{1}\PYG{p}{\PYGZcb{}} \PYG{l+m+mf}{577810.}
\PYG{p}{\PYGZob{}}\PYG{l+m+mi}{2}\PYG{p}{\PYGZcb{}}  \PYG{l+m+mf}{1.75394}\PYG{o}{*}\PYG{l+m+mi}{10}\PYG{o}{\PYGZca{}}\PYG{l+m+mi}{10}
\PYG{p}{\PYGZob{}}\PYG{o}{\PYGZhy{}}\PYG{l+m+mi}{2}\PYG{p}{\PYGZcb{}} \PYG{l+m+mf}{5.26182}\PYG{o}{*}\PYG{l+m+mi}{10}\PYG{o}{\PYGZca{}}\PYG{l+m+mi}{10}
\PYG{p}{\PYGZob{}}\PYG{l+m+mi}{3}\PYG{p}{\PYGZcb{}}  \PYG{l+m+mf}{4.25927}\PYG{o}{*}\PYG{l+m+mi}{10}\PYG{o}{\PYGZca{}}\PYG{l+m+mi}{15}
\PYG{p}{\PYGZob{}}\PYG{o}{\PYGZhy{}}\PYG{l+m+mi}{3}\PYG{p}{\PYGZcb{}} \PYG{l+m+mf}{8.51854}\PYG{o}{*}\PYG{l+m+mi}{10}\PYG{o}{\PYGZca{}}\PYG{l+m+mi}{15}
\PYG{p}{\PYGZob{}}\PYG{l+m+mi}{4}\PYG{p}{\PYGZcb{}}  \PYG{l+m+mf}{1.16361}\PYG{o}{*}\PYG{l+m+mi}{10}\PYG{o}{\PYGZca{}}\PYG{l+m+mi}{21}
\PYG{p}{\PYGZob{}}\PYG{o}{\PYGZhy{}}\PYG{l+m+mi}{4}\PYG{p}{\PYGZcb{}} \PYG{l+m+mf}{1.93935}\PYG{o}{*}\PYG{l+m+mi}{10}\PYG{o}{\PYGZca{}}\PYG{l+m+mi}{21}
\PYG{p}{\PYGZob{}}\PYG{l+m+mi}{5}\PYG{p}{\PYGZcb{}}  \PYG{l+m+mf}{3.76761}\PYG{o}{*}\PYG{l+m+mi}{10}\PYG{o}{\PYGZca{}}\PYG{l+m+mi}{26}
\PYG{p}{\PYGZob{}}\PYG{o}{\PYGZhy{}}\PYG{l+m+mi}{5}\PYG{p}{\PYGZcb{}} \PYG{l+m+mf}{5.65142}\PYG{o}{*}\PYG{l+m+mi}{10}\PYG{o}{\PYGZca{}}\PYG{l+m+mi}{26}
\end{sphinxVerbatim}

Adding in the new frequency:

\begin{sphinxVerbatim}[commandchars=\\\{\}]
\PYG{n}{phi2} \PYG{o}{=} \PYG{n}{Pi}\PYG{o}{/}\PYG{l+m+mi}{2}\PYG{o}{/}\PYG{l+m+mi}{100}\PYG{p}{;}

\PYG{n}{twoaListKK} \PYG{o}{=}\PYG{p}{(}\PYG{o}{*}\PYG{p}{\PYGZob{}}\PYG{l+m+mf}{0.1}\PYG{p}{,}\PYG{l+m+mf}{0.1}\PYG{p}{\PYGZcb{}}\PYG{p}{;}\PYG{o}{*}\PYG{p}{)}\PYG{p}{\PYGZob{}}\PYG{l+m+mf}{0.02} \PYG{p}{(}\PYG{n}{MeVInverse2km}\PYG{p}{[}\PYG{l+m+mi}{2} \PYG{n}{Pi}\PYG{o}{/}\PYG{p}{(}\PYG{n}{omegaMKK}\PYG{o}{*}\PYG{n}{twokListKK}\PYG{p}{[}\PYG{p}{[}\PYG{l+m+mi}{1}\PYG{p}{]}\PYG{p}{]}\PYG{p}{)}\PYG{p}{]}\PYG{o}{/}\PYG{l+m+mi}{1500}\PYG{p}{)}\PYG{o}{\PYGZca{}}\PYG{p}{(}\PYG{l+m+mi}{5}\PYG{o}{/}\PYG{l+m+mi}{3}\PYG{p}{)}\PYG{p}{,}
\PYG{l+m+mf}{0.02} \PYG{p}{(}\PYG{n}{MeVInverse2km}\PYG{p}{[}\PYG{l+m+mi}{2} \PYG{n}{Pi}\PYG{o}{/}\PYG{p}{(}\PYG{n}{omegaMKK} \PYG{n}{twokListKK}\PYG{p}{[}\PYG{p}{[}\PYG{l+m+mi}{2}\PYG{p}{]}\PYG{p}{]}\PYG{p}{)}\PYG{p}{]}\PYG{o}{/}\PYG{l+m+mi}{1500}\PYG{p}{)}\PYG{o}{\PYGZca{}}\PYG{p}{(}\PYG{l+m+mi}{5}\PYG{o}{/}\PYG{l+m+mi}{3}\PYG{p}{)}\PYG{p}{\PYGZcb{}}\PYG{p}{;}

\PYG{n}{lambdaNKK2} \PYG{o}{=} \PYG{l+m+mf}{0.5}\PYG{o}{*}\PYG{n}{Cos}\PYG{p}{[}\PYG{l+m+mi}{2} \PYG{n}{thetaV}\PYG{p}{]} \PYG{n}{omegaVKK} \PYG{o}{+} \PYG{n}{Sin}\PYG{p}{[}\PYG{n}{phi2}\PYG{p}{]}\PYG{o}{*}\PYG{n}{twoaListKK}\PYG{p}{[}\PYG{p}{[}\PYG{l+m+mi}{2}\PYG{p}{]}\PYG{p}{]}\PYG{o}{*}\PYG{n}{omegaMKK}\PYG{p}{(}\PYG{o}{*}\PYG{n}{MeV}\PYG{o}{*}\PYG{p}{)}

\PYG{n}{omegaMKK2} \PYG{o}{=} \PYG{n}{OmegaMatter2}\PYG{p}{[}\PYG{n}{lambdaNKK2}\PYG{p}{,} \PYG{n}{thetaV}\PYG{p}{,} \PYG{n}{omegaVKK}\PYG{p}{]}\PYG{p}{(}\PYG{o}{*}\PYG{n}{MeV}\PYG{o}{*}\PYG{p}{)}
\PYG{n}{thetamV2} \PYG{o}{=} \PYG{n}{Mod}\PYG{p}{[}\PYG{n}{ArcTan}\PYG{p}{[}\PYG{n}{Sin}\PYG{p}{[}\PYG{l+m+mi}{2} \PYG{n}{thetaV}\PYG{p}{]}\PYG{o}{/}\PYG{p}{(}\PYG{n}{Cos}\PYG{p}{[}\PYG{l+m+mi}{2} \PYG{n}{thetaV}\PYG{p}{]} \PYG{o}{\PYGZhy{}} \PYG{p}{(}\PYG{n}{lambdaNKK2}\PYG{o}{/}\PYG{n}{omegaVKK}\PYG{p}{)}\PYG{o}{\PYGZca{}}\PYG{l+m+mi}{2}\PYG{p}{)}\PYG{p}{]}\PYG{o}{/}\PYG{l+m+mi}{2}\PYG{p}{,} \PYG{n}{Pi}\PYG{p}{]}

\PYG{n}{onekListKK2} \PYG{o}{=} \PYG{p}{\PYGZob{}}\PYG{l+m+mi}{1}\PYG{p}{\PYGZcb{}}\PYG{o}{*}\PYG{n}{omegaMKK}\PYG{o}{/}\PYG{n}{omegaMKK2}\PYG{p}{;}
\PYG{n}{oneaListKK2} \PYG{o}{=}\PYG{p}{(}\PYG{o}{*}\PYG{p}{\PYGZob{}}\PYG{l+m+mf}{0.1}\PYG{p}{\PYGZcb{}}\PYG{p}{;}\PYG{o}{*}\PYG{p}{)}\PYG{p}{\PYGZob{}}\PYG{l+m+mf}{0.02} \PYG{p}{(}\PYG{n}{MeVInverse2km}\PYG{p}{[}\PYG{l+m+mi}{2} \PYG{n}{Pi}\PYG{o}{/}\PYG{p}{(}\PYG{n}{omegaMKK2} \PYG{n}{onekListKK2}\PYG{p}{[}\PYG{p}{[}\PYG{l+m+mi}{1}\PYG{p}{]}\PYG{p}{]}\PYG{p}{)}\PYG{p}{]}\PYG{o}{/}\PYG{l+m+mi}{1500}\PYG{p}{)}\PYG{o}{\PYGZca{}}\PYG{p}{(}\PYG{l+m+mi}{5}\PYG{o}{/}\PYG{l+m+mi}{3}\PYG{p}{)}\PYG{p}{\PYGZcb{}}\PYG{p}{;}

\PYG{n}{Part}\PYG{p}{[}\PYG{n}{qValueOrderdList}\PYG{p}{[}\PYG{n}{listNGenerator}\PYG{p}{[}\PYG{l+m+mi}{1}\PYG{p}{,} \PYG{l+m+mi}{10}\PYG{p}{]}\PYG{p}{,} \PYG{n}{onekListKK2}\PYG{p}{,} \PYG{n}{oneaListKK2}\PYG{p}{,} \PYG{n}{onephiListKK}\PYG{p}{,} \PYG{n}{thetamV2}\PYG{p}{]}\PYG{p}{,} \PYG{l+m+mi}{1} \PYG{p}{;}\PYG{p}{;} \PYG{l+m+mi}{10}\PYG{p}{]}
\PYG{n}{Grid}\PYG{o}{@}\PYG{o}{\PYGZpc{}}
\end{sphinxVerbatim}

and the results for Q values of different modes are:

\begin{sphinxVerbatim}[commandchars=\\\{\}]
\PYG{p}{\PYGZob{}}\PYG{l+m+mi}{1}\PYG{p}{\PYGZcb{}}  \PYG{l+m+mf}{246.833}
\PYG{p}{\PYGZob{}}\PYG{o}{\PYGZhy{}}\PYG{l+m+mi}{1}\PYG{p}{\PYGZcb{}} \PYG{l+m+mf}{577687.}
\PYG{p}{\PYGZob{}}\PYG{l+m+mi}{2}\PYG{p}{\PYGZcb{}}  \PYG{l+m+mf}{1.75752}\PYG{o}{*}\PYG{l+m+mi}{10}\PYG{o}{\PYGZca{}}\PYG{l+m+mi}{10}
\PYG{p}{\PYGZob{}}\PYG{o}{\PYGZhy{}}\PYG{l+m+mi}{2}\PYG{p}{\PYGZcb{}} \PYG{l+m+mf}{5.26655}\PYG{o}{*}\PYG{l+m+mi}{10}\PYG{o}{\PYGZca{}}\PYG{l+m+mi}{10}
\PYG{p}{\PYGZob{}}\PYG{l+m+mi}{3}\PYG{p}{\PYGZcb{}}  \PYG{l+m+mf}{4.27026}\PYG{o}{*}\PYG{l+m+mi}{10}\PYG{o}{\PYGZca{}}\PYG{l+m+mi}{15}
\PYG{p}{\PYGZob{}}\PYG{o}{\PYGZhy{}}\PYG{l+m+mi}{3}\PYG{p}{\PYGZcb{}} \PYG{l+m+mf}{8.53505}\PYG{o}{*}\PYG{l+m+mi}{10}\PYG{o}{\PYGZca{}}\PYG{l+m+mi}{15}
\PYG{p}{\PYGZob{}}\PYG{l+m+mi}{4}\PYG{p}{\PYGZcb{}}  \PYG{l+m+mf}{1.16758}\PYG{o}{*}\PYG{l+m+mi}{10}\PYG{o}{\PYGZca{}}\PYG{l+m+mi}{21}
\PYG{p}{\PYGZob{}}\PYG{o}{\PYGZhy{}}\PYG{l+m+mi}{4}\PYG{p}{\PYGZcb{}} \PYG{l+m+mf}{1.94507}\PYG{o}{*}\PYG{l+m+mi}{10}\PYG{o}{\PYGZca{}}\PYG{l+m+mi}{21}
\PYG{p}{\PYGZob{}}\PYG{l+m+mi}{5}\PYG{p}{\PYGZcb{}}  \PYG{l+m+mf}{3.78384}\PYG{o}{*}\PYG{l+m+mi}{10}\PYG{o}{\PYGZca{}}\PYG{l+m+mi}{26}
\PYG{p}{\PYGZob{}}\PYG{o}{\PYGZhy{}}\PYG{l+m+mi}{5}\PYG{p}{\PYGZcb{}} \PYG{l+m+mf}{5.67375}\PYG{o}{*}\PYG{l+m+mi}{10}\PYG{o}{\PYGZca{}}\PYG{l+m+mi}{26}
\end{sphinxVerbatim}

The corresponding plots are shown in \hyperref[\detokenize{matter-stimulated/interference:second-freq-as-bg-example-1-first-freq-only}]{Fig.\@ \ref{\detokenize{matter-stimulated/interference:second-freq-as-bg-example-1-first-freq-only}}} and \hyperref[\detokenize{matter-stimulated/interference:second-freq-as-bg-example-1-with-second-freq-as-bg}]{Fig.\@ \ref{\detokenize{matter-stimulated/interference:second-freq-as-bg-example-1-with-second-freq-as-bg}}}
\begin{figure}[htbp]
\centering
\capstart

\noindent\sphinxincludegraphics{{second-freq-as-bg-example-1-first-freq-only}.png}
\caption{Only the first frequency which is at resonance}\label{\detokenize{matter-stimulated/interference:second-freq-as-bg-example-1-first-freq-only}}\label{\detokenize{matter-stimulated/interference:id2}}\end{figure}
\begin{figure}[htbp]
\centering
\capstart

\noindent\sphinxincludegraphics{{second-freq-as-bg-example-1-with-second-freq-as-bg}.png}
\caption{With the second frequency added in but only as a shift in background density}\label{\detokenize{matter-stimulated/interference:second-freq-as-bg-example-1-with-second-freq-as-bg}}\label{\detokenize{matter-stimulated/interference:id3}}\end{figure}

What determines the amplitude is
\begin{equation*}
\begin{split}\frac{ \left\lvert  B_{n}  \right\rvert^2 }{ \left\lvert    B_{n}  \right\rvert^2 + ( n  k - \omega_m )^2  }.\end{split}
\end{equation*}
In this example, the coefficient in unit of \(\omega_m\) for one perturbation only is
\begin{equation*}
\begin{split}\lvert B_1 \rvert = 3.46135\times 10^{-6},\end{split}
\end{equation*}
while it shifts a little bit when the second frequency is added in as a background shift, in unit of \(\omega_m\),
\begin{equation*}
\begin{split}\lvert B_1' \rvert = 3.46356\times 10^{-6}.\end{split}
\end{equation*}
We set the first frequency at resonance, which means
\begin{equation*}
\begin{split}k - \omega_m = 0.\end{split}
\end{equation*}
With the apprearance of the second frequency, what we have now is, in unit of \(\omega_m\),
\begin{equation*}
\begin{split}k - \omega_m' = 0.000854192,\end{split}
\end{equation*}
which is far beyond the width of the resonance.


\subsubsection{Some Artifical Systems}
\label{\detokenize{matter-stimulated/interference:some-artifical-systems}}\begin{figure}[htbp]
\centering
\capstart

\noindent\sphinxincludegraphics{{second-freq-as-bg-1-divided-10-10-20-30-100-1000-10000-matter-density}.png}
\caption{Matter profiles with one wavelength of the first perturbation}\label{\detokenize{matter-stimulated/interference:id4}}\end{figure}

The resonance from the first perturbation is destroyed as the second perturbation grows much larger than it.
\begin{figure}[htbp]
\centering
\capstart

\noindent\sphinxincludegraphics{{second-freq-as-bg-1-divided-10-10-20-30-100-1000-10000-full-numerical}.png}
\caption{Full numerical solutions}\label{\detokenize{matter-stimulated/interference:id5}}\end{figure}

The hint is, the shift of the background matter profile is related to the destruction effect, whether it's true destruction or effective destruction (destruction within a region).

An investigation of the most important mode shows that it is destroyed due to the a shift of background matter density.
\begin{figure}[htbp]
\centering
\capstart

\noindent\sphinxincludegraphics{{second-freq-as-bg-1-divided-10-10-20-30-100-1000-10000-log}.png}
\caption{Resonance destruction of the first mode}\label{\detokenize{matter-stimulated/interference:id6}}\end{figure}

To verify how the interference actually works, we plot the full numerical calculation and the \{1,0\} mode, for comparision
\begin{figure}[htbp]
\centering
\capstart

\noindent\sphinxincludegraphics{{second-freq-as-bg-1-d-10-1-d-1000-1-d-10000-1-d-200000-1000-full-numerical-with-10-mode-with-gridlines}.png}
\caption{The solid lines are the full numerical calculations of the system; Dashed lines are \{1,0\} mode of the corresponding parameters; Vertical grid lines are the positions of zero amplitudes of the second matter perturbation.}\label{\detokenize{matter-stimulated/interference:second-freq-as-bg-1-d-10-1-d-1000-1-d-10000-1-d-200000-1000-full-numerical-with-10-mode-with-gridlines}}\label{\detokenize{matter-stimulated/interference:id7}}\end{figure}

It is clearly shown in \hyperref[\detokenize{matter-stimulated/interference:second-freq-as-bg-1-d-10-1-d-1000-1-d-10000-1-d-200000-1000-full-numerical-with-10-mode-with-gridlines}]{Fig.\@ \ref{\detokenize{matter-stimulated/interference:second-freq-as-bg-1-d-10-1-d-1000-1-d-10000-1-d-200000-1000-full-numerical-with-10-mode-with-gridlines}}} that each vicinity of zero amplitude for the second perturbation, the transition amplitude will increase, due to the resonance of the first perturbation.

What's more interesting is that as \(A_2\) becomes much larger than \(A_1\), the resonance seams to be destroyed. As an example, here we take \(A_2=0.0346135, A_1=0.0000357347\) both in unit of \(\omega_m\), which means \(A_2\) is almost three orders larger than \(A_1\). The results are shown in \hyperref[\detokenize{matter-stimulated/interference:second-freq-as-bg-1-d-10-1-d-1000-1-d-10000-1-d-200000-10000-full-numerical-with-10-mode-with-gridlines}]{Fig.\@ \ref{\detokenize{matter-stimulated/interference:second-freq-as-bg-1-d-10-1-d-1000-1-d-10000-1-d-200000-10000-full-numerical-with-10-mode-with-gridlines}}}.
\begin{figure}[htbp]
\centering
\capstart

\noindent\sphinxincludegraphics{{second-freq-as-bg-1-d-10-1-d-1000-1-d-10000-1-d-200000-10000-full-numerical-with-10-mode-with-gridlines}.png}
\caption{The solid lines are the full numerical calculations of the system; Dashed lines are \{1,0\} mode of the corresponding parameters; Vertical grid lines are the positions of zero amplitudes of the second matter perturbation.}\label{\detokenize{matter-stimulated/interference:second-freq-as-bg-1-d-10-1-d-1000-1-d-10000-1-d-200000-10000-full-numerical-with-10-mode-with-gridlines}}\label{\detokenize{matter-stimulated/interference:id8}}\end{figure}


\subsection{Refs \& Notes}
\label{\detokenize{matter-stimulated/interference:refs-notes}}

\section{Flavor Isospin Method}
\label{\detokenize{matter-stimulated/flavor-isospin-matter::doc}}\label{\detokenize{matter-stimulated/flavor-isospin-matter:flavor-isospin-method}}

\subsection{Flavor Isospin Applied}
\label{\detokenize{matter-stimulated/flavor-isospin-matter:flavor-isospin-applied}}
Using flavor isospin, the equation of motion is written as
\begin{equation*}
\begin{split}\frac{d\vec s}{dx} = \vec{s} \times \vec H,\end{split}
\end{equation*}
where
\begin{equation*}
\begin{split}\vec s = \begin{pmatrix}
\mathrm{Re}(\psi_1^*\psi_2) \\
\mathrm{Im}(\psi_1^*\psi_2) \\
(\lvert \psi_1 \rvert^2 - \lvert \psi_2 \rvert^2)/2.
\end{pmatrix}\end{split}
\end{equation*}
\begin{sphinxadmonition}{note}{Background Matter Basis}

In background matter basis the Hamiltonian vector is
\begin{equation*}
\begin{split}\vec H = \begin{pmatrix}
\delta \lambda(x) \sin 2\theta_m \\
0 \\
\omega_m - \delta \lambda(x) \cos 2\theta_m
\end{pmatrix}.\end{split}
\end{equation*}
For two perturbations, we write it as
\begin{equation*}
\begin{split}\vec H = \begin{pmatrix}
0 \\
0 \\
\omega_m
\end{pmatrix} + \begin{pmatrix}
\delta \lambda_1(x) \sin 2\theta_m \\
0 \\
 - \delta \lambda_1(x) \cos 2\theta_m
\end{pmatrix} + \begin{pmatrix}
\delta \lambda_2(x) \sin 2\theta_m \\
0 \\
 - \delta \lambda_2(x) \cos 2\theta_m
\end{pmatrix}.\end{split}
\end{equation*}
The initial condition is
\begin{equation*}
\begin{split}\Psi(0) = \begin{pmatrix}
1 \\
0
\end{pmatrix},\end{split}
\end{equation*}
which corresponds to a flavor isospin vector
\begin{equation*}
\begin{split}\vec s(0) = \frac{1}{2} \begin{pmatrix}
0 \\
0 \\
1
\end{pmatrix}.\end{split}
\end{equation*}\end{sphinxadmonition}

\begin{sphinxadmonition}{note}{T-basis}

In this basis, the Hamiltonian is
\begin{equation*}
\begin{split}H_1 &= -\frac{\omega_m}{2} \sigma_3 - \frac{\delta \lambda}{2} \sin 2\theta_m \begin{pmatrix}
0 & e^{2i\eta_1(x)} \\
e^{-2i\eta_1(x)} & 0
\end{pmatrix} \\
& = -\frac{\omega_m}{2} \sigma_3 +\frac{\delta \lambda}{2} \sin 2\theta_m \sin 2\eta_1(x) \sigma_2  - \frac{\delta \lambda}{2} \sin 2\theta_m \cos 2\eta_1(x) \sigma_1,\end{split}
\end{equation*}
or
\begin{equation*}
\begin{split}H_2 &= - \frac{\delta \lambda}{2} \sin 2\theta_m \begin{pmatrix}
0 & e^{2i\eta_2(x)} \\
e^{-2i\eta_2(x)} & 0
\end{pmatrix} \\
&= \frac{\delta \lambda}{2} \sin 2\theta_m \sin 2\eta_2(x) \sigma_2  - \frac{\delta \lambda}{2} \sin 2\theta_m \cos 2\eta_2(x) \sigma_1,\end{split}
\end{equation*}
where the background is removed from diagonal elements in \(H_1\) but not in \(H_2\).

The corresponding vectors are
\begin{equation*}
\begin{split}\vec H_1 = \begin{pmatrix}
\delta\lambda \sin 2\theta_m \cos 2\eta_1(x) \\
-\delta\lambda \sin 2\theta_m \sin 2\eta_1(x)\\
\omega_m
\end{pmatrix},\end{split}
\end{equation*}
and
\begin{equation*}
\begin{split}\vec H_2 = \begin{pmatrix}
\delta\lambda \sin 2\theta_m \cos 2\eta_2(x) \\
-\delta\lambda \sin 2\theta_m \sin 2\eta_2(x)\\
0
\end{pmatrix}.\end{split}
\end{equation*}
Given the initial condition in background matter basis
\begin{equation*}
\begin{split}\Psi(0) = \begin{pmatrix}
1 \\
0
\end{pmatrix},\end{split}
\end{equation*}
we have to apply the T transformation to get the initial condition in the T-basis
\begin{equation*}
\begin{split}\Psi_1(0) &=  \begin{pmatrix} e^{i \eta_1 (x)} & 0 \\  0 & e^{-i \eta_1 (x)}  \end{pmatrix}\Psi(0) = \begin{pmatrix} e^{i \eta_1 (x)} \\  0 \end{pmatrix} \\
\Psi_2(0) &=  \begin{pmatrix} e^{i \eta_2 (x)} & 0 \\  0 & e^{-i \eta_2 (x)}  \end{pmatrix}\Psi(0) = \begin{pmatrix} e^{i \eta_2 (x)} \\  0 \end{pmatrix},\end{split}
\end{equation*}
which correspond to flavor isospin vectors
\begin{equation*}
\begin{split}\vec s_1(0) = \vec s_2(0) = \vec s(0) = \frac{1}{2} \begin{pmatrix}
0 \\
0 \\
1
\end{pmatrix},\end{split}
\end{equation*}
since the T transformation is unitary.
\end{sphinxadmonition}

\begin{sphinxadmonition}{note}{Modes}

For each mode of the multi-frequency case, the Hamiltonian is
\begin{equation*}
\begin{split}H = \frac{1}{2}\begin{pmatrix}
0 & B_N e^{i(n_i k_i -\omega_m)x} \\
B_N^* e^{-i(n_i k_i -\omega_m)x} & 0
\end{pmatrix},\end{split}
\end{equation*}
where \(B_N\) is either real or pure imaginary,
\begin{equation*}
\begin{split}B_N &= -(-i)^{\sum_a n_a} \tan 2\theta_m \left( \sum_a n_a k_a \right) \left( \prod_a J_{n_a}\left( \frac{A_a}{k_a}\cos 2\theta_m \right) \right)\\
& = - \tan 2\theta_m \left( \sum_a n_a k_a \right) \left( \prod_a J_{n_a}\left( \frac{A_a}{k_a}\cos 2\theta_m \right) \right) e^{-i \sum_a n_a \pi/2}\\
& = \rho_{N} e^{-i \sum_a n_a \pi/2}.\end{split}
\end{equation*}
The Hamiltonian vector is
\begin{equation*}
\begin{split}\vec H = \begin{pmatrix}
\rho_N \cos\left( (n_i k_i -\omega_m)x - \sum_a n_a \pi/2 \right) \\
-\rho_N \sin\left( (n_i k_i -\omega_m)x - \sum_a n_a \pi/2 \right) \\
0
\end{pmatrix}.\end{split}
\end{equation*}\end{sphinxadmonition}


\subsection{Equilibrium Points, Linear Stability Analysis, and Limit Cycles}
\label{\detokenize{matter-stimulated/flavor-isospin-matter:equilibrium-points-linear-stability-analysis-and-limit-cycles}}
In background matter basis, the equation of motion is
\begin{equation*}
\begin{split}\frac{d}{dx}\begin{pmatrix}
s_1 \\
s_2 \\
s_3
\end{pmatrix} = \begin{pmatrix}
s_1 \\
s_2 \\
s_3
\end{pmatrix}  \times \begin{pmatrix}
\delta \lambda(x) \sin 2\theta_m \\
0 \\
\omega_m - \delta \lambda(x) \cos 2\theta_m
\end{pmatrix}.\end{split}
\end{equation*}
Such a system is still not easy to solve. However, we can use phase portrait to get some information.

The fixed points are obtained by setting \(\vec s\times \vec H = 0 = \frac{d}{dx}\vec s\). Even though in general we need to obtain the fixed points first before infering the linear stability, this is not needed since this equation is linear to \(\vec s\).

The Jacobian is obtained
\begin{equation*}
\begin{split}J_{mn} & = \frac{d (\vec s\times \vec H)_m}{ds_n} \\
& = \begin{pmatrix}
0 & H_3 & -H_2\\
-H_3 &  0 & H_1 \\
H_2 & -H_1 & 0
\end{pmatrix},\end{split}
\end{equation*}
which comes from the result
\begin{equation*}
\begin{split}\vec s\times \vec H = \begin{pmatrix}
s_2 H_3 - s_3 H_2 \\
s_3 H_1 - s_1 H_3 \\
s_1 H_2 - s_2 H_1
\end{pmatrix}.\end{split}
\end{equation*}
Plugin in the Hamiltonian in background matter basis, the eigenvalues of this Jacobian are
\begin{equation*}
\begin{split}& 0 \\
& -\frac{1}{\sqrt{2}} \sqrt{ - ( A_1 \sin (k_1 x) -\omega_m )^2 + 2 A_1 \omega_m \sin (k_1 x) (1 - \cos 2\theta_m) }\\
& \frac{1}{\sqrt{2}} \sqrt{ - ( A_1 \sin (k_1 x) -\omega_m )^2 + 2 A_1 \omega_m \sin (k_1 x) (1 - \cos 2\theta_m) }.\end{split}
\end{equation*}
For \(- ( A_1 \sin (k_1 x) -\omega_m )^2 + 2 A_1 \omega_m \sin (k_1 x) (1 - \cos 2\theta_m) > 0\), the eigenvalues have real parts, which means the system is a saddle point arround such equilibrium points.
\begin{figure}[htbp]
\centering
\capstart

\noindent\sphinxincludegraphics{{fixed-points-eigenvalues}.png}
\caption{Eigenvalues of Jacobian and fixed points. Source: \href{http://people.uleth.ca/~roussel/nld/stability.pdf}{Stability Analysis for ODEs by Marc R. Roussel}}\label{\detokenize{matter-stimulated/flavor-isospin-matter:id1}}\end{figure}


\section{Physics Picture}
\label{\detokenize{matter-stimulated/picture::doc}}\label{\detokenize{matter-stimulated/picture:physics-picture}}

\subsection{Rabi oscillations}
\label{\detokenize{matter-stimulated/picture:rabi-oscillations}}
Hamiltonian of Rabi oscillation is
\begin{equation*}
\begin{split}H = -\frac{\omega_m}{2} \begin{pmatrix} 1 & 0 \\ 0 & -1 \end{pmatrix} - A \cos(k t)\begin{pmatrix} 0 & 1 \\ 1 & 0  \end{pmatrix} .\end{split}
\end{equation*}
The Hamiltonian we could solve is
\begin{equation*}
\begin{split}H &= -\frac{\omega_m}{2} \sigma_3 - \frac{A}{2} \begin{pmatrix}0 & e^{i k x} \\ e^{-i k x} & 0 \end{pmatrix} \\
& =  -\frac{\omega_m}{2} \sigma_3 - \frac{A}{2} \cos(kx) \sigma_1 + \frac{A}{2} \sin (kx) \sigma_2 ,\end{split}
\end{equation*}
which has a transition probability
\begin{equation*}
\begin{split}P(x) = \frac{\lvert A\rvert^2}{ \lvert A\rvert^2 + (k - \omega_m)^2 }  \sin^2 \left( \sqrt{ \lvert A\rvert^2 + (k - \omega_m)^2 } x/2 \right).\end{split}
\end{equation*}
With two perturbations
\begin{equation*}
\begin{split}H' = -\frac{\omega_m}{2} \sigma_3  - \frac{1}{2} (A_1 \cos(k_1 x) + A_2 \cos(k_2 x)) \sigma_1 + \frac{1}{2}( A_1 \sin (k_1x) + A_2 \sin (k_2 x) ) \sigma_2.\end{split}
\end{equation*}
If \(k_1 \gg k_2\), we can approximate by treating the slow rotating perturbation as a constant added to the energy gap, so that the new energy gap is shifted
\begin{equation*}
\begin{split}\omega_m' = \sqrt{ \omega_m^2 + A_2^2 },\end{split}
\end{equation*}
which could possibly shift the system out of resonance.

The best practice would be applying this to the different modes.


\subsection{Oscillations and Modes}
\label{\detokenize{matter-stimulated/picture:oscillations-and-modes}}
Using Jacobi-Anger expansion, for any system with Hamiltonian
\begin{equation*}
\begin{split}H = -\frac{\omega_m}{2} \sigma_3 + \frac{1}{2}\sum_n A_n \sin (k_n x) \cos 2\theta_m \sigma_3 - \frac{1}{2}\sum_n A_n \sin (k_n x) \sin 2\theta_m \sigma_1,\end{split}
\end{equation*}
we could rewrite the system into a composition of multiple Rabi oscillations
\begin{equation*}
\begin{split}H = -\frac{\omega_m}{2} \sigma_3 + \frac{1}{2} \sum_{n_1} \cdots \sum_{n_N} \begin{pmatrix} 0 & B_{n_1,\cdots,n_N} \Phi_{n_1,\cdots, n_N} e^{i \left( \sum_{a} n_a k_a   \right)x} \\ B_{n_1,\cdots,n_N}^* \Phi_{n_1,\cdots, n_N}^* e^{-i \left( \sum_{a} n_a k_a   \right)x} & 0 \end{pmatrix}.\end{split}
\end{equation*}
For each mode, we have a Rabi oscillation
\begin{equation*}
\begin{split}H_{n_1,\cdots,n_N} =  -\frac{\omega_m}{2} \sigma_3 + \frac{1}{2}
\lvert B_{n_1,\cdots,n_N} \rvert \begin{pmatrix}
0 & e^{i \left( \sum_{a} n_a k_a   \right)x} \\
e^{-i \left( \sum_{a} n_a k_a   \right)x} &  0
\end{pmatrix},\end{split}
\end{equation*}
where we have dropped \(\Phi_{n_1,\cdots, n_N}\) and the possible sign and phase of \(B_{n_1,\cdots,n_N}\) since these phase terms only determines the phase of the perturbation on xy plane.

To explain the interference, we explore the superposition of two modes,
\begin{equation*}
\begin{split}H \equiv -\frac{\omega_m}{2} \sigma_3 + \frac{1}{2}
\lvert B_1 \rvert \begin{pmatrix}
0 & e^{i \left( \sum_{a} n_a k_a   \right)x} \\
e^{-i \left( \sum_{a} n_a k_a   \right)x} &  0
\end{pmatrix} + \frac{1}{2}
\lvert B_2 \rvert \begin{pmatrix}
0 & e^{i \left( \sum_{a} n_a' k_a   \right)x} \\
e^{-i \left( \sum_{a} n_a' k_a   \right)x} &  0
\end{pmatrix},\end{split}
\end{equation*}
which is composed of two Rabi oscillations. We choose the first mode to be the one close to resonance, i.e., \(\sum_a n_a k_a \sim \omega_m\), while the second mode is far away from resonance.

For simplicity we use two perturbations, that is \(a=1,2\). The Hamiltonian can be written as
\begin{equation*}
\begin{split}H = -\frac{\omega_m}{2} \sigma_3 + \frac{1}{2} ( \lvert B_1\rvert \cos(\phi_1 x) + \lvert B_1 \rvert \cos (\phi_2 x) )\sigma_1 - \frac{1}{2} ( \lvert B_1 \rvert \sin (\phi_1 x) + \lvert B_2 \rvert \sin(\phi_2x) ) \sigma_2,\end{split}
\end{equation*}
where we define \(\phi_1 = n_1 k_1 + n_2 k_2\) and \(\phi_2 = n_1' k_1 + n_2' k_2\). Using Pauli matrices are basis, this corresponds to a Hamilton vector
\begin{equation*}
\begin{split}\vec H = \begin{pmatrix}
- \lvert B_1\rvert \cos(\phi_1 x) - \lvert B_2 \rvert \cos (\phi_2 x)  \\
\lvert B_1 \rvert \sin (\phi_1 x) + \lvert B_2 \rvert \sin(\phi_2x) \\
\omega_m
\end{pmatrix} = \begin{pmatrix}
0  \\
0 \\
\omega_m
\end{pmatrix} + \begin{pmatrix}
- \lvert B_1\rvert \cos(\phi_1 x)   \\
\lvert B_1 \rvert \sin (\phi_1 x)  \\
0
\end{pmatrix} + \begin{pmatrix}
- \lvert B_2 \rvert \cos (\phi_2 x)  \\
\lvert B_2 \rvert \sin(\phi_2x) \\
0
\end{pmatrix},\end{split}
\end{equation*}
which has a z component and two rotating perturbations. We choose the system to be
\begin{equation*}
\begin{split}\phi_1 &\sim \omega_m \\
\phi_2 & \neq \omega_m.\end{split}
\end{equation*}
We then have two different situations, \(\phi_2/\omega_m \gg 1\) and \(\phi_2/\omega_m \ll 1\).


\subsubsection{Slow Perturbation}
\label{\detokenize{matter-stimulated/picture:slow-perturbation}}
For \(\phi_2/\omega_m \ll 1\), the second mode is a very slow rotating perturbation, which can be explained using the proposed theory.
\begin{figure}[htbp]
\centering
\capstart

\noindent\sphinxincludegraphics{{rabi-oscillation-interference-of-different-modes}.png}
\caption{Resonance interference of modes. The combination of the first mode and \(B_2=\)}\label{\detokenize{matter-stimulated/picture:id1}}\end{figure}
\begin{figure}[htbp]
\centering
\capstart

\noindent\sphinxincludegraphics{{rabi-oscillation-interference-of-different-modes-compare-rabi-formula}.png}
\caption{Compare with Rabi formula}\label{\detokenize{matter-stimulated/picture:id2}}\end{figure}

As a test of the theory, we can calculate the ratio of each \(B_2\), which depends on the modes, and the critical value \(B_2^C\) which is the crtical value for the destruction of the resonance.
\begin{figure}[htbp]
\centering
\capstart

\noindent\sphinxincludegraphics{{rabi-oscillation-interference-of-different-modes-b2-over-b2critical}.png}
\caption{Ratio \(\lvert B_2\rvert/\lvert B_2^C \rvert\)}\label{\detokenize{matter-stimulated/picture:id3}}\end{figure}

To summarize, in the modes view, resonance of some modes are destroyed by some certain modes.


\subsection{Example of Full System}
\label{\detokenize{matter-stimulated/picture:example-of-full-system}}
First we choose a system that is on resonance
\begin{equation*}
\begin{split}H_1 = -\frac{\omega_m}{2} \sigma_3 + \frac{\delta \lambda_1}{2} \cos 2\theta_m \sigma_3 - \frac{\delta \lambda_1}{2} \sin 2\theta_m \sigma_1,\end{split}
\end{equation*}
where \(\delta\lambda_1 = A_1 \sin (k_1 x)\), where \(k_1 = \omega_m\) and \(A_1 = 3.5\times 10^{-5}\omega_m\). This sets the system to resonance.
\begin{figure}[htbp]
\centering
\capstart

\noindent\sphinxincludegraphics{{resonance-freq-example-1}.png}
\caption{Resonance}\label{\detokenize{matter-stimulated/picture:id4}}\end{figure}

\begin{sphinxadmonition}{note}{Does the Diagonal Term Matter?}

Removing the diagonal elements of the perturbation
\begin{equation*}
\begin{split}H_1' = -\frac{\omega_m}{2} \sigma_3  - \frac{\delta \lambda_1}{2} \sin 2\theta_m \sigma_1,\end{split}
\end{equation*}
will result in \hyperref[\detokenize{matter-stimulated/picture:resonance-freq-example-1-compare-with-diagonal-elements-of-perturbation-removed}]{Fig.\@ \ref{\detokenize{matter-stimulated/picture:resonance-freq-example-1-compare-with-diagonal-elements-of-perturbation-removed}}}.
\begin{figure}[htbp]
\centering
\capstart

\noindent\sphinxincludegraphics{{resonance-freq-example-1-compare-with-diagonal-elements-of-perturbation-removed}.png}
\caption{Remove the diagonal elements of the preturbation}\label{\detokenize{matter-stimulated/picture:resonance-freq-example-1-compare-with-diagonal-elements-of-perturbation-removed}}\label{\detokenize{matter-stimulated/picture:id5}}\end{figure}
\end{sphinxadmonition}

\begin{sphinxadmonition}{note}{Adding in Slowly Changing Field}

Add a new slow perturbation
\begin{equation*}
\begin{split}\delta \lambda_2 = A_2 \sin (k_2 x),\end{split}
\end{equation*}
with
\begin{equation*}
\begin{split}A_2 &= 10^{-2},\\
k_2 &= 0.1.\end{split}
\end{equation*}\begin{figure}[htbp]
\centering
\capstart

\noindent\sphinxincludegraphics{{resonance-freq-example-1-added-new-slow-perturbation}.png}
\caption{Added new slow perturbation}\label{\detokenize{matter-stimulated/picture:id6}}\end{figure}
\end{sphinxadmonition}

\begin{sphinxadmonition}{note}{Removing Diagonal Elements of Slow Perturbation}

Removing the diagonal elements of slow perturbation
\begin{equation*}
\begin{split}H_2' = -\frac{\omega_m}{2} \sigma_3 + \frac{\delta \lambda_1 }{2} \cos 2\theta_m \sigma_3 - \frac{\delta \lambda_1 + \delta \lambda_2}{2} \sin 2\theta_m \sigma_1,\end{split}
\end{equation*}
gives us the result \hyperref[\detokenize{matter-stimulated/picture:resonance-freq-example-1-added-new-slow-perturbation-compare-with-removing-diagonal-elements}]{Fig.\@ \ref{\detokenize{matter-stimulated/picture:resonance-freq-example-1-added-new-slow-perturbation-compare-with-removing-diagonal-elements}}}.
\begin{figure}[htbp]
\centering
\capstart

\noindent\sphinxincludegraphics{{resonance-freq-example-1-added-new-slow-perturbation-compare-with-removing-diagonal-elements}.png}
\caption{Remove diagonal elements of slow perturbation.}\label{\detokenize{matter-stimulated/picture:resonance-freq-example-1-added-new-slow-perturbation-compare-with-removing-diagonal-elements}}\label{\detokenize{matter-stimulated/picture:id7}}\end{figure}
\end{sphinxadmonition}


\subsection{Explaination}
\label{\detokenize{matter-stimulated/picture:explaination}}
Slow perturbation is slow and changes the energy gap of the system. Since the energy gap \(\omega_m\) determines the resonance point, which is
\begin{equation*}
\begin{split}k_1 = \omega_m,\end{split}
\end{equation*}
adding the slow perturbation could increase \(\omega_m\),
\phantomsection\label{\detokenize{matter-stimulated/picture:equation-quadratic-approximation-energy-gap-shift}}\begin{equation}\label{equation:matter-stimulated/picture:quadratic-approximation-energy-gap-shift}
\begin{split}\omega_m' &= \sqrt{A_{2,\bot} ^2 + \omega_m^2} \\
& = \omega_m \sqrt{ \left(\frac{A_{2,\bot}}{\omega_m} \right)^2 + 1 } \\
& \approx  \omega_m + \frac{A_{2,\bot}^2}{2\omega_m},\end{split}
\end{equation}
where \(A_{2,\bot}\) is component perpendicular to z axis.

\begin{sphinxadmonition}{note}{Only Perpendicular Component}

In the calculation of the modified energy gap, we used only the perpendicular component of the new slow perturbation. This only holds for \(A_{2,\bot}  \ll \omega_m\).

\sphinxstylestrong{PROOF}
\end{sphinxadmonition}

\begin{sphinxadmonition}{note}{Shift The System Out of Resonance}

Shift the system out of resonance, it is required that
\begin{equation*}
\begin{split}\lvert \omega_m' - k_1 \rvert \gtrsim \text{width of resonance} A_1.\end{split}
\end{equation*}
Width of resonance is basically determined by \(A_{1,\bot}\). Apply equation \eqref{equation:matter-stimulated/picture:quadratic-approximation-energy-gap-shift}, we can solve the condition to break the resonance,
\begin{equation*}
\begin{split}A_{2,\bot} \gtrsim \sqrt{2\omega_m A_{1,\bot}}.\end{split}
\end{equation*}
In our example, the condition becomes
\begin{equation*}
\begin{split}&A_2 \sin 2\theta_m \gtrsim \sqrt{2\omega_m A_1 \sin 2\theta_m} \\
\Rightarrow & A_2  \gtrsim \sqrt{2\omega_m A_1 \tan 2\theta_m/\cos 2\theta_m}.\end{split}
\end{equation*}\begin{figure}[htbp]
\centering
\capstart

\noindent\sphinxincludegraphics{{resonance-freq-example-1-added-new-slow-perturbation-destruction}.png}
\caption{With \(A_2=\sqrt{2 A_1 \sin (2 \theta_m)}/ \cos ^2(2 \theta_m) =0.0190304\omega_m\)}\label{\detokenize{matter-stimulated/picture:id8}}\end{figure}
\begin{figure}[htbp]
\centering
\capstart

\noindent\sphinxincludegraphics{{resonance-freq-example-1-added-new-slow-perturbation-destruction-compare}.png}
\caption{Compare to show destruction}\label{\detokenize{matter-stimulated/picture:id9}}\end{figure}

Using Rabi formula the amplitudes are not matching the numerical calculations, \hyperref[\detokenize{matter-stimulated/picture:resonance-freq-example-1-added-new-slow-perturbation-destruction-compare-rotating-field}]{Fig.\@ \ref{\detokenize{matter-stimulated/picture:resonance-freq-example-1-added-new-slow-perturbation-destruction-compare-rotating-field}}}.
\begin{figure}[htbp]
\centering
\capstart

\noindent\sphinxincludegraphics{{resonance-freq-example-1-added-new-slow-perturbation-destruction-compare-gridlines}.png}
\caption{Grid lines are the amplitudes predicted by Rabi formula.}\label{\detokenize{matter-stimulated/picture:resonance-freq-example-1-added-new-slow-perturbation-destruction-compare-rotating-field}}\label{\detokenize{matter-stimulated/picture:id10}}\end{figure}

As a reference, the Q values for each line are
\begin{equation*}
\begin{split}Q_1 & =  \frac{\lvert k_1 - \sqrt{A_2 \sin^2(2\theta_m)  + 1 }  }{A_1\sin (2\theta_m)} = 1.11689, \\
Q_2 & = \frac{\lvert k_1 - \sqrt{A_2' \sin^2(2\theta_m)  + 1 }  }{A_1\sin (2\theta_m)} = 4.04469, \\
Q_3 & = \frac{\lvert k_1 - \sqrt{A_2'' \sin^2(2\theta_m)  + 1 }  }{A_1\sin (2\theta_m)} = 402.277.\end{split}
\end{equation*}\end{sphinxadmonition}

However, the important question is whether the modified oscillation really Rabi oscillation. The answer is NO.
\begin{figure}[htbp]
\centering
\capstart

\noindent\sphinxincludegraphics{{really-rabi-question-mark}.png}
\caption{Is the oscillation with slow perturbation really Rabi oscillation? Upper panel: Theoretical and numerical calculation of original system;
Lower panel: Theoretical and numerical calculation with slow perturbation added.}\label{\detokenize{matter-stimulated/picture:id11}}\end{figure}

We can not predict the oscillation when we add in the new perturbation using the Rabi oscillation formula. That makes sense!


\subsubsection{Introducing Another Component Perturbation}
\label{\detokenize{matter-stimulated/picture:introducing-another-component-perturbation}}
We add in the term that has two components,
\begin{equation*}
\begin{split}H = - \frac{\omega_m}{2} \sigma_3 + \frac{\delta \lambda(x)}{2} \cos 2\theta_m \sigma_3 - \frac{\delta \lambda(x)}{2} \sin 2\theta_m \sigma_1 + \frac{\delta \lambda(x)}{2} \sin 2\theta_m \sigma_2.\end{split}
\end{equation*}\begin{figure}[htbp]
\centering
\capstart

\noindent\sphinxincludegraphics{{resonance-freq-example-1-added-new-slow-perturbation-destruction-compare-rotating-field}.png}
\caption{Add another component xy plane}\label{\detokenize{matter-stimulated/picture:id12}}\end{figure}


\subsubsection{Rotating Perturbation with Constant Strength}
\label{\detokenize{matter-stimulated/picture:rotating-perturbation-with-constant-strength}}
Construct a system with a mode at resonance and another rotating perturbation of constant length,
\begin{equation*}
\begin{split}H = - \frac{\omega_m}{2} \sigma_3 - \frac{1}{2} (A_1\cos(k_1x) + A_2 \sin(k_2x))  \sigma_1 + \frac{1}{2} ( A_1\sin (k_1 x) + A_2 \sin(k_2 x) ) \sigma_2,\end{split}
\end{equation*}
where we choose \(k_1\gg k_2\).

The new \(\sigma_2\) term is a rotating field with constant length, which makes sure the modified energy gap has a constant length rather than the slowly changing energy gap.
\begin{figure}[htbp]
\centering
\capstart

\noindent\sphinxincludegraphics{{rabi-oscillations-energy-gap-change}.png}
\caption{Reduction of transition amplitudes. Black dashed line: one perturbation at exact resonance; Green long dashed line: \(A_2=A_{2,\mathrm{Critical}}=0.0083666\); Blue dotted line: \(A_2=0.01\); Red line: \(A_2=0.02\). The grid lines are the amplitude predicted using Rabi formula correspondingly.}\label{\detokenize{matter-stimulated/picture:id13}}\end{figure}


\subsection{Refs \& Notes}
\label{\detokenize{matter-stimulated/picture:refs-notes}}

\section{Linear Background Matter Profile}
\label{\detokenize{matter-stimulated/linear-background-matter::doc}}\label{\detokenize{matter-stimulated/linear-background-matter:linear-background-matter-profile}}
The equation of motion in flavor basis for arbitary matter profile \(\lambda(x)\) is
\begin{equation*}
\begin{split}i \partial_x \begin{pmatrix}
\psi_e\\
\psi_x
\end{pmatrix} = \left[
\left( \frac{\lambda(x)}{2} - \frac{\omega_v}{2} \cos 2\theta_v  \right) \sigma_3 + \frac{\omega_v}{2} \sin 2\theta_v \sigma_1
\right]\begin{pmatrix}
\psi_e\\
\psi_x
\end{pmatrix}.\end{split}
\end{equation*}
Introduce the T transformation
\begin{equation*}
\begin{split}\begin{pmatrix}
\psi_e\\
\psi_x
\end{pmatrix} = \mathbf{T} \begin{pmatrix}
\psi_{b1}\\
\psi_{b2}
\end{pmatrix} = \begin{pmatrix}
e^{-i \eta(x)} & 0 \\
0 & e^{i \eta(x)}
\end{pmatrix} \begin{pmatrix}
\psi_{b1}\\
\psi_{b2}
\end{pmatrix}.\end{split}
\end{equation*}
By multiplying on both sides \(\mathbf{T}^\dagger\), we get
\begin{equation*}
\begin{split}i \partial_x \begin{pmatrix}
\psi_{b1}\\
\psi_{b2}
\end{pmatrix} = \frac{\omega_v}{2} \sin 2\theta_v \begin{pmatrix}
0 & e^{2i\eta(x)} \\
e^{-2i\eta(x)} & 0
\end{pmatrix}\begin{pmatrix}
\psi_{b1}\\
\psi_{b2}
\end{pmatrix} + \left( \frac{\lambda(x)}{2} - \frac{\omega_v}{2} \cos 2\theta_v - \partial_x \eta(x) \right) \sigma_3 \begin{pmatrix}
\psi_{b1}\\
\psi_{b2}
\end{pmatrix}.\end{split}
\end{equation*}
To remove the diagonal elements we require
\begin{equation*}
\begin{split}\frac{\lambda(x)}{2} - \frac{\omega_v}{2} \cos 2\theta_v - \partial_x \eta(x)  = 0,\end{split}
\end{equation*}\begin{equation*}
\begin{split}\eta(x) = \eta(0) + \frac{1}{2}\int_0^x dx' \left( \lambda(x) - \frac{\omega_v}{2} \cos 2\theta_v \right) .\end{split}
\end{equation*}
Perodic perturbation can be used
\begin{equation*}
\begin{split}\delta \lambda(x) = A\sin (kx + \phi).\end{split}
\end{equation*}
\(\eta(x)\) has the form
\begin{equation*}
\begin{split}\eta(x) &= \eta(0) + \frac{1}{2} \int_0^x dx' (\lambda_0(x) + A\sin(kx+\phi) ) - \frac{1}{2}\omega_v \cos 2\theta_v x \\
& = - \frac{1}{2} \omega_v \cos 2\theta_v x - \frac{A}{2k} \cos(kx+\phi) + \frac{1}{2} g(x) {\color{grey}- \frac{1}{2}g(0) - \frac{A}{2k} \cos \phi + \eta(0) },\end{split}
\end{equation*}
where \(g(x)=\int dx' \lambda_0(x')\) and the grey part is usally set to zero.

The equation we are dealing with is
\begin{equation*}
\begin{split}i \partial_x  \begin{pmatrix}
\psi_{b1}\\
\psi_{b2}
\end{pmatrix} = \frac{\omega_v}{2} \sin 2\theta_v \begin{pmatrix}
0 & e^{2i\eta(x)} \\
e^{-2i\eta(x)} & 0
\end{pmatrix}\begin{pmatrix}
\psi_{b1}\\
\psi_{b2}
\end{pmatrix},\end{split}
\end{equation*}
where
\begin{equation*}
\begin{split}\eta(x) = - \frac{1}{2} \omega_v \cos 2\theta_v x - \frac{A}{2k} \cos(kx+\phi) + \frac{1}{2} g(x) .\end{split}
\end{equation*}
\begin{sphinxadmonition}{note}{Can we solve the equation}

Here is the problem. For most general \(g(x)\) we are not guaranteed to have a analycal solution of the system.

For \(g(x)\propto x\) we have demonstrated that the system is solvale.

What about \(g(x)\propto x^2\) (linear background matter profile) and \(g(x)\propto e^{x}\) (exponential background matter profile)?
\end{sphinxadmonition}


\subsection{Linear Background Matter Profile}
\label{\detokenize{matter-stimulated/linear-background-matter:id1}}
\begin{sphinxadmonition}{note}{Equation Solved}

It turns out that we can solve the system
\begin{equation*}
\begin{split}i\partial_x \begin{pmatrix}
\psi_{b1}\\
\psi_{b2}
\end{pmatrix} = A_s  \begin{pmatrix}
0 & e^{if x^2} \\
e^{-if x^2} & 0
\end{pmatrix}\begin{pmatrix}
\psi_{b1}\\
\psi_{b2}
\end{pmatrix}.\end{split}
\end{equation*}
The solution involves hypergeometric functions. For initial condition
\begin{equation*}
\begin{split}\begin{pmatrix}
\psi_{b1}(0)\\
\psi_{b2}(0)
\end{pmatrix} = \begin{pmatrix}
1\\
0
\end{pmatrix},\end{split}
\end{equation*}
we have the solution
\begin{equation*}
\begin{split}\begin{pmatrix}
\psi_{b1}(x)\\
\psi_{b2}(x)
\end{pmatrix} = \begin{pmatrix}
{}_1F_1 \left( \frac{i A_s^2}{4f};\frac{1}{2}; i f x^2 \right) \\
-i A_s x e^{-i f x^2} {}_1F_1 \left( \frac{i A_s^2}{4f}+1;\frac{3}{2}; i f x^2 \right)
\end{pmatrix},\end{split}
\end{equation*}
where \({}_1F_1(a;b;z)\) is \href{https://en.wikipedia.org/wiki/Confluent\_hypergeometric\_function}{hypergeometric function} .
\end{sphinxadmonition}

We choose the background matter profile to be
\begin{equation*}
\begin{split}\lambda_0(x) = a x + b,\end{split}
\end{equation*}
while the perturbation to be
\begin{equation*}
\begin{split}\delta \lambda(x) = A\sin (kx + \phi).\end{split}
\end{equation*}
\(\eta(x)\) is solved
\begin{equation*}
\begin{split}g(x) & = \frac{1}{2} ax^2+b \\
\eta(x) &= - \frac{1}{2} \omega_v \cos 2\theta_v x - \frac{A}{2k} \cos (kx + \phi) + \frac{1}{2} \left( \frac{1}{2}a x^2 + b x \right),\end{split}
\end{equation*}
so that
\begin{equation*}
\begin{split}e^{2i\eta(x)} = e^{-i \omega_v \cos 2\theta_v x} e^{i ( ax^2/2+ bx )} e^{-i\left( \frac{A}{k} \cos (kx+\phi) \right)}.\end{split}
\end{equation*}
\begin{sphinxadmonition}{note}{Jacobi-Anger Expansion}
\begin{equation*}
\begin{split}e^{-i\left( \frac{A}{k} \cos (kx+\phi) \right)} = \sum_{n=-\infty}^{\infty} (-i)^n J_n\left(\frac{A}{k}\right) e^{in(kx+\phi)}.\end{split}
\end{equation*}\end{sphinxadmonition}

Collect terms
\begin{equation*}
\begin{split}e^{2i\eta(x)} &= \sum_{n=-\infty}^{\infty} (-i)^n J_n\left(\frac{A}{k}\right) e^{in(kx+\phi) + \frac{1}{2}ax^2 +bx - \omega_v \cos 2\theta_v x } \\
&=  \sum_{n=-\infty}^{\infty} (-i)^n J_n\left(\frac{A}{k}\right) e^{ i \left( \frac{1}{2} a \left( x + \frac{b_n}{a} \right)^2 - \frac{b_n^2}{2a} + n\phi \right) } \\
& = \sum_{n=-\infty}^{\infty} J_n\left(\frac{A}{k}\right) e^{ i \left( \frac{1}{2} a \left( x + \frac{b_n}{a} \right)^2 - \frac{b_n^2}{2a} + n\phi + n\pi \right) } \\
& = \sum_{n=-\infty}^{\infty} J_n\left(\frac{A}{k}\right) e^{ i \left( \frac{1}{2} a \left( x + \frac{b_n}{a} \right)^2 - \frac{b_n^2}{2a} + n\phi' \right) }  ,\end{split}
\end{equation*}
where
\begin{equation*}
\begin{split}b_n &= b + nk - \omega_v \cos 2\theta_v,\\
\phi'&= \phi + \pi,\end{split}
\end{equation*}
and we have used
\begin{equation*}
\begin{split}(-i)^n = e^{i n\pi}.\end{split}
\end{equation*}
The equation to be sovled becomes
\begin{equation*}
\begin{split}i\partial_x \begin{pmatrix}
\psi_{b1}\\
\psi_{b2}
\end{pmatrix} =   \sum_{n=-\infty}^\infty \frac{\omega_v}{2}\sin 2\theta_v  J_n\left( \frac{A}{k} \right) \begin{pmatrix}
0 & e^{i\left( \frac{1}{2} a \left( x + \frac{b_n}{a} \right)^2 - \frac{b_n^2}{2a} + n\phi' \right) } \\
e^{-i\left( \frac{1}{2} a \left( x + \frac{b_n}{a} \right)^2 - \frac{b_n^2}{2a} + n\phi' \right)} & 0
\end{pmatrix}\begin{pmatrix}
\psi_{b1}\\
\psi_{b2}
\end{pmatrix}.\end{split}
\end{equation*}
With initial condition
\begin{equation*}
\begin{split}\begin{pmatrix}
\psi_{b1}(0)\\
\psi_{b2}(0)
\end{pmatrix} = \begin{pmatrix}
1\\
0
\end{pmatrix},\end{split}
\end{equation*}
the equation
\begin{equation*}
\begin{split}i\partial_x \begin{pmatrix}
\psi_{b1}\\
\psi_{b2}
\end{pmatrix} =   \frac{\omega_v}{2}\sin 2\theta_v  J_n\left( \frac{A}{k} \right) \begin{pmatrix}
0 & e^{i\left( \frac{1}{2} a \left( x + \frac{b_n}{a} \right)^2 - \frac{b_n^2}{2a} + n\phi' \right) } \\
e^{-i\left( \frac{1}{2} a \left( x + \frac{b_n}{a} \right)^2 - \frac{b_n^2}{2a} + n\phi' \right)} & 0
\end{pmatrix}\begin{pmatrix}
\psi_{b1}\\
\psi_{b2}
\end{pmatrix}.\end{split}
\end{equation*}
can be solved. To save keystroke, we define
\begin{equation*}
\begin{split}A_s= J_n\left(\frac{A}{k}\right)\frac{\omega_v}{2} \sin 2\theta_v.\end{split}
\end{equation*}
We write down the solution
\begin{equation*}
\begin{split}\psi_2(x) =  (-1)^{1/4} \sqrt{2} A_s e^{-i \left( \frac{1}{2} a \left( x + \frac{b_n}{a} \right)^2 - \frac{b_n^2}{2a} + n\phi'  \right) } b_n  \left[ - H_{-1 - i A_s^2/a} (p_1 + p_2 x) F_1 \left( 1 + i \frac{A_s^2}{2a};\frac{3}{2};\frac{i b_n^2}{2a} \right) + (1+\frac{a}{b_n} x) H_{-1 - i A_s^2/a} (p_1)  F_1 \left( 1 + i \frac{A_s^2}{2a};\frac{3}{2}; (p_1 + p_2 x)^2\right) \right] \bigg /  \left\{ \sqrt{a} H_{-i A_s^2/a} (p_1) \left[ (-1)^{3/4} \sqrt{2a} F_1 \left(  i \frac{A_s^2}{2a};\frac{1}{2};\frac{i b_n^2}{2a} \right) - b_n F_1 \left(  1 + i \frac{A_s^2}{2a};\frac{3}{2};\frac{i b_n^2}{2a} \right) \right] \right\},\end{split}
\end{equation*}
where
\begin{equation*}
\begin{split}p_1 & = \frac{(-1)^{1/4} b_n}{\sqrt{2a}}, \\
p_2 & = \frac{ (-1)^{1/4} \sqrt{a}  }{ \sqrt{2} }.\end{split}
\end{equation*}
\(F_1\equiv {}_1 F_1\) is the hypergeometric function. What's important is that we can always define
\begin{equation*}
\begin{split}M_n & =  \sqrt{a} H_{-i A_s^2/a} (p_1) \left[ (-1)^{3/4} \sqrt{2a} F_1 \left(  i \frac{A_s^2}{2a};\frac{1}{2};\frac{i b_n^2}{2a} \right) - b_n F_1 \left(  1 + i \frac{A_s^2}{2a};\frac{3}{2};\frac{i b_n^2}{2a} \right) \right] \\
K_{F,n} &= F_1 \left( 1 + i \frac{A_s^2}{2a};\frac{3}{2};\frac{i b_n^2}{2a} \right) \\
K_{H,n} &= H_{-1 - i A_s^2/a} (p_1),\end{split}
\end{equation*}
which are all constants given the parameters and n. We can reduce the solution to
\begin{equation*}
\begin{split}\psi_2 =  (-1)^{1/4} \sqrt{2} A_s b_n \frac{K_{H,n} F_1 \left( 1 + i \frac{A_s^2}{2a};\frac{3}{2}; (p_1 + p_2 x)^2\right) (1+\frac{a}{b_n} x) - K_{F,n} H_{-1 - i A_s^2/a} (p_1 + p_2 x) }{ M_n } e^{-i \left( \frac{1}{2} a \left( x + \frac{b_n}{a} \right)^2 - \frac{b_n^2}{2a} + n\phi'  \right) }.\end{split}
\end{equation*}
Some points:
\begin{enumerate}
\item {} 
The third argument of \(F_1\), which is denoted as \(p_1 +p_2 x\) is in fact imaginary. As a result, this value of hypergeometric function is actually dropping as \(x\) increase.

\item {} 
In this result, the x dependent Hermite polynomial is a monotonically decreasing function.

\item {} 
The overall phase doesn't play a role in the transition probability.

\end{enumerate}
\begin{enumerate}
\item {} 
Patton, K. M., Kneller, J. P., \& McLaughlin, G. C. (2014). Stimulated neutrino transformation through turbulence. Physical Review D, 89(7), 073022. doi:10.1103/PhysRevD.89.073022

\item {} 
Kneller, J. P., McLaughlin, G. C., \& Patton, K. M. (2013). Stimulated neutrino transformation in supernovae. AIP Conference Proceedings, 1560, 176\textendash{}178. doi:10.1063/1.4826746

\end{enumerate}


\chapter{Collective}
\label{\detokenize{collective/index:collective}}\label{\detokenize{collective/index::doc}}

\section{Fast Modes}
\label{\detokenize{collective/fast-modes::doc}}\label{\detokenize{collective/fast-modes:fast-modes}}
\begin{sphinxadmonition}{note}{Four Beams Model}

There are many things to consider for the four beams line model.
\begin{enumerate}
\item {} 
Different emission angle for neutrinos and antineutrinos;

\item {} 
Different density for neutrinos and antineutrinos;

\item {} 
Left and right difference.

\end{enumerate}
\end{sphinxadmonition}

Is the system unstable even for \(\omega_v=0\) and \(\lambda=0\)?
\begin{figure}[htbp]
\centering
\capstart

\noindent\sphinxincludegraphics{{fast-mode-alpha-0.8}.png}
\caption{The maximum imaginary part in linear stability analysis. These calculations are for fix \(k_m/\hat E=1\), where \(\hat E\) is the energy scale used to scale all the quantities.}{\small \begin{equation*}
\begin{split}\omega_v =& 0\\
\lambda =& 0\\
\alpha = & 0.8.\end{split}
\end{equation*}
The grid of figure are arranged according to the following values of \(\{\theta_1,\theta_2\}\).
\begin{equation*}
\begin{split}\begin{array}{cccc}
\left\{\frac{\pi }{6},\frac{\pi }{6}\right\} & \left\{\frac{\pi }{6},\frac{2 \pi }{9}\right\} & \left\{\frac{\pi }{6},\frac{5 \pi }{18}\right\} & \left\{\frac{\pi }{6},\frac{\pi }{3}\right\} \\
\left\{\frac{2 \pi }{9},\frac{\pi }{6}\right\} & \left\{\frac{2 \pi }{9},\frac{2 \pi }{9}\right\} & \left\{\frac{2 \pi }{9},\frac{5 \pi }{18}\right\} & \left\{\frac{2 \pi }{9},\frac{\pi }{3}\right\} \\
\left\{\frac{5 \pi }{18},\frac{\pi }{6}\right\} & \left\{\frac{5 \pi }{18},\frac{2 \pi }{9}\right\} & \left\{\frac{5 \pi }{18},\frac{5 \pi }{18}\right\} & \left\{\frac{5 \pi }{18},\frac{\pi }{3}\right\} \\
\left\{\frac{\pi }{3},\frac{\pi }{6}\right\} & \left\{\frac{\pi }{3},\frac{2 \pi }{9}\right\} & \left\{\frac{\pi }{3},\frac{5 \pi }{18}\right\} & \left\{\frac{\pi }{3},\frac{\pi }{3}\right\} \\
\end{array}\end{split}
\end{equation*}}\label{\detokenize{collective/fast-modes:id1}}\end{figure}
\begin{figure}[htbp]
\centering
\capstart

\noindent\sphinxincludegraphics{{fast-mode-alpha-1.2}.png}
\caption{The maximum imaginary part in linear stability analysis. These calculations are for fix \(k_m/\hat E=1\), where \(\hat E\) is the energy scale used to scale all the quantities.}{\small \begin{equation*}
\begin{split}\omega_v =& 0\\
\lambda =& 0\\
\alpha = & 1.2.\end{split}
\end{equation*}
The grid of figure are arranged according to the following values of \(\{\theta_1,\theta_2\}\).
\begin{equation*}
\begin{split}\begin{array}{cccc}
\left\{\frac{\pi }{6},\frac{\pi }{6}\right\} & \left\{\frac{\pi }{6},\frac{2 \pi }{9}\right\} & \left\{\frac{\pi }{6},\frac{5 \pi }{18}\right\} & \left\{\frac{\pi }{6},\frac{\pi }{3}\right\} \\
\left\{\frac{2 \pi }{9},\frac{\pi }{6}\right\} & \left\{\frac{2 \pi }{9},\frac{2 \pi }{9}\right\} & \left\{\frac{2 \pi }{9},\frac{5 \pi }{18}\right\} & \left\{\frac{2 \pi }{9},\frac{\pi }{3}\right\} \\
\left\{\frac{5 \pi }{18},\frac{\pi }{6}\right\} & \left\{\frac{5 \pi }{18},\frac{2 \pi }{9}\right\} & \left\{\frac{5 \pi }{18},\frac{5 \pi }{18}\right\} & \left\{\frac{5 \pi }{18},\frac{\pi }{3}\right\} \\
\left\{\frac{\pi }{3},\frac{\pi }{6}\right\} & \left\{\frac{\pi }{3},\frac{2 \pi }{9}\right\} & \left\{\frac{\pi }{3},\frac{5 \pi }{18}\right\} & \left\{\frac{\pi }{3},\frac{\pi }{3}\right\} \\
\end{array}\end{split}
\end{equation*}}\label{\detokenize{collective/fast-modes:id2}}\end{figure}

When we have symmetric geometry, the instability region is gone. Such a result is exactly what we expect. However, different
\begin{figure}[htbp]
\centering
\capstart

\noindent\sphinxincludegraphics{{collective/assets/fast-modes-fast-mode-no-matter-asymmetric-alpha-0.8}.png}
\caption{Linear stability analysis for}{\small \begin{equation*}
\begin{split}\omega_v =& 0\\
\lambda =& 0\\
\alpha = & 0.8 \\
\theta_L =& 2\pi/9 \\
\theta_R =& \pi/6.\end{split}
\end{equation*}}\label{\detokenize{collective/fast-modes:id3}}\end{figure}


\subsection{Regions of Instability}
\label{\detokenize{collective/fast-modes:regions-of-instability}}
For convinience, we define some quantities for four beam case.
\begin{enumerate}
\item {} 
We define the a parameter \(\alpha=(1-a)/(1+a)\) so that \(\alpha \in [0,\infty]\) is mapped onto \(a\in [-1,1]\).

\item {} 
The summation of the two angles \(\Sigma\theta=\theta_1+\theta_2\) and the difference between two angles \(\Delta\theta=\theta_1-\theta_2\), where \(\theta_1\) is for neutrino beams.

\item {} 
Every quantity is in unit of \(\mu\).

\end{enumerate}

First we check the result without matter, without vacuum frequency, and \(\Sigma\theta=2\pi/3\).
\begin{figure}[htbp]
\centering
\capstart

\noindent\sphinxincludegraphics{{plt2-sigmatheta-2Pi-divided-by-3-mk-divided-by-mu-0-lambda-divided-by-mu-0}.png}
\caption{No matter, no vacuum frequency}\label{\detokenize{collective/fast-modes:id4}}\end{figure}

We can also check the matter effect.
\begin{figure}[htbp]
\centering
\capstart

\noindent\sphinxincludegraphics{{plt2-sigmatheta-2Pi-divided-by-3-mk-divided-by-mu-0-lambda-divided-by-mu-1}.png}
\caption{With matter, no vacuum frequency.}\label{\detokenize{collective/fast-modes:id5}}\end{figure}
\begin{figure}[htbp]
\centering
\capstart

\noindent\sphinxincludegraphics{{plt2-sigmatheta-2Pi-divided-by-3-mk-divided-by-mu-0-lambda-divided-by-mu-10}.png}
\caption{With matter, no vacuum frequency.}\label{\detokenize{collective/fast-modes:id6}}\end{figure}
\begin{figure}[htbp]
\centering
\capstart

\noindent\sphinxincludegraphics{{plt2-sigmatheta-2Pi-divided-by-3-mk-divided-by-mu-0-lambda-divided-by-mu-100}.png}
\caption{With matter, no vacuum frequency.}\label{\detokenize{collective/fast-modes:id7}}\end{figure}

Then we check the result without matter, without vacuum frequency, and \(\Sigma\theta=2\pi/3\), and \(\frac{m k}{\mu}=0.1\).
\begin{figure}[htbp]
\centering
\capstart

\noindent\sphinxincludegraphics{{plt2-sigmatheta-2Pi-divided-by-3-mk-divided-by-mu-0.1-lambda-divided-by-mu-0}.png}
\caption{No matter, no vacuum frequency.}\label{\detokenize{collective/fast-modes:id8}}\end{figure}

The effect of \(m k/\mu\) is also similar to matter effect.
\begin{figure}[htbp]
\centering
\capstart

\noindent\sphinxincludegraphics{{plt2-sigmatheta-2Pi-divided-by-3-mk-divided-by-mu-1-lambda-divided-by-mu-0}.png}
\caption{Higher order Fourier modes, without matter, no vacuum frequency.}\label{\detokenize{collective/fast-modes:id9}}\end{figure}
\begin{figure}[htbp]
\centering
\capstart

\noindent\sphinxincludegraphics{{plt2-sigmatheta-2Pi-divided-by-3-mk-divided-by-mu-10-lambda-divided-by-mu-0}.png}
\caption{Higher order Fourier modes, without matter, no vacuum frequency.}\label{\detokenize{collective/fast-modes:id10}}\end{figure}

Matter + Fourier modes also has suppression
\begin{figure}[htbp]
\centering
\capstart

\noindent\sphinxincludegraphics{{plt2-sigmatheta-2Pi-divided-by-3-mk-divided-by-mu-10-lambda-divided-by-mu-100}.png}
\caption{Higher order Fourier modes, with matter, no vacuum frequency.}\label{\detokenize{collective/fast-modes:id11}}\end{figure}
\begin{figure}[htbp]
\centering
\capstart

\noindent\sphinxincludegraphics{{plt2-sigmatheta-2Pi-divided-by-3-mk-divided-by-mu-1-lambda-divided-by-mu-10}.png}
\caption{Higher order Fourier modes, with matter, no vacuum frequency.}\label{\detokenize{collective/fast-modes:id12}}\end{figure}


\section{Directional Instability}
\label{\detokenize{collective/directional-instability::doc}}\label{\detokenize{collective/directional-instability:directional-instability}}
\begin{sphinxadmonition}{note}{Conventions}

We use the metric
\begin{equation*}
\begin{split}\eta \to \mathrm{diag}(+,-,-,-).\end{split}
\end{equation*}\end{sphinxadmonition}

From the paper by \phantomsection\label{\detokenize{collective/directional-instability:id1}}{\hyperref[\detokenize{collective/directional-instability:izaguirre2016}]{\sphinxcrossref{{[}Izaguirre2016{]}}}}, we notice that dispersion relation is related to the direction of `flavor wave vector'.

In this section, we explore the possible significance of the directional instability.


\subsection{Formalism}
\label{\detokenize{collective/directional-instability:formalism}}
\begin{sphinxadmonition}{note}{What does the EoM Describe?}

The equation of motion that describe the flavor transition is not really the equation of motion that describes the dynamics or kinematics of neutrino itself.

It rather describes the `wave of flavor content'.

However, the problem is that the velocity/momentum should be that of neutrinos. How to reconcile?
\end{sphinxadmonition}

To implement the idea of dispersion relation, we need the EoM
\begin{equation*}
\begin{split}i\frac{d}{dt}\rho = [H,\rho],\end{split}
\end{equation*}
where
\begin{equation*}
\begin{split}\frac{d}{dt} = \partial_t + \mathbf v^{\mathrm T} \cdot \boldsymbol \nabla.\end{split}
\end{equation*}
The equation of motion for {\hyperref[\detokenize{collective/some-clarifications:two-beams-model}]{\sphinxcrossref{\DUrole{std,std-ref}{two beams model}}}} is
\phantomsection\label{\detokenize{collective/directional-instability:equation-eqn-line-model-two-beams-linearized-eom-liouville}}\begin{equation}\label{equation:collective/directional-instability:eqn-line-model-two-beams-linearized-eom-liouville}
\begin{split}i (\partial_t+\mathbf v^{\mathrm T}\cdot\boldsymbol\nabla) \begin{pmatrix}
\epsilon_1 \\
\epsilon_2
\end{pmatrix} =
\frac{1}{2}\begin{pmatrix}
(\lambda+ \mu_2 - \eta \omega_1 - \mu_2 \cos(\theta_1-\theta_2) ) & -\mu_2 (1-\cos(\theta_1-\theta_2)) \\
-\mu_1 (1- \cos(\theta_1-\theta_2)) & (\lambda + \mu_1 - \eta \omega_2 - \mu_1 \cos(\theta_1-\theta_2) )
\end{pmatrix}\begin{pmatrix}
\epsilon_1 \\
\epsilon_2
\end{pmatrix}.\end{split}
\end{equation}
\begin{sphinxadmonition}{note}{Caveat of Notations}

One has to notice that this notation is retarded in linearized equation. The velocity is in fact different for different beams. But for simplicity we keep it here. Please do NOT use this notion for linearized equation in any formal writings.

The term
\begin{equation*}
\begin{split}i (\partial_t+\mathbf v^{\mathrm T}\cdot\boldsymbol\nabla) \begin{pmatrix}
\epsilon_1 \\
\epsilon_2
\end{pmatrix}\end{split}
\end{equation*}
is actually
\begin{equation*}
\begin{split}i\begin{pmatrix}
(\partial_t+\mathbf v_1^{\mathrm T}\cdot\boldsymbol\nabla) \epsilon_1 \\
(\partial_t+\mathbf v_2^{\mathrm T}\cdot\boldsymbol\nabla) \epsilon_2
\end{pmatrix}\end{split}
\end{equation*}\end{sphinxadmonition}

Then we Fourier transform the perturbations \(\begin{pmatrix}\epsilon_1 &  \epsilon_2\end{pmatrix}^{\mathrm T}\),
\begin{equation*}
\begin{split}\begin{pmatrix}
\epsilon_1 \\
\epsilon_2
\end{pmatrix} = \int d\Omega\int d^2 k \mathbf Q(\Omega,\mathbf k) e^{-i(\Omega t- \mathbf k \cdot \mathbf r)},\end{split}
\end{equation*}
where
\begin{equation*}
\begin{split}\mathbf Q = \begin{pmatrix}
Q_1 \\
Q_2
\end{pmatrix}.\end{split}
\end{equation*}
Eq. \eqref{equation:collective/directional-instability:eqn-line-model-two-beams-linearized-eom-liouville} becomes
\begin{equation*}
\begin{split}(\Omega - \mathbf v^{\mathrm T}\cdot \mathbf k)\begin{pmatrix}
Q_1 \\
Q_2
\end{pmatrix} =
\frac{1}{2}\begin{pmatrix}
(\lambda+ \mu_2 - \eta \omega_1 - \mu_2 \cos(\theta_1-\theta_2) ) & -\mu_2 (1-\cos(\theta_1-\theta_2)) \\
-\mu_1 (1- \cos(\theta_1-\theta_2)) & (\lambda + \mu_1 - \eta \omega_2 - \mu_1 \cos(\theta_1-\theta_2) )
\end{pmatrix}\begin{pmatrix}
Q_1 \\
Q_2
\end{pmatrix}.\end{split}
\end{equation*}
Define four velocity
\begin{equation*}
\begin{split}u^\mu\to(1, \mathbf v)\end{split}
\end{equation*}
and the four wave vector
\begin{equation*}
\begin{split}k^\mu\to(\Omega, \mathbf k).\end{split}
\end{equation*}
For the two dimensional geometry we have been discussing,
\begin{equation*}
\begin{split}\Omega - \mathbf v^{\mathrm T}\cdot \mathbf k = \Omega- v_x k_x -v_z k_z.\end{split}
\end{equation*}
\begin{sphinxadmonition}{note}{Proceed?}

The analysis following this equation becomes rather tricky. What kind of instability are we interested in? We could, in principle, solve the instability for time domain, which means \(\Omega\) has positive imaginary part. We could also solve the instability in z direction, which corresponds to \(k_z\) has negative imaginary part. Beware that these are related the eigenvalues of the matrix though a sign and some constant.
\end{sphinxadmonition}

What if we are interested in the instabilities in an arbitrary direction? Assume the direction we are interested in is
\begin{equation*}
\begin{split}\mathbf k'\to \begin{pmatrix}
k_1\\
k_2
\end{pmatrix} = \begin{pmatrix}
\cos\gamma & \sin\gamma\\
-\sin\gamma & \cos\gamma
\end{pmatrix}\begin{pmatrix}
k_x\\
k_y
\end{pmatrix}.\end{split}
\end{equation*}
We plug this relation into the linearized equation
\begin{equation*}
\begin{split}(\Omega-\mathbf v^{\mathrm T}\mathbf R^{\mathrm T} \mathbf k')\begin{pmatrix}
Q_1 \\
Q_2
\end{pmatrix} =
\frac{1}{2}\begin{pmatrix}
(\lambda+ \mu_2 - \eta \omega_1 - \mu_2 \cos(\theta_1-\theta_2) ) & -\mu_2 (1-\cos(\theta_1-\theta_2)) \\
-\mu_1 (1- \cos(\theta_1-\theta_2)) & (\lambda + \mu_1 - \eta \omega_2 - \mu_1 \cos(\theta_1-\theta_2) )
\end{pmatrix}\begin{pmatrix}
Q_1 \\
Q_2
\end{pmatrix},\end{split}
\end{equation*}
where
\begin{equation*}
\begin{split}\mathbf R =\begin{pmatrix}
\cos\gamma & \sin\gamma\\
-\sin\gamma & \cos\gamma
\end{pmatrix}.\end{split}
\end{equation*}
We can define velocity vector in new basis
\phantomsection\label{\detokenize{collective/directional-instability:equation-eqn-new-velocity-with-rotation}}\begin{equation}\label{equation:collective/directional-instability:eqn-new-velocity-with-rotation}
\begin{split}\mathbf v' \equiv \begin{pmatrix}
v_1 \\
v_2
\end{pmatrix} = \begin{pmatrix}
\cos\gamma & \sin\gamma\\
-\sin\gamma & \cos\gamma
\end{pmatrix}\begin{pmatrix}
v_x \\
v_z
\end{pmatrix}\end{split}
\end{equation}

\subsection{Instability in Arbitrary Direction}
\label{\detokenize{collective/directional-instability:instability-in-arbitrary-direction}}
Suppose the direction we are interested in is \(k_2\), the equation becomes
\begin{equation*}
\begin{split}k_2 \begin{pmatrix}
v_{2,1} Q_1 \\
v_{2,2} Q_2
\end{pmatrix} =& - \Omega  \begin{pmatrix}
Q_1 \\
Q_2
\end{pmatrix} + k_1 \begin{pmatrix}
v_{1,1} Q_1 \\
v_{1,2} Q_2
\end{pmatrix} \\
&+\frac{1}{2}\begin{pmatrix}
(\lambda+ \mu_2 - \eta \omega_1 - \mu_2 \cos(\theta_1-\theta_2) ) & -\mu_2 (1-\cos(\theta_1-\theta_2)) \\
-\mu_1 (1- \cos(\theta_1-\theta_2)) & (\lambda + \mu_1 - \eta \omega_2 - \mu_1 \cos(\theta_1-\theta_2) )
\end{pmatrix}\begin{pmatrix}
Q_1 \\
Q_2
\end{pmatrix},\end{split}
\end{equation*}
where \(v_1\) and \(v_2\) is obtained from Eq. \eqref{equation:collective/directional-instability:eqn-new-velocity-with-rotation}
\begin{equation*}
\begin{split}v_{1,i} =& \cos\gamma v_x + \sin\gamma v_z \\
=& \cos\gamma \cos\theta_i + \sin\gamma \sin\theta_i \\
=& \cos(\theta_i - \gamma)\\
v_{2,i} =& -\sin\gamma v_x + \cos\gamma v_z \\
=& -\sin\gamma \cos\theta_i + \cos\gamma \sin\theta_i\\
=& \sin(\theta_i - \gamma),\end{split}
\end{equation*}
where \(\theta_i\) is the angle of the corresponding beam i, with angle define in {\hyperref[\detokenize{collective/some-clarifications:two-beams-model}]{\sphinxcrossref{\DUrole{std,std-ref}{two beams model}}}}.

We will define a new angle
\begin{equation*}
\begin{split}\theta_i' = \theta_i - \gamma,\end{split}
\end{equation*}
then the linearized equation will be similar to the one we have solved.

With the definition of the new beam angles, we have
\begin{equation*}
\begin{split}v_{1,i} =& \cos \theta_i' \\
v_{2,i} =& \sin \theta_i'.\end{split}
\end{equation*}
The linearized equation can be rewritten into
\begin{equation*}
\begin{split}-k_2 \begin{pmatrix}
Q_1 \\
Q_2
\end{pmatrix} =& - \Omega \begin{pmatrix}
1/v_{2,1} & 0 \\
0 & 1/v_{2,2}
\end{pmatrix}  \begin{pmatrix}
Q_1 \\
Q_2
\end{pmatrix} + k_1 \begin{pmatrix}
v_{1,1}/v_{2,1} & 0 \\
0 & v_{1,2}/v_{2,2}
\end{pmatrix} \begin{pmatrix}
Q_1 \\
Q_2
\end{pmatrix} \\
&+\frac{1}{2}\begin{pmatrix}
(\lambda+ \mu_2 - \eta \omega_1 - \mu_2 \cos(\theta_1-\theta_2) )/v_{2,1}  & -\mu_2 (1-\cos(\theta_1-\theta_2))/v_{2,1}  \\
-\mu_1 (1- \cos(\theta_1-\theta_2))/v_{2,2} & (\lambda + \mu_1 - \eta \omega_2 - \mu_1 \cos(\theta_1-\theta_2) )/v_{2,2}
\end{pmatrix}\begin{pmatrix}
Q_1 \\
Q_2
\end{pmatrix}.\end{split}
\end{equation*}
Plug in the velocities
\phantomsection\label{\detokenize{collective/directional-instability:equation-eqn-linearized-eom-k2-direction}}\begin{equation}\label{equation:collective/directional-instability:eqn-linearized-eom-k2-direction}
\begin{split}-k_2 \begin{pmatrix}
Q_1 \\
Q_2
\end{pmatrix} =& - \Omega \begin{pmatrix}
1/\sin\theta_1' & 0 \\
0 & 1/\sin\theta_2'
\end{pmatrix}  \begin{pmatrix}
Q_1 \\
Q_2
\end{pmatrix} + k_1 \begin{pmatrix}
\cos\theta_1'/\sin\theta_1' & 0 \\
0 & \cos\theta_2'/\sin\theta_2'
\end{pmatrix} \begin{pmatrix}
Q_1 \\
Q_2
\end{pmatrix} \\
&+\frac{1}{2}\begin{pmatrix}
(\lambda+ \mu_2 - \eta \omega_1 - \mu_2 \cos(\theta_1-\theta_2) )/\sin\theta_1'  & -\mu_2 (1-\cos(\theta_1-\theta_2))/\sin\theta_1'  \\
-\mu_1 (1- \cos(\theta_1-\theta_2))/\sin\theta_2' & (\lambda + \mu_1 - \eta \omega_2 - \mu_1 \cos(\theta_1-\theta_2) )/\sin\theta_2'
\end{pmatrix}\begin{pmatrix}
Q_1 \\
Q_2
\end{pmatrix}.\end{split}
\end{equation}
We notice that
\begin{enumerate}
\item {} 
\(\Omega\) is in the same position as \(\lambda\).

\item {} 
Though \(\theta_i\in [0,\pi/2]\), with the rotation, \(\theta_i\) can be arbitrary. We could in principle have singularity in the terms \(1/\sin\theta_i'\). This is quite interesting because we can delibrately remove one of the beams by make the other one's effective angle close to 0 or Pi.

\end{enumerate}

\begin{sphinxadmonition}{note}{Singularities?}

In fact it's such a bad idea to divide on both sides \(\sin\theta_i'\). The singularities comes from this shitty move. What I prefer is to rewrite the linearized equation into
\phantomsection\label{\detokenize{collective/directional-instability:equation-eqn-linearized-eom-k2-direction-non-singularity}}\begin{equation}\label{equation:collective/directional-instability:eqn-linearized-eom-k2-direction-non-singularity}
\begin{split}\begin{pmatrix}
\Omega - k_1\cos\theta_1' - k_2 \sin\theta_1' & 0\\
0 & \Omega - k_1\cos\theta_2' - k_2 \sin\theta_2'
\end{pmatrix}\begin{pmatrix}
Q_1 \\
Q_2
\end{pmatrix} =\frac{1}{2}\begin{pmatrix}
(\lambda+ \mu_2 - \eta \omega_1 - \mu_2 \cos(\theta_1-\theta_2) )  & -\mu_2 (1-\cos(\theta_1-\theta_2))  \\
-\mu_1 (1- \cos(\theta_1-\theta_2)) & (\lambda + \mu_1 - \eta \omega_2 - \mu_1 \cos(\theta_1-\theta_2) )
\end{pmatrix}\begin{pmatrix}
Q_1 \\
Q_2
\end{pmatrix}.\end{split}
\end{equation}\end{sphinxadmonition}

Dispite the complexity, we assume that translational symmetry is found in direction \(k_1\), leaving only \(k_2\) non zero. We solve the dispersion relation for Eq. \eqref{equation:collective/directional-instability:eqn-linearized-eom-k2-direction}.

Define
\begin{equation*}
\begin{split}\bar\omega_i =& \lambda - \eta\omega_i\\
\bar\mu_i =& \mu_i(1-\cos(\theta_1-\theta_2)).\end{split}
\end{equation*}
Eq. \eqref{equation:collective/directional-instability:eqn-linearized-eom-k2-direction} becomes
\begin{equation*}
\begin{split}k_2 \begin{pmatrix}
Q_1 \\
Q_2
\end{pmatrix} =-\frac{1}{2}\begin{pmatrix}
\frac{\bar\omega_1 + \bar\mu_2-2\Omega+2k_1\cos\theta_1'}{\sin\theta_1'} & -\frac{\bar\mu_2}{\sin\theta_1'} \\
-\frac{\bar\mu_1}{\sin\theta_2'} & \frac{ \bar\omega_2+\bar\mu_1-2\Omega + 2k_1\cos\theta_2' }{\sin\theta_2'}
\end{pmatrix}\begin{pmatrix}
Q_1 \\
Q_2
\end{pmatrix}.\end{split}
\end{equation*}
We find the relation between \(k_2\), \(k_1\) and \(\Omega\) by solving the following determinant.
\begin{equation*}
\begin{split}\left| 2 k_2 \mathbf I +  \begin{pmatrix}
\frac{\bar\omega_1 + \bar\mu_2-2\Omega+2k_1\cos\theta_1'}{\sin\theta_1'} & -\frac{\bar\mu_2}{\sin\theta_1'} \\
-\frac{\bar\mu_1}{\sin\theta_2'} & \frac{ \bar\omega_2+\bar\mu_1-2\Omega + 2k_1\cos\theta_2' }{\sin\theta_2'}
\end{pmatrix} \right| = 0,\end{split}
\end{equation*}
which is in fact the dispersion relation
\begin{equation*}
\begin{split}\left(2k_2 + \frac{\bar\omega_1 + \bar\mu_2 -2\Omega +2k_1\cos\theta_1'}{\sin\theta_1'} \right) \left( 2k_2 + \frac{\bar\omega_2+\bar\mu_1 - 2\Omega + 2k_1\cos\theta_1'}{\sin\theta_2'} \right) + \frac{\bar\mu_1\bar\mu_2}{\sin\theta_1'\sin\theta_2'} =0,\end{split}
\end{equation*}
which is simplified
\begin{equation*}
\begin{split}(2k_2\sin\theta_1'+ \bar\omega_1+\bar\mu_2-2\Omega+2k_1\cos\theta_1')(2k_2\sin\theta_2' + \bar\omega_2 + \bar\mu_1 -2\Omega+2k_1\cos\theta_2') + \bar\mu_1\bar\mu_2 = 0.\end{split}
\end{equation*}
\begin{sphinxadmonition}{note}{Other Methods?}

Is is possible to find the dispersion relation starting from Eq. \eqref{equation:collective/directional-instability:eqn-linearized-eom-k2-direction-non-singularity}?

Since \(\Omega\) has no coefficient in this problem. We move it to the left and find the dispersion relation. In this case we do not have to deal with singularities and the demoninators. And the dispersion is simply the determinant of the matrix if we move everything to oneside.
\end{sphinxadmonition}

We define
\begin{equation*}
\begin{split}\bar\Omega_i = \Omega - k_1\cos\theta_i' - k_2\sin\theta_i'\end{split}
\end{equation*}
so that
\begin{equation*}
\begin{split}(\bar\omega_1 + \bar\mu_2 - 2\bar\Omega_1) (\bar\omega_2+\bar\mu_1 - 2\bar\Omega_2)+ \bar\mu_1\bar\mu_2=0,\end{split}
\end{equation*}
which is a good form for programming.

We know that
\begin{equation*}
\begin{split}\bar\Omega_1\bar\Omega_2=& \Omega^2 + k_1^2\cos\theta_1'\cos\theta_2' + k_2^2\sin\theta_1'\sin\theta_2' + k_1k_2\sin(\theta_1'+\theta_2')\\
(\bar\omega_1+\bar\mu_2)\bar\Omega_1 + (\bar\omega_2+\bar\mu_1)\bar\Omega_2
=& \Omega(\bar\omega_1+\bar\omega_2+\bar\mu_1+\bar\mu_2) \\
&- k_1( (\bar\omega_1+\bar\mu_2)\cos\theta_2' + (\bar\omega_2+\bar\mu_1)\cos\theta_1' ) \\
& -k_2((\bar\omega_1+\bar\mu_2)\sin\theta_2' + (\bar\omega_2+\bar\mu_1)\sin\theta_1').\end{split}
\end{equation*}
To avoid super long equations, we define
\begin{equation*}
\begin{split}A =& \bar\omega_1+\bar\omega_2+\bar\mu_1+\bar\mu_2\\
B_c =& (\bar\omega_1+\bar\mu_2)\cos\theta_2' + (\bar\omega_2+\bar\mu_1)\cos\theta_1' \\
B_s =& (\bar\omega_1+\bar\mu_2)\sin\theta_2' + (\bar\omega_2+\bar\mu_1)\sin\theta_1'\\
c_i=&\cos\theta_i' \\
s_i=& \sin\theta_i\\
s_{i+j}=&\sin(\theta_i'+\theta_j').\end{split}
\end{equation*}
The dispersion relation
\begin{equation*}
\begin{split}4\Omega^2-2A\Omega+2c_1c_2k_1^2 + 2 B_c k_1 + 2s_1s_2 k_2^2 +2 B_s k_2 + 4s_{1+2}k_1k_2 + (\bar\omega_1+\bar\mu_2)(\bar\omega_2+\bar\mu_1) + \bar\mu_1\bar\mu_2 = 0.\end{split}
\end{equation*}
This will be the equation for conic sections.


\subsection{Phase Velocity and Solitary Waves}
\label{\detokenize{collective/directional-instability:phase-velocity-and-solitary-waves}}
\begin{sphinxadmonition}{note}{Solitary Waves in Neutrino Flavor Wave Propagation}

It seems that solitary waves can occur in neutrino flavor oscillations due to its non-linarity.
\end{sphinxadmonition}


\subsection{How to Understand Every Step of Raffelt's Paper}
\label{\detokenize{collective/directional-instability:how-to-understand-every-step-of-raffelt-s-paper}}
Make the formalism 4 dimensional.


\section{Some Clarifications}
\label{\detokenize{collective/some-clarifications::doc}}\label{\detokenize{collective/some-clarifications:some-clarifications}}
I need to understand several phenomena people mentioned.
\begin{enumerate}
\item {} 
Why crossing leads to instability?

\item {} 
What exactly is happening when we break the symmetries?

\end{enumerate}


\subsection{Two Beams Model}
\label{\detokenize{collective/some-clarifications:two-beams-model}}\label{\detokenize{collective/some-clarifications:id1}}
We use a simple two-beam line model.
\begin{figure}[htbp]
\centering
\capstart

\noindent\sphinxincludegraphics{{two-beam-line-model}.png}
\caption{Two-beam model. Using this model we can check the effect of different symmetries.}\label{\detokenize{collective/some-clarifications:id2}}\end{figure}

For convinience of notations, we define a number density distribution function
\begin{equation*}
\begin{split}f(\hat\nu,\omega)= \frac{n(\hat \nu,\omega)}{n_t},\end{split}
\end{equation*}
where \(n_t\) is the total number density of all neutrino emitted, \(n(\hat\nu,\omega)\) is the number density with momentum direction \(\hat \nu\) and energy \(\omega\).

We also define
\begin{equation*}
\begin{split}\mu = \sqrt{2}G_F n_t.\end{split}
\end{equation*}
For two dimensional systems, we can calculate the neutrinos within an angle \([\theta,\theta+d\theta]\)
\begin{equation*}
\begin{split}n_t f(\hat\nu,\omega) d\theta.\end{split}
\end{equation*}
Similarly we can define the angular distribution for antineutrinos.


\subsubsection{All Neutrino Beams}
\label{\detokenize{collective/some-clarifications:all-neutrino-beams}}
If all the beams are neutrinos, but with different energies for the left and right beams. The distribution function for beams is delta function. In fact, each beam is just half of the total neutrino number density \(n_t\).

The Hamiltonian is a sum of vacuum terms, matter terms, and self-interaction terms,
\begin{equation*}
\begin{split}H= H_v + H_m + H_{\nu\nu},\end{split}
\end{equation*}
where
\begin{equation*}
\begin{split}H_v =& - \eta \frac{1}{2}\omega \sigma_3 \\
H_m =& \frac{1}{2}\lambda \sigma_3\\
H_{\nu\nu}^L =& \frac{1}{2}\mu \rho^R (1-\cos(\theta_1-\theta_2))\\
H_{\nu\nu}^R =& \frac{1}{2}\mu \rho^L (1-\cos(\theta_1-\theta_2)).\end{split}
\end{equation*}
To linearize the equation of motion, we define the perturbed density matrix as
\begin{equation*}
\begin{split}\rho = \frac{1}{2}\begin{pmatrix}
1 & \epsilon\\
\epsilon^* & -1
\end{pmatrix},\end{split}
\end{equation*}
where we have removed the trace part because it is alway time independent.

The linearized equation of motion becomes
\phantomsection\label{\detokenize{collective/some-clarifications:equation-eqn-line-model-two-beams-all-neutrino-linearized-eom}}\begin{equation}\label{equation:collective/some-clarifications:eqn-line-model-two-beams-all-neutrino-linearized-eom}
\begin{split}i \partial_z \begin{pmatrix}
\epsilon_1 \\
\epsilon_2
\end{pmatrix} =&  - i \begin{pmatrix}\cot\theta_1\partial_x & 0 \\
0 & \cot\theta_2 \partial_x
\end{pmatrix} \begin{pmatrix}
\epsilon_1 \\
\epsilon_2
\end{pmatrix} \\
&+
\frac{1}{2}\begin{pmatrix}
(\lambda+ \mu_2 - \eta \omega_1 - \mu_2 \cos(\theta_1-\theta_2) )/\sin \theta_1 & -\mu_2 (1-\cos(\theta_1-\theta_2)) /\sin \theta_1\\
-\mu_1 (1- \cos(\theta_1-\theta_2))/\sin\theta_2 & (\lambda + \mu_1 - \eta \omega_2 - \mu_1 \cos(\theta_1-\theta_2) )/\sin\theta_2
\end{pmatrix}\begin{pmatrix}
\epsilon_1 \\
\epsilon_2
\end{pmatrix},\end{split}
\end{equation}
where
\begin{equation*}
\begin{split}\mu_1 =& \sqrt{2}G_F n_{t,1}\\
\mu_2 =& \sqrt{2}G_F n_{t,2}.\end{split}
\end{equation*}
We know that real symmetric matrix has only real eigenvalues, from which we infer that \(\mu_1=\mu_2\) and \(\theta_1=\theta_2\) removes the instability.

For translational symmetric models, that is \(\partial_x\to 0\), we have the eigenvalues
\begin{equation*}
\begin{split}\Omega = \frac{1}{4}(A\pm B),\end{split}
\end{equation*}
where
\begin{equation*}
\begin{split}A=& -\eta \omega_1/\sin\theta_1 - \mu_2 /\sin\theta_1 + \eta \omega_2 /\sin\theta_2 + \mu_1 \xi /\sin\theta_2 + \lambda(1/\sin\theta_1 + 1/\sin\theta_2)  \\
B=& \sqrt{
   -4[(\lambda-\eta\omega_1)(\lambda +\eta\omega_2) + (\lambda (\mu_1-\mu_2) -\eta (\mu_1\omega_1 + \mu_2\omega_2) )\xi ] \sin\theta_1 \sin\theta_2 + [(\lambda + \eta\omega_2 + \mu_1\xi) \sin\theta_1 + (\lambda - \eta \omega_1 - \mu_2\xi) \sin\theta_2 ]^2
}/(\sin\theta_1\sin\theta_2)\\
\xi=&1-\cos(\theta_1-\theta_2).\end{split}
\end{equation*}
\begin{sphinxadmonition}{note}{All Antineutrino Beams}

I only need to change \(\mu_i\to -\bar\mu_i\) and \(\omega_i\to -\bar\omega_i\), where \(\bar\mu=\sqrt{2}G_F \bar n_t\).
\begin{equation*}
\begin{split}i \partial_z \begin{pmatrix}
\epsilon_1 \\
\epsilon_2
\end{pmatrix} =&  - i \begin{pmatrix}\cot\theta_1\partial_x & 0 \\
0 & \cot\theta_2 \partial_x
\end{pmatrix} \begin{pmatrix}
\epsilon_1 \\
\epsilon_2
\end{pmatrix} \\
&+
\frac{1}{2}\begin{pmatrix}
(\lambda-\bar\mu_2 + \eta \bar\omega_1 + \bar\mu_2 \cos(\theta_1-\theta_2) )/\sin \theta_1 & \bar\mu_2 (1-\cos(\theta_1-\theta_2)) /\sin \theta_1 \\
\bar\mu_1 (1- \cos(\theta_1-\theta_2))/\sin\theta_2 & (\lambda -\bar\mu_1 + \eta \bar\omega_2 +\bar\mu_1 \cos(\theta_1-\theta_2) )/\sin\theta_2
\end{pmatrix}\begin{pmatrix}
\epsilon_1 \\
\epsilon_2
\end{pmatrix}\end{split}
\end{equation*}\end{sphinxadmonition}

\begin{sphinxadmonition}{note}{One Antineutrino and One Neutrino Beams}

Assume that the left beam is neutrino beam and the right beam is antineutrno beam. The linearized equation of motion becomes
\begin{equation*}
\begin{split}i\partial_z \begin{pmatrix}
\epsilon_1 \\
\epsilon_2
\end{pmatrix} = & -i\begin{pmatrix}
\cot\theta_1 \partial_x & 0 \\
0 & \cot\theta_2 \partial_x
\end{pmatrix}\begin{pmatrix}
\epsilon_1 \\
\epsilon_2
\end{pmatrix} \\
&+ \frac{1}{2}\begin{pmatrix}
(\lambda - \bar\mu - 2\eta \omega_1 + \bar\mu \cos(\theta_1-\theta_2) )/\sin\theta_1 & \bar\mu (1-\cos(\theta_1-\theta_2))/\sin\theta_1 \\
-\mu(1-\cos(\theta_1-\theta_2))/\sin\theta_2 & (\lambda + \mu + \eta \omega_2 - \mu \cos(\theta_1-\theta_2) )/\sin\theta_2
\end{pmatrix}\begin{pmatrix}
\epsilon_1 \\
\epsilon_2
\end{pmatrix}\end{split}
\end{equation*}\end{sphinxadmonition}


\subsubsection{Simple Cases}
\label{\detokenize{collective/some-clarifications:simple-cases}}
We first consider a simple case, where \(\theta_1=\theta_2\equiv\theta\) \(\lambda=0\), \(\eta=1\), and homogeneous in x direction. For simplicity we define
\begin{equation*}
\begin{split}\mu =& \sqrt{2}G_F (n_1 + n_2)\\
\mu_i =& \mu \frac{n_i}{n_1+n_2}\equiv \mu f_i \\
\xi = & 1-\cos(\theta_1-\theta_2)\\
\omega'_i = & \lambda - \eta\omega_i.\end{split}
\end{equation*}
\begin{sphinxadmonition}{note}{Not A Self-consistant Example}

This is not a self-consistant example since \(\theta_1=\theta_2\) indicates that \(\xi=0\). As we will see, no instability is present in this case.

However, we keep the term \(\xi\) because we need to analyze the effect of symmetry breaking. This example builds up a formalism.
\end{sphinxadmonition}

The equation for perturbations becomes
\phantomsection\label{\detokenize{collective/some-clarifications:equation-eqn-linearized-eom-symmetric-eg}}\begin{equation}\label{equation:collective/some-clarifications:eqn-linearized-eom-symmetric-eg}
\begin{split}i\partial_z\begin{pmatrix}
\epsilon_1 \\
\epsilon_2
\end{pmatrix} = \frac{1}{2\sin\theta} \begin{pmatrix}
\omega'_i + \mu f_2\xi & -\mu f_2 \xi \\
-\mu f_1 \xi & \omega'_2 + \mu f_1 \xi
\end{pmatrix}\begin{pmatrix}
\epsilon_1 \\
\epsilon_2
\end{pmatrix}.\end{split}
\end{equation}
Since \(\mu\) is the most important energy scale in this problem, we scale all energies with it.
\begin{equation*}
\begin{split}i\partial_{\hat z}\begin{pmatrix}
\epsilon_1 \\
\epsilon_2
\end{pmatrix} = \frac{1}{2\sin\theta} \begin{pmatrix}
\hat\omega'_1 +  f_2\xi & - f_2 \xi \\
- f_1 \xi & \hat\omega'_2 +  f_1 \xi
\end{pmatrix}\begin{pmatrix}
\epsilon_1 \\
\epsilon_2
\end{pmatrix},\end{split}
\end{equation*}
where
\begin{equation*}
\begin{split}\partial_{\hat z} =& \frac{d}{\mu dz} \\
\hat \omega'_i =& \frac{\omega'_i}{\mu}.\end{split}
\end{equation*}
The characteristic equation for this equation is
\phantomsection\label{\detokenize{collective/some-clarifications:equation-eqn-two-beam-line-characteristic-eqn-simple}}\begin{equation}\label{equation:collective/some-clarifications:eqn-two-beam-line-characteristic-eqn-simple}
\begin{split}\left( ( \Omega - \hat\omega'_1 - f_2\xi )(\Omega - \hat\omega'_2-f_1\xi) - f_1 f_2 \xi^2 \right) =0,\end{split}
\end{equation}
which is simplified to
\begin{equation*}
\begin{split}(\Omega-\Omega_1)(\Omega-\Omega_2) -f_1f_2\xi^2 = 0,\end{split}
\end{equation*}
where
\begin{equation*}
\begin{split}\Omega_1 = & \hat\omega'_1 + f_2 \xi\\
\Omega_2 = & \hat\omega'_2 + f_1 \xi.\end{split}
\end{equation*}
Complete the square
\begin{equation*}
\begin{split}(\Omega - (\Omega_1 + \Omega_2)/2)^2 = \frac{1}{4}(\Omega_1-\Omega_2) + f_1f_2\xi^2.\end{split}
\end{equation*}
The solution becomes
\begin{equation*}
\begin{split}\Omega = \frac{1}{2}(\Omega_1+\Omega_2)\pm\sqrt{ (\Omega_1-\Omega_2)^2/4 + f_1f_2\xi^2 }.\end{split}
\end{equation*}
The condition to have positive imaginary part is
\begin{equation*}
\begin{split}(\Omega_1-\Omega_2)^2 + 4f_1f_2\xi^2 < 0,\end{split}
\end{equation*}
or
\begin{equation*}
\begin{split}-2\sqrt{-f_1f_2\xi^2}<\Omega_1-\Omega_2<2\sqrt{-f_1f_2\xi^2},\end{split}
\end{equation*}
and \(f_1f_2\xi^2<0\). Recall the meaning of \(f_i\),
\begin{equation*}
\begin{split}f_i = \frac{n_i}{n_1+n_2},\end{split}
\end{equation*}
instability requires that we have a spectrum crossing, i.e., \(n_1\) and \(n_2\) have different signs.

Plug in the definitions of \(\Omega_i\),
\begin{equation*}
\begin{split}-2\sqrt{-f_1f_2\xi^2}< \eta(- \omega_1 + \omega_2)/\mu + (f_2 - f_1)\xi < 2\sqrt{-f_1f_2\xi^2}.\end{split}
\end{equation*}
From this we can infer
\begin{enumerate}
\item {} 
\(f_1f_2\) has to be negative, which means we can NOT have instabilities with only neutrinos or antineutrinos with all the symmetries we assumed. This is \DUrole{highlight-text}{crossing}.

\item {} 
\(-\omega_1+\omega_2=0\) will remove the instability. So we have to have both neutrinos and antineutrinos.

\item {} 
\(f_2-f_1\), \(\eta(\omega_2-\omega_1)\), and \(\mu\) set limit on each other.

\item {} 
\(\theta_1=\theta_2\equiv \theta\) removes the instability since it leads to \(\xi=0\). The emission has to be asymmetric in this simple two beams model. \sphinxstylestrong{This is trivial since equal emission angle means the beams are not colliding.}

\end{enumerate}

\begin{sphinxadmonition}{note}{But why?}

We have these conclusions. But why?

What are the roles of
\begin{enumerate}
\item {} 
\(f_i\),

\item {} 
neutrino beam and antineutrino beam,

\item {} 
hierarchy,

\item {} 
neutrino number density variations,

\item {} 
variations of angular distributions of neutrinos,

\item {} 
variations of energy spectrum of neutrinos.

\end{enumerate}
\end{sphinxadmonition}

\begin{sphinxadmonition}{note}{Real Symmetric and Skew Symmetric}

Another way of understanding this equation is to think of it as the growth of the length of the vector \(\vec v = (\epsilon_1,\epsilon_2)^T\). For an arbitrary matrix differential equation of the form
\begin{quote}
\begin{equation*}
\begin{split}\partial_z \mathbf v = \mathbf A \mathbf v,\end{split}
\end{equation*}
we can always decompose the matrix \(\mathbf A\) into symmetric part and skew-symmetric part
\begin{equation*}
\begin{split}\mathbf A = \frac{1}{2}(\mathbf A + \mathbf A^T) + \frac{1}{2}(\mathbf A - \mathbf A^T) \equiv \mathbf A^+ + \mathbf A^-.\end{split}
\end{equation*}\end{quote}

We can indentify the effect of \(f_1-f_2\) but this is not particularly useful since we can not say anything about the eigenvalues of matrix \(\mathbf A\) from the eigenvalues of matrix \(\mathbf A^+\) and \(\mathbf A^-\).
\end{sphinxadmonition}


\subsection{Breaking Symmetries}
\label{\detokenize{collective/some-clarifications:breaking-symmetries}}
For a line model, the symmetries we have are
\begin{enumerate}
\item {} 
Time translation symmetry;

\item {} 
Translational symmetry along the line;

\item {} 
Energy spectrum of the beams; \sphinxstylestrong{One of particular interest is to have different neutrino spheres for different energies which can be investigated using two beam model.}

\item {} 
Number density of left and right beams;

\item {} 
Angle of left and right beams;

\item {} 
With and without matter.

\end{enumerate}

In this subsection we provide simple pictures of some the symmetries mentioned above.


\subsubsection{Emission Angle Parity Symmetry}
\label{\detokenize{collective/some-clarifications:emission-angle-parity-symmetry}}
The emission angles change the value of \(\xi=1-\cos(\theta_1-\theta_2)\) as well as rescale the quantities by angle dependent factor \(1/\sin\theta_i\).

To see the importance of angles, we can redefine some quantities
\begin{equation*}
\begin{split}\omega''_i=& \frac{\omega'_i}{\sin\theta_i}\\
f''_1=&\frac{f_1}{\sin\theta_2} \\
f''_2=&\frac{f_2}{\sin\theta_1}.\end{split}
\end{equation*}
The we will reach the same characteristic equation as Eq. \eqref{equation:collective/some-clarifications:eqn-two-beam-line-characteristic-eqn-simple}. So the angles serves as shift of energy gap and angular distribution.

The region of instability changes in a convoluted way. Given angles we can always write down the expression and find out.
\begin{enumerate}
\item {} 
The criteria of existance of instability doesn't change.

\item {} 
The region of instability changes.

\end{enumerate}


\subsubsection{Matter Effect}
\label{\detokenize{collective/some-clarifications:matter-effect}}
Including matter will define vacuum frequencies, \(\omega'_i\), which is effectively just a shift of vacuum frequencies. In the symmetric emission case, \(\omega'_1-\omega'_2\) is independent of matter effect. But breaking the emission symmetry generates the degeneracy,
\begin{equation*}
\begin{split}\hat\omega''_1-\hat\omega''_2=( \lambda/\sin\theta_1 - \lambda/\sin\theta_2 + \eta(- \omega_1/\sin\theta_1 + \omega_2/\sin\theta_2) )/\mu`.\end{split}
\end{equation*}\begin{enumerate}
\item {} 
Very large matter density shift the region to very small \(\mu\).

\end{enumerate}


\subsubsection{Translational Symmetry}
\label{\detokenize{collective/some-clarifications:translational-symmetry}}
Translational symmetry is explained by introducing Fourier transform in x direction. For each mode, a term that is proportional to Fourier mode index m. It only appears in diagonal elements, thus is effectively a shift of vacuum frequencies, thus energies of neutrinos.

For each Fourier mode
\begin{equation*}
\begin{split}\begin{pmatrix}
\epsilon_1 \\
\epsilon_2
\end{pmatrix} =  \mathbf Q(\Omega,k) e^{-i(\Omega t- k x)},\end{split}
\end{equation*}
where we set \(\Omega=0\).

First term in RHS of Eq. \eqref{equation:collective/some-clarifications:eqn-line-model-two-beams-all-neutrino-linearized-eom} becomes
\begin{equation*}
\begin{split}- i \begin{pmatrix}\cot\theta_1\partial_x & 0 \\
0 & \cot\theta_2 \partial_x
\end{pmatrix} \begin{pmatrix}
\epsilon_1 \\
\epsilon_2
\end{pmatrix} = k \begin{pmatrix}\cot\theta_1 & 0 \\
0 & \cot\theta_2
\end{pmatrix} \begin{pmatrix}
Q_1 \\
Q_2
\end{pmatrix}.\end{split}
\end{equation*}
We now define \(\hat\omega''_i\),
\begin{equation*}
\begin{split}\hat\omega''_{k,i} = \hat \omega''_i + 2\hat k\cot\theta_i,\end{split}
\end{equation*}
where \(\hat k=k/\mu\).

The k term contributes to the difference between \(\Omega_{k,i}\equiv \hat\omega''_{k,i}+ f''_i\xi\).

\sphinxstylestrong{Instability criteria doesn't change. However, the regime of instability changes.} We also know that the instability region is determined by
\begin{equation*}
\begin{split}\lvert \Delta\hat\omega''_{12} + 2\hat k (\cot \theta_1 - \cot\theta_2) + \Delta f''_{12}\xi \rvert < \sqrt{-f_1f_2\xi^2},\end{split}
\end{equation*}
where \(\Delta \hat \omega''_{12} = \hat\omega''_1-\hat\omega''_2\). The instability region shift from
\begin{equation*}
\begin{split}-\sqrt{-f''_1f''_2\xi^2} -\Delta f''_{12}\xi < (\Delta\omega''_{12} + 2 k(\cot\theta_1-\cot\theta_2))/\mu < \sqrt{-f''_1f''_2\xi^2} -\Delta f''_{12}\xi\end{split}
\end{equation*}
If \(\lvert \Delta\omega''_{12} + 2 k(\cot\theta_1-\cot\theta_2) \rvert\) becomes larger, the region of instability is shifted to larger \(\mu\), i.e., larger number density.


\subsubsection{Number Density of Emission}
\label{\detokenize{collective/some-clarifications:number-density-of-emission}}
A crossing is required to have instability, i.e., \(-f''_1f''_2>0\). Meanwhile the number density on the left and right have little effects on the existance of instability. It shifts the region of instability for \(\mu\).


\subsubsection{Energy of Emission}
\label{\detokenize{collective/some-clarifications:energy-of-emission}}

\subsubsection{Time Translational Symmetry}
\label{\detokenize{collective/some-clarifications:time-translational-symmetry}}
\begin{sphinxadmonition}{note}{Time Translational Symmetry}

How about time translational symmetry? I need to write down the equation of motion that is related to time.

Two limits are of particular interest.
\begin{enumerate}
\item {} 
Adiabatic limit,

\item {} 
Superfast time variants.

\end{enumerate}
\end{sphinxadmonition}


\subsubsection{Numerical Calculations}
\label{\detokenize{collective/some-clarifications:numerical-calculations}}
We assume the two beams have different energy, as indicated by \(\omega_1\) and \(\omega_2\) in Eq. \eqref{equation:collective/some-clarifications:eqn-line-model-two-beams-all-neutrino-linearized-eom}.

For numerical calcualtions, we scale quantities using \(\mu\).

With symmetric angles for the two beams, I didn't find instabilities. However, \(\theta_1\neq \theta_2\) leads to instabilities in IH, which is consistant with our expections.

For NH:

\noindent\sphinxincludegraphics[width=0.310\linewidth]{{line-two-beam-eta-1-lambda-0-mu-10-alpha-0.5-theta1-pi-div-3-theta2-pi-div-6}.png}

\noindent\sphinxincludegraphics[width=0.310\linewidth]{{line-two-beam-eta-1-lambda-0-mu-10-alpha-1.-theta1-pi-div-3-theta2-pi-div-6}.png}

\noindent\sphinxincludegraphics[width=0.310\linewidth]{{line-two-beam-eta-1-lambda-0-mu-10-alpha-1.5-theta1-pi-div-3-theta2-pi-div-6}.png}

\noindent\sphinxincludegraphics[width=0.310\linewidth]{{line-two-beam-eta-1-lambda-0-mu-10-alpha-0.5-theta1-pi-div-6-theta2-pi-div-3}.png}

\noindent\sphinxincludegraphics[width=0.310\linewidth]{{line-two-beam-eta-1-lambda-0-mu-10-alpha-1.-theta1-pi-div-6-theta2-pi-div-3}.png}

\noindent\sphinxincludegraphics[width=0.310\linewidth]{{line-two-beam-eta-1-lambda-0-mu-10-alpha-1.5-theta1-pi-div-6-theta2-pi-div-3}.png}

\noindent\sphinxincludegraphics[width=0.310\linewidth]{{line-two-beam-eta-1-lambda-0-mu-10-alpha-0.5-theta1-pi-div-3-theta2-pi-div-3}.png}

\noindent\sphinxincludegraphics[width=0.310\linewidth]{{line-two-beam-eta-1-lambda-0-mu-10-alpha-1.-theta1-pi-div-3-theta2-pi-div-3}.png}

\noindent\sphinxincludegraphics[width=0.310\linewidth]{{line-two-beam-eta-1-lambda-0-mu-10-alpha-1.5-theta1-pi-div-3-theta2-pi-div-3}.png}


\section{Dispersion Relation}
\label{\detokenize{collective/dispersion-relation:dispersion-relation}}\label{\detokenize{collective/dispersion-relation::doc}}
Reference to notes:

Izaguirre, I., Raffelt, G., \& Tamborra, I. (2016). Fast Pairwise Conversion of Supernova Neutrinos: Dispersion-Relation Approach, 21101(January), 1\textendash{}6. \url{https://doi.org/10.1103/PhysRevLett.118.021101}


\subsection{Polarization Tensor}
\label{\detokenize{collective/dispersion-relation:polarization-tensor}}
In Raffelt's paper, they defined the polarization tensor as
\begin{equation*}
\begin{split}\Pi^{\mu\nu} = \eta^{\mu\nu} + \int \frac{d\Omega}{4\pi} G(\theta,\phi) \frac{v^\mu v^\nu}{k^\mu v_\mu}.\end{split}
\end{equation*}
For numerical calculations, we lower the second index and multiply on both side \(\omega\),
\begin{equation*}
\begin{split}\omega\Pi^\mu_\nu = \omega\delta^\mu_\nu + \int \frac{d\Omega}{4\pi} G(\theta,\phi) \frac{v^\mu v_\nu}{ 1- \frac{k}{\omega} \hat k\cdot \mathbf v },\end{split}
\end{equation*}
where
\begin{equation*}
\begin{split}v^\mu &= \begin{pmatrix} 1 & \sin\theta \cos\phi & \sin\theta \sin\phi & \cos\theta \end{pmatrix}\\
k^\mu &= \begin{pmatrix} \omega & k \sin\theta_k \cos\phi_k & k\sin\theta_k \sin\phi_k & k\cos\theta_k \end{pmatrix}.\end{split}
\end{equation*}

\subsubsection{Parametrization of Polarization Tensor}
\label{\detokenize{collective/dispersion-relation:parametrization-of-polarization-tensor}}
Raffelt et al parametrize \(k^\mu(n)\) where \(n=k/\omega\). Then the polarization tensor is decomposed into two parts,
\begin{equation*}
\begin{split}\eta^{\mu\nu}\end{split}
\end{equation*}
and
\begin{equation*}
\begin{split}\frac{1}{\omega}N^{\mu\nu},\end{split}
\end{equation*}
where
\begin{equation*}
\begin{split}N^{\mu\nu} = \int d\Gamma G_{\mathbf v} \frac{v^\mu v^\nu}{ 1-n \hat{\mathbf k} \cdot \mathbf v },\end{split}
\end{equation*}
with \(k^\mu=(\omega,\mathbf k)\).

\begin{sphinxadmonition}{note}{Note to self}

The actual wave vector that determines the instability is \(K^\mu\) which is related to \(k^\mu\),
\begin{equation*}
\begin{split}k^\mu= K^\mu - (\Lambda^\mu + \Phi^\mu).\end{split}
\end{equation*}
Since \(\Lambda^\mu\) and \(\Phi^\mu\) are real, imaginary part of \(\omega\) (\(\mathbf k\)) equal imaginary part of \(\Omega\) (\(\mathbf K\)). Thus we only discuss the dispersion relation of \(k^\mu\).

Density matrix is written as
\begin{equation*}
\begin{split}\rho = \begin{pmatrix}
1 & \epsilon \\
\epsilon^* & -1
\end{pmatrix}.\end{split}
\end{equation*}
The perturbation \(\epsilon\) is assumed to have the form
\begin{equation*}
\begin{split}\epsilon = Q(\Omega, \mathbf K) e^{ -i( \Omega t - \mathbf K \cdot \mathbf x ) }.\end{split}
\end{equation*}
This assumption indicates that even though we find instabilities, a proper initial condition/boundary condition is required to stimulate this instability.
\end{sphinxadmonition}

The polarization tensor is in fact
\begin{equation*}
\begin{split}\Pi^{\mu\nu} = \eta + \frac{1}{\omega}N^{\mu\nu}.\end{split}
\end{equation*}
The equation of motion becomes
\begin{equation*}
\begin{split}v_\mu \Pi^{\mu\nu} a_\nu =0 ,\end{split}
\end{equation*}
where
\begin{equation*}
\begin{split}a_\nu = - \int d\Gamma v_\nu G_{\mathbf v} Q_{\mathbf v}.\end{split}
\end{equation*}
Since \(v_\mu\) is (component) of a null dual vector, we require \(\Pi^{\mu\nu} a_\nu\) to be (component) proportional to \(v^\mu\). Since we have a lot of directions, different \(v^\mu\) are independent of each other. So we require \(\Pi^{\mu\nu} a_\nu=0\).

Then we need to find the solution to
\begin{equation*}
\begin{split}\mathrm{Det}(\Pi^{\mu\nu})=0,\end{split}
\end{equation*}
which is simplified to
\begin{equation*}
\begin{split}\mathrm{Det}(\omega \eta^{\mu\nu} + N^{\mu\nu}) = 0.\end{split}
\end{equation*}
We can also use the polarization tensor \(\Pi^\mu_\nu\)
\begin{equation*}
\begin{split}\mathrm{Det}(\omega \delta^\mu_\nu + N^{\mu}_\nu) =0\end{split}
\end{equation*}
is the determinant of a matrix
\begin{equation*}
\begin{split}\omega I + \mathbf N^\mu_\nu.\end{split}
\end{equation*}
Equivalently, we only need to find the eigenvalues of \(\mathbf N^\mu_\nu\) then multiply on each of them by negative sign.


\subsubsection{Examples of Parametrization}
\label{\detokenize{collective/dispersion-relation:examples-of-parametrization}}
Raffelt et al proposed that can now solve the dispersion relation by finding the value of \(k^\mu(n)\) for each n. We make the plot \(\omega\) vs \(\mathbf k\).

Here is an example that I calculated.

The axial symmetric system can be calculated easily using this method. The paper gave an example of two polar angle beams with axial symmetry.
\begin{figure}[htbp]
\centering
\capstart

\noindent\sphinxincludegraphics{{listpltOmegan1}.png}
\caption{\(\omega(n)\) for \(G=0.5 \delta(\cos\theta- 0.8) + 0.5 \delta(\cos\theta+0.2)\).}\label{\detokenize{collective/dispersion-relation:id2}}\end{figure}
\begin{figure}[htbp]
\centering
\capstart

\noindent\sphinxincludegraphics{{listpltDispersionRelationDecompose1}.png}
\caption{Dispersion relation.}\label{\detokenize{collective/dispersion-relation:id3}}\end{figure}

We can check what happens for multibeams. I can plot the dispersion relation for similar configuration but with different number of beams.
\begin{figure}[htbp]
\centering
\capstart

\noindent\sphinxincludegraphics{{listanimi1}.png}
\caption{Animition of dispersion relation.}{\small 
\begin{sphinxVerbatim}[commandchars=\\\{\}]
\PYG{n}{dataPltNBeamsPlt}\PYG{p}{[}\PYG{n}{Join}\PYG{p}{[}\PYG{n}{Table}\PYG{p}{[}\PYG{l+m+mi}{1}\PYG{o}{/}\PYG{n}{beams}\PYG{p}{,} \PYG{p}{\PYGZob{}}\PYG{n}{n}\PYG{p}{,} \PYG{l+m+mi}{1}\PYG{p}{,} \PYG{n}{beams}\PYG{o}{/}\PYG{l+m+mi}{2}\PYG{p}{\PYGZcb{}}\PYG{p}{]}\PYG{p}{,}
\PYG{n}{Table}\PYG{p}{[}\PYG{o}{\PYGZhy{}}\PYG{l+m+mi}{1}\PYG{o}{/}\PYG{n}{beams}\PYG{p}{,} \PYG{p}{\PYGZob{}}\PYG{n}{n}\PYG{p}{,} \PYG{l+m+mi}{1}\PYG{p}{,} \PYG{n}{beams}\PYG{o}{/}\PYG{l+m+mi}{2}\PYG{p}{\PYGZcb{}}\PYG{p}{]}\PYG{p}{]}\PYG{p}{,}
\PYG{n}{Table}\PYG{p}{[}\PYG{n}{Pi}\PYG{o}{/}\PYG{l+m+mi}{3} \PYG{o}{+} \PYG{n}{n} \PYG{n}{Pi}\PYG{o}{/}\PYG{l+m+mi}{2}\PYG{o}{/}\PYG{p}{(}\PYG{n}{beams} \PYG{o}{\PYGZhy{}} \PYG{l+m+mi}{1}\PYG{p}{)}\PYG{p}{,} \PYG{p}{\PYGZob{}}\PYG{n}{n}\PYG{p}{,} \PYG{l+m+mi}{0}\PYG{p}{,} \PYG{n}{beams} \PYG{o}{\PYGZhy{}} \PYG{l+m+mi}{1}\PYG{p}{\PYGZcb{}}\PYG{p}{]}\PYG{p}{,} \PYG{p}{\PYGZob{}}\PYG{o}{\PYGZhy{}}\PYG{l+m+mi}{10}\PYG{p}{,} \PYG{l+m+mi}{10}\PYG{p}{\PYGZcb{}}\PYG{p}{,} \PYG{l+m+mf}{0.049}\PYG{p}{,} \PYG{p}{\PYGZob{}}\PYG{p}{\PYGZob{}}\PYG{o}{\PYGZhy{}}\PYG{l+m+mi}{10}\PYG{p}{,} \PYG{l+m+mi}{10}\PYG{p}{\PYGZcb{}}\PYG{p}{,} \PYG{p}{\PYGZob{}}\PYG{o}{\PYGZhy{}}\PYG{l+m+mi}{10}\PYG{p}{,} \PYG{l+m+mi}{10}\PYG{p}{\PYGZcb{}}\PYG{p}{\PYGZcb{}}\PYG{p}{]}
\end{sphinxVerbatim}
}\label{\detokenize{collective/dispersion-relation:id4}}\end{figure}

I plot the \(\omega(n)\) relation for different number of beams

\noindent\sphinxincludegraphics[width=0.490\linewidth]{{listpltOmegan12List-2}.png}

\noindent\sphinxincludegraphics[width=0.490\linewidth]{{listpltOmegan12List-4}.png}

\noindent\sphinxincludegraphics[width=0.490\linewidth]{{listpltOmegan12List-8}.png}

\noindent\sphinxincludegraphics[width=0.490\linewidth]{{listpltOmegan12List-10}.png}

Similar to the previous example, confining the range of \(n\) leads to only a partial patch of the dispersion relation.
\begin{figure}[htbp]
\centering
\capstart

\noindent\sphinxincludegraphics{{pltDiffBeamsConfined-n-in--1-to-1-beams-10}.png}
\caption{The code for it}{\small 
\begin{sphinxVerbatim}[commandchars=\\\{\}]
\PYG{n}{pltDiffBeamsConfined}\PYG{p}{[}\PYG{n}{beams\PYGZus{}}\PYG{p}{]} \PYG{p}{:}\PYG{o}{=} \PYG{n}{dataPltNBeamsPlt}\PYG{p}{[}
\PYG{n}{Join}\PYG{p}{[}\PYG{n}{Table}\PYG{p}{[}\PYG{l+m+mi}{1}\PYG{o}{/}\PYG{n}{beams}\PYG{p}{,} \PYG{p}{\PYGZob{}}\PYG{n}{n}\PYG{p}{,} \PYG{l+m+mi}{1}\PYG{p}{,} \PYG{n}{beams}\PYG{o}{/}\PYG{l+m+mi}{2}\PYG{p}{\PYGZcb{}}\PYG{p}{]}\PYG{p}{,}
\PYG{n}{Table}\PYG{p}{[}\PYG{o}{\PYGZhy{}}\PYG{l+m+mi}{1}\PYG{o}{/}\PYG{n}{beams}\PYG{p}{,} \PYG{p}{\PYGZob{}}\PYG{n}{n}\PYG{p}{,} \PYG{l+m+mi}{1}\PYG{p}{,} \PYG{n}{beams}\PYG{o}{/}\PYG{l+m+mi}{2}\PYG{p}{\PYGZcb{}}\PYG{p}{]}\PYG{p}{]}\PYG{p}{,}
\PYG{n}{Table}\PYG{p}{[}\PYG{n}{Pi}\PYG{o}{/}\PYG{l+m+mi}{3} \PYG{o}{+} \PYG{n}{n} \PYG{n}{Pi}\PYG{o}{/}\PYG{l+m+mi}{2}\PYG{o}{/}\PYG{p}{(}\PYG{n}{beams} \PYG{o}{\PYGZhy{}} \PYG{l+m+mi}{1}\PYG{p}{)}\PYG{p}{,} \PYG{p}{\PYGZob{}}\PYG{n}{n}\PYG{p}{,} \PYG{l+m+mi}{0}\PYG{p}{,} \PYG{n}{beams} \PYG{o}{\PYGZhy{}} \PYG{l+m+mi}{1}\PYG{p}{\PYGZcb{}}\PYG{p}{]}\PYG{p}{,} \PYG{p}{\PYGZob{}}\PYG{o}{\PYGZhy{}}\PYG{l+m+mi}{1}\PYG{p}{,} \PYG{l+m+mi}{1}\PYG{p}{\PYGZcb{}}\PYG{p}{,}
\PYG{l+m+mf}{0.049}\PYG{p}{,} \PYG{p}{\PYGZob{}}\PYG{p}{\PYGZob{}}\PYG{o}{\PYGZhy{}}\PYG{l+m+mi}{10}\PYG{p}{,} \PYG{l+m+mi}{10}\PYG{p}{\PYGZcb{}}\PYG{p}{,} \PYG{p}{\PYGZob{}}\PYG{o}{\PYGZhy{}}\PYG{l+m+mi}{10}\PYG{p}{,} \PYG{l+m+mi}{10}\PYG{p}{\PYGZcb{}}\PYG{p}{\PYGZcb{}}\PYG{p}{]}
\end{sphinxVerbatim}
}\label{\detokenize{collective/dispersion-relation:id5}}\end{figure}

\begin{sphinxadmonition}{note}{This should be the continuous limit?}

As a comparison, we can plot the dispersion relation in a larger range of n for 10 beams.
\begin{figure}[htbp]
\centering
\capstart

\noindent\sphinxincludegraphics{{listpltOmegan12List-10}.png}
\caption{10 beams.}\label{\detokenize{collective/dispersion-relation:id6}}\end{figure}

On the other hand, we can calculate the continuous limit for the same angle range.
\begin{figure}[htbp]
\centering
\capstart

\noindent\sphinxincludegraphics{{compare-continuous-and-10-beams-within-n-range--1-to-1}.png}
\caption{Dispersion relation for 10 beams (\(n\in [-1,1]\)), and continuous limit.}\label{\detokenize{collective/dispersion-relation:id7}}\end{figure}

MEH
\end{sphinxadmonition}


\subsection{Analyze the Symmetries in Polarization Tensor}
\label{\detokenize{collective/dispersion-relation:analyze-the-symmetries-in-polarization-tensor}}
\begin{sphinxadmonition}{note}{Vectors Using Spherical Harmonics}

Four velocity can be expressed in terms of spherical harmonics.
\begin{equation*}
\begin{split}v^\mu = \sqrt{\pi}\begin{pmatrix} 2 & \sqrt{2/3} (Y_1^{-1} - Y_1^1) & i \sqrt{2/3} (Y_1^{-1} + Y_1^1) & 2\sqrt{1/3} Y_1^0 \end{pmatrix}.\end{split}
\end{equation*}\end{sphinxadmonition}

In principle, solving the dispersion relation is not easy. Neverthless, symmetries would significantly simplify the problem.

Axial symmetry indicates that the integrals of first orders of \(\sin\phi\cos\phi\), \(\sin\phi\), and \(\cos\phi\) are 0 on the range \(\phi\in [0,2\Pi]\).

We denote the integral
\begin{equation*}
\begin{split}\int \frac{d\Omega}{4\pi} G(\theta,\phi) \frac{v^\mu v_\nu}{\omega- k \hat{\mathbf k}\cdot \mathbf v}\end{split}
\end{equation*}
as \(P^\mu_\nu\). The polarization tensor becomes
\begin{equation*}
\begin{split}\Pi^\mu_{\phantom{\mu}\nu} = I + P^\mu_{\phantom{\mu}\nu}.\end{split}
\end{equation*}
For axial symmetric emission, only terms \(P^0_{\phantom{0}0}, P^0_{\phantom{0}3}, P^3_{\phantom{3}0}, P^3_{\phantom{3}3}, P^1_{\phantom{1}1}, P^2_{\phantom{2}2}\) are nonzero, given \(\mathbf k\) in z direction, i.e., \(\phi_k=\theta_k=0\).

To simplify the calcualtion, we denote \(n=\frac{\lvert \mathbf k\rvert}{\omega}\). We will NOT solve \(\omega(n)\). Instead we write down the form of the eigenvalues of
\begin{equation*}
\begin{split}N^\mu_{\phantom{\mu}\nu} = \omega P^\mu_{\phantom{\mu}\nu},\end{split}
\end{equation*}
which shows is an analytical expression of \(\omega\). We do not solve this relation. Instead, we plugin the definition \(n=\frac{\lvert \mathbf  k\rvert}{\omega}\) and find out the relation between \(\omega\) and \(k=\lvert \mathbf k \rvert\)

We define
\begin{equation*}
\begin{split}I_n(\theta)=\int_{\cos\theta_2}^{\cos\theta_1} d\cos\theta G(\theta) \frac{\cos^n\theta}{1 - n \cos\theta },\end{split}
\end{equation*}
where \(\theta_1\) and \(\theta_2\) are

Since
\begin{equation*}
\begin{split}\int_0^{2\pi} d\phi&=2\pi \\
\int_0^{2\pi} d\phi \cos^2\phi &=\int_0^{2\phi} d\phi \sin^2\phi = \pi,\end{split}
\end{equation*}
the matrix \(N^\mu_{\phantom{\mu}\nu}\) is simplified,
\begin{equation*}
\begin{split}N^\mu_{\phantom{\mu}\nu} = \omega P^\mu_{\phantom{\mu}\nu}\to  \begin{pmatrix}
\frac{1}{2}  I_0 & 0 & 0 & -\frac{1}{2}I_1\\
0 & -\frac{1}{4}(I_0-I_2) & 0 & 0\\
0 & 0 & -\frac{1}{4}(I_0-I_2) & 0 \\
\frac{1}{2}I_1 & 0 & 0 & -\frac{1}{2}I_2
\end{pmatrix}.\end{split}
\end{equation*}
We express the eigen values of matrix \(N^\mu_{\phantom{\mu}\nu}\), which we denote as \(\lambda_N\),
\phantomsection\label{\detokenize{collective/dispersion-relation:equation-eqn-omega-n-relation-axial-sym-general}}\begin{equation}\label{equation:collective/dispersion-relation:eqn-omega-n-relation-axial-sym-general}
\begin{split}\omega = -\lambda_N = \frac{1}{4}(I_0-I_2), \quad -\frac{1}{4}\left(I_0-I_2\pm \sqrt{ (I_0-2I_1+I_2)(I_0+2I_1+I_2) }\right).\end{split}
\end{equation}
We plug in the definition \(n=k/\omega\) then solve dispersion relation from each of the solutions in Eq. \eqref{equation:collective/dispersion-relation:eqn-omega-n-relation-axial-sym-general}.

The questions are
\begin{enumerate}
\item {} 
What does each of the solutions mean?

\end{enumerate}


\subsubsection{Eigenvalues and Axial Symmetry}
\label{\detokenize{collective/dispersion-relation:eigenvalues-and-axial-symmetry}}
By definition, the meaning of polarization tensor, \(\Pi^\mu_\nu a^\nu=0\) implies that \(a^1\) and \(a^2\) are the \(\phi\) angle dependent components. To prove this, we rewrite \(Q\),
\begin{equation*}
\begin{split}Q = \frac{a^\mu v_\mu}{k^\nu v_\nu},\end{split}
\end{equation*}
which clearly shows that the 1, and 2 component of \(a^\mu\) is related to the phi dependence of \(Q\). \(a^1=a^2=0\) indicates that \(Q\) has no \(\phi\) dependence.

\begin{sphinxadmonition}{note}{Is this related to eigenvectors?}

The eigenvalue \(\frac{1}{4}(I_0-I_2)\) of matrix \(N^\mu_{\phantom{\mu}\nu}\) corresponds to eigenvectors \((0,0,1,0)\) and \((0,1,0,0)\).

I don't think it is related to eigenvalues. However, eigenvalues set limit on the actual solution. When we write down the solution to \(a^\mu\), the coefficients are related to each other because we have determinant of coefficient matrix being 0. There are degeneracies.
\end{sphinxadmonition}

That is to say, the part
\begin{equation*}
\begin{split}\begin{pmatrix}
-\frac{1}{4}(I_0-I_2) & 0 \\
0 & -\frac{1}{4}(I_0-I_2)
\end{pmatrix}\end{split}
\end{equation*}
are the only elements that determines the whether we have a \(\phi\) dependence in \(Q\), since this is the only part that needs to be acted on in Gaussian elimination method. It is obvious that we have
\begin{equation*}
\begin{split}a^1=a^2=0.\end{split}
\end{equation*}
In turn, it determines the angle dependence of \(Q\),
\begin{equation*}
\begin{split}Q =  \frac{a^0 - a^1\sin\theta\cos\phi - a^2\sin\theta\sin\phi - a^3 \cos\theta}{k^\mu v_\mu} = Q_0 + Q_3(\theta).\end{split}
\end{equation*}
We have no \(\phi\) dependence in \(Q\) if we foce the emission to be axially symmetric.


\subsubsection{Continuous Emission within Angle Range}
\label{\detokenize{collective/dispersion-relation:continuous-emission-within-angle-range}}
In this case we have to calculate \(I_n\) specifically for the angle range, then plug in the expression \(n=k/\omega\) to find the dispersion relation.


\subsubsection{Discrete Emission Beams}
\label{\detokenize{collective/dispersion-relation:discrete-emission-beams}}
For discrete emission \(G(\theta)=\sum_i G_i \delta(\cos\theta-\cos\theta_i)\), we can define new quantities
\begin{equation*}
\begin{split}\tilde I_n(\theta)= \sum_i G_i \frac{\cos^n\theta_i}{1 - n \cos\theta_i }.\end{split}
\end{equation*}
Thus
\begin{equation*}
\begin{split}&\omega = \frac{1}{4}(I_0-I_2) \\
\Rightarrow &\omega = \frac{1}{4} \sum_i G_i \frac{1-\cos^2\theta_i}{1 - n \cos\theta_i }.\end{split}
\end{equation*}
For two sets of beams, we have
\begin{equation*}
\begin{split}4 =  G_1 \frac{1-\cos^2\theta_1}{\omega - k\cos\theta_1 } + G_2 \frac{1-\cos^2\theta_2}{\omega - k \cos\theta_2 },\end{split}
\end{equation*}
which is a conic section. We have already used \(n=k/\omega\).

\begin{sphinxadmonition}{note}{Hyperbola}

For an quadratic equation \phantomsection\label{\detokenize{collective/dispersion-relation:id1}}{\hyperref[\detokenize{collective/dispersion-relation:hyperbolawikipedia}]{\sphinxcrossref{{[}HyperbolaWikipedia{]}}}}
\begin{equation*}
\begin{split}A_{xx} x^2 + 2 A_{xy} xy + A_{yy} y^2 + 2 B_x x + 2 B_y y + C = 0,\end{split}
\end{equation*}
it is hyperbola if
\begin{equation*}
\begin{split}D := \begin{vmatrix} A_{xx} & A_{xy}\\A_{xy} & A_{yy} \end{vmatrix} < 0.\end{split}
\end{equation*}
Center of the hyperbola \((x_c,y_c)\) is
\begin{equation*}
\begin{split}x_c &= -\frac{1}{D} \begin{vmatrix} B_x & A_{xy} \\ B_y & A_{yy} \end{vmatrix}\\
y_c &= -\frac 1 D \begin{vmatrix} A_{xx} & B_x \\A_{xy} & B_y \end{vmatrix}.\end{split}
\end{equation*}
Principal axis is tilted away from x axis by angle \(\beta\)
\begin{equation*}
\begin{split}\tan 2\beta = \frac{2A_{xy}}{A_{xx}-A_{yy}}.\end{split}
\end{equation*}\end{sphinxadmonition}

We can prove that this is a hyperbola. Simplify the equation to standard form of conic sections
\begin{equation*}
\begin{split}4\cos\theta_1\cos\theta_2 k^2 - 2 \times 2(\cos \theta_1+\cos\theta_2) \omega k +4\omega^2  + 2 \times \frac{ G_1(1-\cos^2\theta_1)\cos\theta_2 + G_2(1-\cos^2\theta_2)\cos\theta_1 }{2} k - 2 \times\frac{ G_1(1-\cos^2\theta_1) + G_2(1-\cos^2\theta_2) }{2}  \omega = 0.\end{split}
\end{equation*}
\begin{sphinxadmonition}{note}{The \(\omega~k\) plane}

We use \(\omega~k\) plane, so that we map \(k\) to \(x\) and \(\omega\) to \(y\).

So the coefficients are defined as
\begin{equation*}
\begin{split}A_{kk} &= 4\cos\theta_1\cos\theta_2\\
A_{k\omega} &= -2(\cos\theta_1+\cos\theta_2)\\
A_{\omega\omega} & = 4 \\
B_k & =  \frac{ G_1(1-\cos^2\theta_1)\cos\theta_2 + G_2(1-\cos^2\theta_2)\cos\theta_1 }{2} = \frac{\tilde G_1 \cos\theta_2 + \tilde G_2 \cos\theta_1}{2}\\
B_\omega & =  - \frac{ G_1(1-\cos^2\theta_1) + G_2(1-\cos^2\theta_2) }{2} = -\frac{\tilde G_1 + \tilde G_2}{2}.\end{split}
\end{equation*}\end{sphinxadmonition}

The condition for it to be hyperbola is \(D<0\), where
\begin{equation*}
\begin{split}D = \begin{vmatrix}
4\cos\theta_1\cos\theta_2 & -2(\cos\theta_1+\cos\theta_2) \\
-2(\cos\theta_1+\cos\theta_2)  & 4
\end{vmatrix} = - 4 (\cos\theta_1-\cos\theta_2)^2.\end{split}
\end{equation*}
As long as we have different angles, \(D\) is always less than 0. We always have a hyperbola. The center of the hyperbola is
\begin{equation*}
\begin{split}k_c &= -\frac{1}{D} \begin{vmatrix}
B_k & A_{k\omega} \\
B_\omega & A_{\omega\omega}
\end{vmatrix} = \frac{1}{4 (\cos\theta_1-\cos\theta_2)^2} \begin{vmatrix}
\frac{ \tilde G_1\cos\theta_2 + \tilde G_2\cos\theta_1 }{2} & -2(\cos\theta_1+\cos\theta_2)\\
- \frac{ \tilde G_1 + \tilde G_2 }{2} & 4
\end{vmatrix}   \\
\omega_c &= -\frac{1}{D} \begin{vmatrix}
A_{kk} & B_k \\
A_{k\omega} & B_\omega
\end{vmatrix} = \frac{1}{4 (\cos\theta_1-\cos\theta_2)^2} \begin{vmatrix}
4\cos\theta_1\cos\theta_2 & \frac{ \tilde G_1\cos\theta_2 + \tilde G_2\cos\theta_1 }{2}\\
-2(\cos\theta_1+\cos\theta_2) & - \frac{ \tilde G_1 + \tilde G_2 }{2}
\end{vmatrix}  .\end{split}
\end{equation*}
A special case for it is \(G_1=\pm G_2\), however, the expression for the center doesn't really simplify that much.

We are interested in gaps, so the asymptotic lines are the lines that we are interested in.

First of all, we need to find out the principal axis. The angle between the principal axis and x axis is defined to be \(\beta\),
\begin{equation*}
\begin{split}\tan2 \beta = \frac{2A_{k\omega}}{A_{kk}-A_{\omega\omega}} = \frac{\cos\theta_1+\cos\theta_2}{1-\cos\theta_1\cos\theta_2}.\end{split}
\end{equation*}
Suppose we have angles \(\theta'_i=\arctan ( \cos\theta_i )\),
\begin{equation*}
\begin{split}\beta = \frac{\theta'_1+\theta'_2}{2}.\end{split}
\end{equation*}
This indicates that the angle \(\beta\) is always within range \(\beta \in [\pi/4,\pi/4]\).

\begin{sphinxadmonition}{note}{The other solutions}

For the solutions
\begin{equation*}
\begin{split}\omega = -\frac{1}{4}\left(I_0-I_2\pm \sqrt{ (I_0-2I_1+I_2)(I_0+2I_1+I_2) }\right),\end{split}
\end{equation*}
it becomes much more complicated.
\end{sphinxadmonition}


\subsection{Why is it called Polarization Tensor}
\label{\detokenize{collective/dispersion-relation:why-is-it-called-polarization-tensor}}
Why is \(a^\mu\) called polarization vector?

We kind of see why \(a^\mu\) is some kind of polarization given the definition
\begin{equation*}
\begin{split}a^\mu = -\int d\Gamma v^\mu G(\theta,\phi) Q(\theta,\phi).\end{split}
\end{equation*}
In some sense it is a weighted average of \(Q\). And \(S = Q e^{-i k^\mu x_\mu}\) is the ``field'' we are insterested in.

Comparing to electrodynamics, where we have the field \(A^\mu\) which tells us about the polarization,

\begin{sphinxadmonition}{note}{Polarization in Electrodynamics}

We can assume that the four vector field is
\begin{equation*}
\begin{split}A^\mu = \epsilon^\mu a e^{-ikx} + {\epsilon^\mu}^*  a^* e^{ikx} \qquad \text{with}\quad k_\mu k^\mu =0.\end{split}
\end{equation*}
We make sense of it by interpretating \(\epsilon^\mu\) as the polarization vector and \(a\) as the amplitude of the field strength. This solution is only for one particular case. We use it as an example because it is simple.

To calculate the electric field, \(E^i\) where \(i=1,2,3\), we apply the definition of it
\begin{equation*}
\begin{split}E^i = - F^{0i}.\end{split}
\end{equation*}
By working it out, we find \(\epsilon^i\), which is the spatial part of \(\epsilon^\mu\), indeed plays a role in the direction of field.
\end{sphinxadmonition}


\subsection{Solving Continuous Emission}
\label{\detokenize{collective/dispersion-relation:solving-continuous-emission}}
Suppose neutrinos are emitted within a angle range \([\theta_1,\theta_2]\). Using Mathematica, we find the three important integrals
\begin{equation*}
\begin{split}I_0 &= \int_{c_2 }^{c_1 } d x \frac{1}{1-k\cos\theta/\omega} \\
   &= \frac{\omega}{k} \ln \left( \left\vert\frac{\omega-c_2 k}{\omega-c_1 k} \right\vert \right) \\
I_1 &= \int_{c_2 }^{c_1 } d x \frac{x}{1-k\cos\theta/\omega} \\
& =  \frac{\omega}{\omega} \left( c_2 -c_1  +  \frac{ \omega }{k} \ln \left(\left\vert \frac{\omega-c_2 k}{\omega-c_1 k} \right\vert \right) \right) \\
I_2 &= \int_{c_2 }^{c_1 } d x \frac{x^2}{1-k\cos\theta/\omega} \\
&= \frac{\omega}{k} \left(  (c_2 -c_1 )\left(\frac{\omega}{k} + c_1 +c_2  \right) + \left(\frac{\omega}{k}\right)^2 \ln \left( \left\vert \frac{\omega-c_2 k}{\omega-c_1 k} \right\vert \right) \right).\end{split}
\end{equation*}
where \(c_1=\cos\theta_1\) and \(c_2=\cos\theta_2\).


\subsubsection{Homogeneous Emission}
\label{\detokenize{collective/dispersion-relation:homogeneous-emission}}
Assuming \(G=1\), the MAA solution is
\begin{equation*}
\begin{split}-4\omega = I_0 - I_2,\end{split}
\end{equation*}
which becomes
\phantomsection\label{\detokenize{collective/dispersion-relation:equation-eqn-dr-continuous-angle-range}}\begin{equation}\label{equation:collective/dispersion-relation:eqn-dr-continuous-angle-range}
\begin{split}\omega=\frac{(c_2-c_1)(1+(c_1+c_2)n/2)}{4n^3} + \frac{ 1-n^2 }{4n^3}\ln\left( \left\vert \frac{1-c_2 n}{1-c_1 n} \right\vert \right)\end{split}
\end{equation}
Meanwhile we could write down the MZA/bimodal solution in the form of \(\omega(n)\).

For MAA and MZA we can plot \(\omega\) as a function of n.
\begin{figure}[htbp]
\centering
\capstart

\noindent\sphinxincludegraphics{{DR-omega-k-direct-continuous-maa-no-crossing}.png}
\caption{\(\omega(n)\) for MAA solution.}\label{\detokenize{collective/dispersion-relation:id8}}\end{figure}
\begin{figure}[htbp]
\centering
\capstart

\noindent\sphinxincludegraphics{{DR-omega-k-direct-continuous-mza-no-crossing}.png}
\caption{\(\omega(n)\) for MAA solution.}\label{\detokenize{collective/dispersion-relation:id9}}\end{figure}

On the other hand, we know \(k=n\omega\), so we have parametrized the dispersion relation using a parameter \(n\).
\begin{figure}[htbp]
\centering
\capstart

\noindent\sphinxincludegraphics{{DR-omega-k-direct-continuous-dr-no-crossing}.png}
\caption{Dispersion relation out of Eq. \eqref{equation:collective/dispersion-relation:eqn-dr-continuous-angle-range}. For large k, the relation becomes proportional. The discontinuties are at location of \(\lim_{n\to \infty}\{n\omega(n),\omega(n)\}\). In this example, \(\{ -0.184653, 0 \}\) (MAA solution) and \(\{  0.729306, 0 \}\) (MZA solution).}\label{\detokenize{collective/dispersion-relation:id10}}\end{figure}

\begin{sphinxadmonition}{note}{The Limits}

There are several limits in the dispersion relation.

From the figure of \(\omega\), we notice the singularities at the two ends of the distribution, \(1/c_1\) and \(1/c_2\). At these points, a tiny change of n will cause a significant change in \(\omega\) and \(k\). In fact the relation between them becomes a proportional relation since \(n\) remains almost constant.

Another limit is \(n\to \infty\). Since \(\ln(n)\) increases slower than \(n\), we have
\begin{equation*}
\begin{split}\lim_{n\to\infty}\omega(n) =\lim_{n\to\infty} \frac{ I_0-I_2 }{4} = 0.\end{split}
\end{equation*}
We can calculate \(\lim_{n\to \infty}\{n\omega(n),\omega(n)\}\), for \(c_1=0.9\) and \(c_1=0.3\),
\begin{equation*}
\begin{split}\lim_{n\to\infty}\{n\omega(n),\omega(n)\}  &= \{-0.184653, 0.  \}, &\qquad \text{for MAA solution}\\
\lim_{n\to\infty}\{n\omega(n),\omega(n)\}  &= \{0.729306, 0.  \}, &\qquad \text{for MZA solution}.\end{split}
\end{equation*}
In general, we have the limits
\begin{equation*}
\begin{split}\lim_{n\to\infty}\{n\omega(n),\omega(n)\}  &= \{(c_1^2-c_2^2 + 2\ln \left\vert\frac{c_2}{c_1}\right\vert )/8, 0  \}, &\qquad \text{for MAA solution} \\
\lim_{n\to\infty}\{n\omega(n),\omega(n)\}  &= \{ ( c_1^2-c_2^2 - 2\ln \left\vert \frac{c_2}{c_1} \right\vert )/4, 0  \}, &\qquad \text{for MZA solution}.\end{split}
\end{equation*}\end{sphinxadmonition}


\subsubsection{Emission with Crossing}
\label{\detokenize{collective/dispersion-relation:emission-with-crossing}}
I have to break each of the integral into two parts. I calculate \(I_0-I_2\) for the first region then add to it the second region. Within a region \([\theta_1,\theta_2]\) and
\begin{equation*}
\begin{split}G=\begin{cases}
g_1, \qquad & \theta_1<\theta<\theta_0 \\
g_2, \qquad & \theta_0<\theta<\theta_2
\end{cases}.\end{split}
\end{equation*}
For MAA solution we define a function,
\begin{equation*}
\begin{split}RHS(c_1,c_2,g)=\frac{I_0-I_2}{4} = \frac{g}{4} \left[ \left( \frac{1}{n} - \frac{1}{n^3} \right)\ln\left( \frac{1-n c_2}{1-n c_1} \right) - \frac{c_2-c_1}{n} \left( \frac{c_1+c_2}{2} + \frac{1}{n} \right) \right].\end{split}
\end{equation*}
The dispersion relation is given by
\begin{equation*}
\begin{split}\omega = RHS(c_1,c_0,g_1) + RHS(c_0,c_2,g_2).\end{split}
\end{equation*}
Then we parametrically plot \(\{ n \omega(n), \omega(n)\}\) to get the dispersion relation, for MAA solution. Similarly I can find that of MZA solution.

\begin{sphinxadmonition}{note}{Limits}

Before we do any numerical calculations, we can calculate the limits first.
\begin{equation*}
\begin{split}\lim_{n\to\infty}\{n\omega(n),\omega(n)\}  &= \{( g_1(c_1^2-c_0^2) + g_2(c_0^2-c_2^2) + 2 g_1 \ln \left\vert\frac{c_0}{c_1}\right\vert + 2 g_2\ln \left\vert\frac{c_2}{c_0}\right\vert )/8, 0  \}, &\qquad \text{for MAA solution} \\
\lim_{n\to\infty}\{n\omega(n),\omega(n)\}  &= \{ ( g_1(c_1^2-c_0^2) + g_2(c_0^2-c_2^2) - 2 g_1 \ln \left\vert\frac{c_0}{c_1}\right\vert - 2 g_2\ln \left\vert\frac{c_2}{c_0}\right\vert )/4, 0  \}, &\qquad \text{for MZA solution}.\end{split}
\end{equation*}
For \(g_1=g_2=1\) these limit match the homogeous result, which they should.

We also have the large k limit which are \(\omega = \frac{1}{c_i}k\).
\end{sphinxadmonition}

For simplicity, we choose \(g_1=-g_2=1\).
\begin{figure}[htbp]
\centering
\capstart

\noindent\sphinxincludegraphics{{DR-omega-k-direct-continuous-two-regions-dr-crossing}.png}
\caption{Dispersion relation for spectral crossing. The discontinuties are at \(\{ 0.0944205, 0 \}\) (MAA solution) and \(\{  -0.098841, 0 \}\) (MZA solution).}\label{\detokenize{collective/dispersion-relation:id11}}\end{figure}

I can also plot the MAA and MZA soltions for \(\omega(n)\).
\begin{figure}[htbp]
\centering
\capstart

\noindent\sphinxincludegraphics{{DR-omega-k-direct-continuous-two-regions-maa-crossing}.png}
\caption{\(\omega(n)\) for MAA solution. The vertical grid lines are \(n=1/c_1, 1/c_2\).}\label{\detokenize{collective/dispersion-relation:id12}}\end{figure}
\begin{figure}[htbp]
\centering
\capstart

\noindent\sphinxincludegraphics{{DR-omega-k-direct-continuous-two-regions-mza-crossing}.png}
\caption{\(\omega(n)\) for MZA solution. The vertical grid lines are \(n=1/c_1, 1/c_2\).}\label{\detokenize{collective/dispersion-relation:id13}}\end{figure}

The reason we have no real values between \(1/0.9\) and \(1/0.3\) is because the argument of the ln function is negative within this regime.
\begin{figure}[htbp]
\centering
\capstart

\noindent\sphinxincludegraphics{{DR-omega-k-direct-continuous-two-regions-arg-ln}.png}
\caption{The argument of the ln function. The vertical grid lines are \(n=1/0.3,1/0.6,1/0.9\). Even \(n=1/0.6\) gives us the zero argument, which means the ln function is infinite, I do not think this is some point that we can have a \(\omega,k\) relation.}\label{\detokenize{collective/dispersion-relation:id14}}\end{figure}

\begin{sphinxadmonition}{note}{Some Discussions about \(\omega(n)\)}

It seems that if we plot \(-\omega(n)\) for MZA we will have some kind of similar plot compared to MAA. Well the singularities are all the same location. If we do that we will have the same \(\lvert\omega\rvert\) for \(n\) very close to the sigularities thus same \(\lvert k\rvert\). Of couse for MAA and MZA, \(\omega\) have different signs.

In other words, the slopes of \(\omega(k)\) will have the same value for MZA and MAA but in different quadrant.
\end{sphinxadmonition}


\subsection{General Discussions of Significance of Spectra}
\label{\detokenize{collective/dispersion-relation:general-discussions-of-significance-of-spectra}}\begin{figure}[htbp]
\centering
\capstart

\noindent\sphinxincludegraphics{{omega-of-n-for-different-spectra}.png}
\caption{Function \(\omega(n)\) for different \(g_1,g_2\). The emission was set to \(G=\begin{cases}
g_1, \qquad & \theta_1<\theta<\theta_0 \\
g_2, \qquad & \theta_0<\theta<\theta_2
\end{cases}\) where \(\cos\theta_1=0.9\) and \(\cos\theta_2=0.3\) \(\cos\theta_0=0.6\)}\label{\detokenize{collective/dispersion-relation:id15}}\end{figure}
\begin{figure}[htbp]
\centering
\capstart

\noindent\sphinxincludegraphics{{omega-of-n-for-different-spectra-c0-0.4}.png}
\caption{Function \(\omega(n)\) for different \(g_1,g_2\). The emission was set to \(G=\begin{cases}
g_1, \qquad & \theta_1<\theta<\theta_0 \\
g_2, \qquad & \theta_0<\theta<\theta_2
\end{cases}\) where \(\cos\theta_1=0.9\) and \(\cos\theta_2=0.3\) \(\cos\theta_0=0.4\)}\label{\detokenize{collective/dispersion-relation:id16}}\end{figure}

\begin{sphinxadmonition}{note}{Why}

Can I derive some expression for the \(\ln\) function for a continuous distribution \(G(\theta)\)?
\end{sphinxadmonition}


\subsection{Discrete Case and Continuous Case}
\label{\detokenize{collective/dispersion-relation:discrete-case-and-continuous-case}}
Will we have a continuous case if the number of beams is infinite.

For discrete case
\phantomsection\label{\detokenize{collective/dispersion-relation:equation-eqn-discrete-beams-i0-i2}}\begin{equation}\label{equation:collective/dispersion-relation:eqn-discrete-beams-i0-i2}
\begin{split}I_0 - I_2 = \sum_{i=1}^N G_i \frac{ 1-\cos^2\theta_i }{\omega - k \cos \theta_i} = \sum_i G_i \frac{1-u_i^2}{\omega-k u_i}.\end{split}
\end{equation}
The continuous case is
\phantomsection\label{\detokenize{collective/dispersion-relation:equation-eqn-conti-beams-i0-i2}}\begin{equation}\label{equation:collective/dispersion-relation:eqn-conti-beams-i0-i2}
\begin{split}I_0 - I_2 = \int d\cos\theta  G(\cos\theta) \frac{1 - \cos^2\theta}{\omega- k \cos\theta} = \int du G(u) \frac{1-u^2}{\omega-k u} .\end{split}
\end{equation}
We notice that Eq. \eqref{equation:collective/dispersion-relation:eqn-discrete-beams-i0-i2} and Eq. \eqref{equation:collective/dispersion-relation:eqn-conti-beams-i0-i2} are the same when number of beams becomes large.

\(G_i\) is in fact \(G_i = g_i \Delta u_i\), where \(\Delta u_i\) is the range of \(\cos \theta_i\) around \(\theta_i\).


\chapter{Gravity}
\label{\detokenize{gravity/index::doc}}\label{\detokenize{gravity/index:gravity}}

\section{Equation of Motioin}
\label{\detokenize{gravity/equation-of-motion:equation-of-motioin}}\label{\detokenize{gravity/equation-of-motion::doc}}\begin{itemize}
\item {} 
Understand the equation of motion for neutrinos in curved spacetime.

\end{itemize}


\subsection{Equation of Motion}
\label{\detokenize{gravity/equation-of-motion:equation-of-motion}}

\subsection{Refs \& Notes}
\label{\detokenize{gravity/equation-of-motion:refs-notes}}

\section{Gravitational Waves}
\label{\detokenize{gravity/gravitational-waves:gravitational-waves}}\label{\detokenize{gravity/gravitational-waves::doc}}
\begin{sphinxadmonition}{note}{Conventions}
\begin{enumerate}
\item {} 
We use signature \(+2\) for the metric.

\item {} 
\(\omega,k\) are the frequencies from Fourier expansion of the off diagonal element of density matrix.

\end{enumerate}
\end{sphinxadmonition}


\subsection{Basics}
\label{\detokenize{gravity/gravitational-waves:basics}}
In general, we write down the components of metric for weak gravitational field as
\begin{equation*}
\begin{split}g_{\alpha\beta} = \eta_{\alpha\beta} + h_{\alpha\beta},\end{split}
\end{equation*}
where \(h_{\alpha\beta}\) is the perturbation.

In general, a weak gravitational wave has two different modes, the \(h_+\) mode \phantomsection\label{\detokenize{gravity/gravitational-waves:id1}}{\hyperref[\detokenize{gravity/gravitational-waves:schutz}]{\sphinxcrossref{{[}Schutz{]}}}},
\begin{equation*}
\begin{split}h_{\alpha\beta} \to \begin{pmatrix}
0 & 0 & 0 & 0 \\
0 & h_+ & h_\times & 0 \\
0 & h_\times & -h_+ & 0 \\
0 & 0 & 0 & 0
\end{pmatrix},\end{split}
\end{equation*}
where
\begin{equation*}
\begin{split}h_+ &= A_{xx} \cos(\omega_{gw}(t-z))\\
h_\times &= A_{xy} \cos(\omega_{gw}(t-z)).\end{split}
\end{equation*}
\begin{sphinxadmonition}{note}{Binary Neutron Stary System}

In a binary neutron star system, which we detect at a distance \(r\), the gravitational waves carries the metric \phantomsection\label{\detokenize{gravity/gravitational-waves:id2}}{\hyperref[\detokenize{gravity/gravitational-waves:hendry2007}]{\sphinxcrossref{{[}Hendry2007{]}}}}
\begin{equation*}
\begin{split}h_{xx} &= -h_{yy} =h\cos(4\pi f t)\\
h_{xy}&=h_{yx} = - h\sin (4\pi f t),\end{split}
\end{equation*}
where
\begin{equation*}
\begin{split}h = \frac{ 32\pi^2 G M R^2 f^2 }{c^4 r}.\end{split}
\end{equation*}\end{sphinxadmonition}

For simplicity, we ignore the cross terms and consider only the diagonal elements in perturbation, aka \(h_+\) polarization, the metric should be
\begin{equation*}
\begin{split}g_{\alpha\beta}\to \begin{pmatrix}
-1 & 0 & 0 & 0 \\
0 & 1 + h_+ & 0 & 0 \\
0 & 0 & 1-h_+ & 0 \\
0 & 0 & 0 & 1
\end{pmatrix}.\end{split}
\end{equation*}
where
\begin{equation*}
\begin{split}h_+ = A_{xx} \cos(\omega_{gw}(t-z)).\end{split}
\end{equation*}

\subsection{Some Estimations}
\label{\detokenize{gravity/gravitational-waves:some-estimations}}

\subsubsection{Magnitude of Strain}
\label{\detokenize{gravity/gravitational-waves:magnitude-of-strain}}
As an estimation, the power of gravitation waves drop as \(1/r^2\). Thus the strain drops as \(1/r\).

For compact binary coalescence, the circular polarized wave at distance r is \phantomsection\label{\detokenize{gravity/gravitational-waves:id3}}{\hyperref[\detokenize{gravity/gravitational-waves:riles2013}]{\sphinxcrossref{{[}Riles2013{]}}}}
\begin{equation*}
\begin{split}h(t) =  \frac{1}{r} \left( \frac{5 G^5 M^5}{2 c^{11}} \right)^{1/4} \frac{1}{(t_{\mathrm{coal}} -t)^{1/4}},\end{split}
\end{equation*}
which shows that the frequency diverges at \(t_{\mathrm{coal}}\).

In the astrophysicists' form, we have \phantomsection\label{\detokenize{gravity/gravitational-waves:id4}}{\hyperref[\detokenize{gravity/gravitational-waves:riles2013}]{\sphinxcrossref{{[}Riles2013{]}}}}
\begin{equation*}
\begin{split}h(\tau) = (1.7\times 10^{-23}) \left( \frac{15\mathrm{Mpc}}{r} \right) \left( \frac{1 \text{ day} }{\tau} \right)^{1/4} \left( \frac{M}{1.4M_{\odot}}\right)^{5/4}.\end{split}
\end{equation*}
As an estimation, we can show that at \(r=1000 km\) from the source,
\begin{equation*}
\begin{split}h(\tau) \sim 10^{-6}\left( \frac{1 \text{ day} }{\tau} \right)^{1/4} \left( \frac{M}{1.4M_{\odot}}\right)^{5/4}.\end{split}
\end{equation*}
As an example, we plot the evolution as a function of \(\tau\) for 1.4 solar mass binaries.
\begin{figure}[htbp]
\centering
\capstart

\noindent\sphinxincludegraphics{{binary-mergers-h-tau}.png}
\caption{Binary mergers. In the beginning, gravitational waves are very strong.}\label{\detokenize{gravity/gravitational-waves:id11}}\end{figure}

\begin{sphinxadmonition}{note}{Strong Field}

However, the problem is, for stong waves, this estimation fails. I need to find a paper that numerically calculates the strain for strong fields.

\sphinxstylestrong{And I haven't found the right paper.}
\end{sphinxadmonition}

The size of the disk of neutron star mergers are of the order \(10 km\) \phantomsection\label{\detokenize{gravity/gravitational-waves:id5}}{\hyperref[\detokenize{gravity/gravitational-waves:foucart2012}]{\sphinxcrossref{{[}Foucart2012{]}}}}.


\subsubsection{Some Scales}
\label{\detokenize{gravity/gravitational-waves:some-scales}}\begin{enumerate}
\item {} 
Vacuum frequency: \(\omega_v = 3.75\times 10^{-11}eV \frac{ \delta m^2 }{ 7.5\times 10^{-5} \mathrm{eV^2} } \frac{1\mathrm{MeV} }{E}\).

\item {} 
Vacuum frequency corresponds to oscillation length scale of \(L_v \sim 197\times 10^6\times \frac{ 1 }{ \omega_v } \approx 10^1 \mathrm{km}\).

\item {} 
The gravitational wave frequency are usually of the order kHz \phantomsection\label{\detokenize{gravity/gravitational-waves:id6}}{\hyperref[\detokenize{gravity/gravitational-waves:bauswein2016}]{\sphinxcrossref{{[}Bauswein2016{]}}}}. Such a frequency corresponds to a periodic potential in space, which has a length scale of \(L_{GW} \sim 3\times 10^{5}\mathrm{km/s} \times 10^{-3} \mathrm{s}\sim 10^2 \mathrm{km}\).

\end{enumerate}
\begin{figure}[htbp]
\centering
\capstart

\noindent\sphinxincludegraphics{{gw-frequency-neutron-star-mergers}.png}
\caption{Gravitational wave frequencies and their strains. \label{\detokenize{gravity/gravitational-waves:id7}}{\hyperref[\detokenize{gravity/gravitational-waves:bauswein2016}]{\sphinxcrossref{{[}Bauswein2016{]}}}}}\label{\detokenize{gravity/gravitational-waves:id12}}\end{figure}


\subsection{Build a Model}
\label{\detokenize{gravity/gravitational-waves:build-a-model}}
I need a practical model to demonstrate the effect and be able to numerically solve it up to nonlinear regime of neutrino oscillations.

In any case, we could linearize the equation of motion and explore the linear regime. With the help of dispersion relation, the linear regime can be analyzed.

In principle, the equation of motion is changed due to gravitational field with derivatives becoming covariant ones. However, for fast neutrino oscillations, we can consider the local effect by ignoring the connections. I have to replace the Minkowski metric with the metric of gravitational waves. Without any calculation, I expect gravitational waves breaks the symmetrics intrinsically. For example it breaks the degeneracy of the MAA solution for axial symmetric system.

\begin{sphinxadmonition}{note}{Shashank's Comment}

He said that a possible resonance could bring in other effects.

\sphinxstylestrong{Update (20/03/2017)}:

I think this is a very nice point. If we write down the Dirac equation for neutrino oscillations in matter \phantomsection\label{\detokenize{gravity/gravitational-waves:id8}}{\hyperref[\detokenize{gravity/gravitational-waves:cardall1996}]{\sphinxcrossref{{[}Cardall1996{]}}}},
\begin{equation*}
\begin{split}[\gamma^\mu(\partial_\mu + i A_{f\mu} \mathscr P_L) + M_f] \psi_f = 0,\end{split}
\end{equation*}
where \(A_{f\mu}\) is the neutrino-matter interaction.

For stimulated oscillations, this ``potential'' \(A_{f\mu}\) is periodic.

The vacuum oscillation in gravitational field is determined by \phantomsection\label{\detokenize{gravity/gravitational-waves:id9}}{\hyperref[\detokenize{gravity/gravitational-waves:cardall1996}]{\sphinxcrossref{{[}Cardall1996{]}}}}
\begin{equation*}
\begin{split}[\gamma^\mu e^\mu_a (\partial_\mu + \Gamma_{\mu})+M]\psi=0,\end{split}
\end{equation*}
where \(\Gamma_\mu\) is the so called spin connection, which has the form
\begin{equation*}
\begin{split}\Gamma_\mu = \frac{1}{8} [\gamma^b,\gamma^c] e^\nu_b e_{c\nu;\mu}.\end{split}
\end{equation*}
Cardall and Fuller calcualted the contribution from gravity
\begin{equation*}
\begin{split}\gamma^a e^\mu_a \Gamma_\mu = \gamma^a e_a^\mu \left[ i A_{G\mu} \left( - (-g)^{-1/2} \frac{\gamma_5}{2} \right) \right],\end{split}
\end{equation*}
where
\begin{equation*}
\begin{split}A_G^\mu = \frac{1}{4} \sqrt{-g} e_a^\mu \epsilon^{abcd} (e_{b\nu,\sigma} - e_{b\sigma,\nu})e^\nu_{c} e^\sigma_d.\end{split}
\end{equation*}
For gravitational waves, it effectively provides a potential that is periodic since \(-g\) has periodic components.

As for length scales, vacuum oscillations has length scale \(10\,\mathrm{km}\), while GW from neutron star mergers has length scale \(10^2\,\mathrm{km}\). They are not exactly of the same order.

Another point is that matter effect increases the oscillation length scales.
\end{sphinxadmonition}


\subsection{Resonances}
\label{\detokenize{gravity/gravitational-waves:resonances}}
For resonance of linear EoM, we do not need to consider neutrino self-interactions.

\begin{sphinxadmonition}{note}{What's the EoM}

We need a Schrodinger equation formalism instead of the Dirac equation one.
\end{sphinxadmonition}


\subsection{Nonlinear Effect}
\label{\detokenize{gravity/gravitational-waves:nonlinear-effect}}
To consider the nonlinear effect on neutrino flavor conversions, we can apply linear stability analysis.


\subsubsection{Polarization Tensor}
\label{\detokenize{gravity/gravitational-waves:polarization-tensor}}
I can use polarization tensor to solve the linear regime \phantomsection\label{\detokenize{gravity/gravitational-waves:id10}}{\hyperref[\detokenize{gravity/gravitational-waves:izaguirre2017}]{\sphinxcrossref{{[}Izaguirre2017{]}}}}. This method is nothing different from solving the linearized EoM for k and finding the imaginary part in it.

\begin{sphinxadmonition}{note}{Comments on Dispersion Relation}

Any dispersion relation \(f(\omega,k)=0\) indicates whether it is possible to have imaginary parts in \(\omega,k\).

What we usually do is to set \(\omega=0\) and find \(k\).

The equation of motion is simply of the form
\begin{equation*}
\begin{split}v^\mu k_\mu Q = v^\mu a_\mu,\end{split}
\end{equation*}
where \(Q\) is the amplitude of Fourier mode. So in principle we could simply change all the Minkowski metric to the one with perturbation.

There is another concern. The integrals also depends on the metric. But it is of a smaller effect. We need to prove this/make sense of it.
\end{sphinxadmonition}


\subsubsection{Gravitational Waves + Mode}
\label{\detokenize{gravity/gravitational-waves:gravitational-waves-mode}}
\begin{sphinxadmonition}{note}{Assumptions}

Assume the EoM derived by Raffelt is valid for weak gravitational field, as explained above.

As an approximation, we do not consider other effects on Schrodinger equation but only the change in distances.
\end{sphinxadmonition}

The polarization tensor
\begin{equation*}
\begin{split}\Pi^\mu_\nu = g^\mu_\nu + \int \frac{d\Gamma}{4\pi} \frac{v^\mu v_\nu}{g_{\mu\nu}k^\mu v^\nu}.\end{split}
\end{equation*}
Since we choose to calculate the dispersion relation in \(k_z\) direction, \(g_{\mu\nu}k^\mu v^\nu=\omega - g_{33} k^z v^z\).

The first situation we demonstrate is for gravitational waves propagating in \(z\) direction. At a certain time and z, we can write down the dispersion relation,
\begin{equation*}
\begin{split}I + \begin{pmatrix}
\frac{1}{2} I_0 & 0 & 0 & -\frac{1}{2}I_1\\
0 & -\frac{1}{4}(1+h_+) (I_0-I_2) & 0  & 0 \\
0 & 0 & -\frac{1}{4}(1+h_+) (I_0-I_2) & 0  \\
\frac{1}{2}I_1 & 0 & 0 & -\frac{1}{2}I_2
\end{pmatrix},\end{split}
\end{equation*}
where \(h_+\sim h_0 \cos (\omega_{gw}(t-z))\) is small, so we expect small effect from + mode. At least we do not expect something completely different.


\subsubsection{Gravitational Waves x Mode}
\label{\detokenize{gravity/gravitational-waves:gravitational-waves-x-mode}}
The x mode will bring in cross terms. The polarization tensor becomes
\begin{equation*}
\begin{split}I + \begin{pmatrix}
\frac{1}{2} I_0 & 0 & 0 & -\frac{1}{2}I_1\\
0 & -\frac{1}{4}(1+h_+) (I_0-I_2) & \frac{1}{4} h_\times (I_0-I_2)  & 0 \\
0 & \frac{1}{4}h_\times (I_0 - I_2) & -\frac{1}{4}(1+h_+) (I_0-I_2) & 0  \\
\frac{1}{2}I_1 & 0 & 0 & -\frac{1}{2}I_2
\end{pmatrix}.\end{split}
\end{equation*}
To look at the MAA solution, we need to write down the eigenvalues for the 2 by 2 matrix in the center. We fine the relation between \(\omega\) and \(k\) is
\begin{equation*}
\begin{split}4 &= -(1-h_+ - h_\times) (I_0-I_2)\\
4 &= -(1-h_+ + h_\times) (I_0-I_2).\end{split}
\end{equation*}
For neutron star mergers, \(h_\times=-h_+\). The first solution is reduced to the flat space time solution.

\begin{sphinxadmonition}{note}{A Lot More to Think abut}

How can I make sure I am can use this method?

Even the previous calculations are valid, gravitational waves seems to break the symmetries in the emission surface. The question to ask is \sphinxstylestrong{the effect of breaking emission surface symmetry anyway}.
\end{sphinxadmonition}


\subsection{References and Notes}
\label{\detokenize{gravity/gravitational-waves:references-and-notes}}
Kip Thorne wrote a review paper about gravitational waves: \href{https://www.its.caltech.edu/~kip/scripts/PubScans/II-68.pdf}{The Generation of Gravitational Waves: A Review of Computational Tecniques}.

A paper about modern techniques: \href{http://www.nikhef.nl/pub/services/biblio/theses\_pdf/thesis\_T\_G\_F\_Li.pdf}{Extracting Physics from Gravitational Waves}.


\chapter{Group Meeting}
\label{\detokenize{group-meeting/index:group-meeting}}\label{\detokenize{group-meeting/index::doc}}

\section{Group Meeting 2017}
\label{\detokenize{group-meeting/2017/index:group-meeting-2017}}\label{\detokenize{group-meeting/2017/index::doc}}

\subsection{2017-03 Group Meeting}
\label{\detokenize{group-meeting/2017/2017-03:group-meeting}}\label{\detokenize{group-meeting/2017/2017-03::doc}}

\subsubsection{2017-03-09}
\label{\detokenize{group-meeting/2017/2017-03:id1}}
Topic: Dispersion Relation of Neutrino Flavor Oscillations

About numerical calculations
\begin{enumerate}
\item {} 
My \(\omega\) and \(k\) have a minus sign compared to Raffelts result.

\item {} 
Not exactly the same plot. How can I determine the scaling of spectrum. In principle I have the same definition of \(G\). However, it definitely is not the case when comparing the result.

\end{enumerate}

About analytical analysis
\begin{enumerate}
\item {} 
I should have the discrete case approaching the continuous case when the number of beams becomes large.

\end{enumerate}


\subsubsection{2017-03-16}
\label{\detokenize{group-meeting/2017/2017-03:id2}}\begin{enumerate}
\item {} 
Solved the continuous case but the result is weird, both for homogeneous and nonhonogeous case.

\end{enumerate}

\begin{sphinxthebibliography}{HyperbolaWikipedia}
\bibitem[shigeyama1990]{\detokenize{shigeyama1990}}{\phantomsection\label{\detokenize{matter-stimulated/scales:shigeyama1990}} 
Shigeyama, T., \& Nomoto, K. (1990). Theoretical light curve of SN 1987A and mixing of hydrogen and nickel in the ejecta. The Astrophysical Journal, 360, 242. doi:10.1086/169114
}
\bibitem[EBorriello2014]{\detokenize{EBorriello2014}}{\phantomsection\label{\detokenize{matter-stimulated/scales:eborriello2014}} 
Borriello, E., Chakraborty, S., Janka, H.-T., Lisi, E., \& Mirizzi, A. (2014). Turbulence patterns and neutrino flavor transitions in high-resolution supernova models. Journal of Cosmology and Astroparticle Physics, 2014(11), 030\textendash{}030. doi:10.1088/1475-7516/2014/11/030
}
\bibitem[Krastev1989]{\detokenize{Krastev1989}}{\phantomsection\label{\detokenize{matter-stimulated/parametric-resonance-revisted:krastev1989}} 
Krastev, P. I., \& Smirnov, A. Y. (1989). Parametric effects in neutrino oscillations. Physics Letters B, 226(3-4), 341\textendash{}346. doi:10.1016/0370-2693(89)91206-9
}
\bibitem[Ploumistakis2009]{\detokenize{Ploumistakis2009}}{\phantomsection\label{\detokenize{matter-stimulated/single-frequency:ploumistakis2009}} \begin{enumerate}
\item {} 
Ploumistakis, S.D. Moustaizis, I. Tsohantjis, Towards laser based improved experimental schemes for multiphoton pair production from vacuum, Physics Letters A, Volume 373, Issue 32, 3 August 2009, Pages 2897-2900, ISSN 0375-9601, \url{http://dx.doi.org/10.1016/j.physleta.2009.06.015}.

\end{enumerate}
}
\bibitem[Izaguirre2016]{\detokenize{Izaguirre2016}}{\phantomsection\label{\detokenize{collective/directional-instability:izaguirre2016}} 
Izaguirre, I., Raffelt, G., \& Tamborra, I. (2016). \href{http://arxiv.org/abs/1610.01612}{Fast Pairwise Conversion of Supernova Neutrinos: Dispersion-Relation Approach}, 1\textendash{}6.
}
\bibitem[HyperbolaWikipedia]{\detokenize{HyperbolaWikipedia}}{\phantomsection\label{\detokenize{collective/dispersion-relation:hyperbolawikipedia}} 
\href{https://en.wikipedia.org/wiki/Hyperbola\#Quadratic\_equation}{Hyperbola @ Wikipedia}
}
\bibitem[Schutz]{\detokenize{Schutz}}{\phantomsection\label{\detokenize{gravity/gravitational-waves:schutz}} 
A First Course in General Relativity, Bernard Schutz.
}
\bibitem[Hendry2007]{\detokenize{Hendry2007}}{\phantomsection\label{\detokenize{gravity/gravitational-waves:hendry2007}} 
\href{http://star-www.st-and.ac.uk/~hz4/gr/hendry\_GRwaves.pdf}{An Introduction to General Relativity, Gravitational Waves and Detection Principles}, Dr Martin Hendry. This discussion about the gravitational waves in a binary neutron star system is for slow motion approximation.
}
\bibitem[Riles2013]{\detokenize{Riles2013}}{\phantomsection\label{\detokenize{gravity/gravitational-waves:riles2013}} \begin{enumerate}
\setcounter{enumi}{10}
\item {} 
Riles, \href{http://dx.doi.org/10.1016/j.ppnp.2012.08.001}{Gravitational waves: Sources, detectors and searches}, Progress in Particle and Nuclear Physics, Volume 68, January 2013, Pages 1-54, ISSN 0146-6410.

\end{enumerate}
}
\bibitem[Foucart2012]{\detokenize{Foucart2012}}{\phantomsection\label{\detokenize{gravity/gravitational-waves:foucart2012}} 
Francois Foucart, \href{http://journals.aps.org/prd/abstract/10.1103/PhysRevD.86.124007}{Black-hole\textendash{}neutron-star mergers: Disk mass predictions}, Phys. Rev. D 86, December 2012.
}
\bibitem[Izaguirre2017]{\detokenize{Izaguirre2017}}{\phantomsection\label{\detokenize{gravity/gravitational-waves:izaguirre2017}} 
Izaguirre, I., Raffelt, G., \& Tamborra, I. (2017). \href{https://doi.org/10.1103/PhysRevLett.118.021101}{Fast Pairwise Conversion of Supernova Neutrinos: A Dispersion Relation Approach}. Physical Review Letters, 118(2), 21101.
}
\bibitem[Cardall1996]{\detokenize{Cardall1996}}{\phantomsection\label{\detokenize{gravity/gravitational-waves:cardall1996}} 
Cardall, C. Y., \& Fuller, G. M. (1996). \href{https://doi.org/10.1103/PhysRevD.55.7960}{Neutrino oscillations in curved spacetime: an heuristic treatment}, 55(12), 7.
}
\bibitem[Bauswein2016]{\detokenize{Bauswein2016}}{\phantomsection\label{\detokenize{gravity/gravitational-waves:bauswein2016}} \begin{enumerate}
\item {} 
Bauswein, J. Clark, N. Stergioulas, H.-T. J. (n.d.). Dynamics and gravitational-wave emission of neutron-star merger remnants, arXiv:1602.00950. Retrieved from https://arxiv.org/abs/1602.00950

\end{enumerate}
}
\end{sphinxthebibliography}



\renewcommand{\indexname}{Index}
\printindex
\end{document}